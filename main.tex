\documentclass[11pt,twoside,openany]{book}

%%% PACKAGES %%%%%%%%%%%%%%%%%%%%%%%%%%%%%%%%%%%%%%%%%%%%%%%%%%%%%%%%%%%%%%%%%%

\usepackage[utf8]{inputenc}
\usepackage[T1]{fontenc}
\usepackage[osf]{ebgaramond}
\usepackage[francais]{babel}
\usepackage{geometry}     % Configuration de la mise en page
\usepackage{microtype}    % Améliorations typographiques
\usepackage{hyperref}     % Liens hypertextes & métadonnées

%%% CONFIGURATION %%%%%%%%%%%%%%%%%%%%%%%%%%%%%%%%%%%%%%%%%%%%%%%%%%%%%%%%%%%%%

\geometry{
  papersize={9.33cm,15cm}, % In-18 carré
  textwidth=5.3cm,
  lines=20,
  headsep=5pt,
  marginparsep=7mm,
}

\frenchsetup{
  AutoSpacePunctuation=false,
  og=«, fg=»,
}
\renewcommand{\frenchcontentsname}{Table}

\hypersetup{
    colorlinks=true,
    linkcolor=black,
    pdftitle={Considérations sur le jeu, les joueurs, la théorie des jeux, etc.},
    pdfauthor={Lablée, Jacques (1751-1841)},
    pdfproducer={Éric Guirbal},
}

% Formatage des paragraphes
\setlength{\parindent}{0.8em}

%%% MACROS POUR LA SAISIE DU TEXTE %%%%%%%%%%%%%%%%%%%%%%%%%%%%%%%%%%%%%%%%%%%%

% Folio du document source dans la marge
\newcommand{\folio}[1]{%
  \marginpar[\raggedleft\scriptsize#1]{\scriptsize#1}%
  \ignorespaces%
}

% Lettrines
\newcommand*{\lettrine}[2]{\noindent{\Large#1}\textsc{#2}}

% Texte entre parenthèses
\newcommand*{\paren}[1]{(~#1~)}


%%% DÉBUT DU DOCUMENT %%%%%%%%%%%%%%%%%%%%%%%%%%%%%%%%%%%%%%%%%%%%%%%%%%%%%%%%%

\begin{document}

\frontmatter

\chapter{Introduction}

\folio{i}
\lettrine{D}{ans} le roman de la Roulette
\footnote{
  La sixième édition de cet ouvrage, avec
  un tableau gravé et colorié, représentant un
  jeu de roulette, se vend chez Rapet, commis%
  sionnaire en librairie, rue Saint-André-des-%
  Arcs, n.~41. Prix : 1~franc 5o~centimes ;  et
  franc de port, 1~fr. 75 cent.},
j'ai parlé le langage du sentiment ;
j'ai tâché d'offrir des tableaux qui
pussent émouvoir l'imagination ;
j'ai peint un joueur en action,
\folio{ii}
entraîné, comme ils le sont pres%
que tous, par des erreurs de cal%
culs, et livrés aux illusions d'une
passion désastreuse. Quel écrivain,
ami de l'humanité et des mœurs,
n'est pas pressé par le besoin d'ar%
rêter dans leur cours les vices et les
excès qui leur portent le plus d'at%
teinte, et d'attirer tous les regards
sur les pièges tendus à la crédulité,
à l'ignorance et à la faiblesse ?

Je vais encore m'occuper du
même sujet : j'en parlerai dans les
\folio{iii}
mêmes sentimens et dans les
mêmes principes ; mais je le con%
sidérerai sous d'autres rapports.
Je réprimerai tout mouvement
passionné ; et, recherchant plus
ce qui est vrai que ce qui peut
faire sensation, je donnerai à mes
idées un développement plus mé%
thodique. Moins jaloux d'émou%
voir, que d'éclairer et de con%
vaincre, je m'adresserai moins au
cœur qu'à l'esprit ; en un mot,
j'écrirai plutôt sur le jeu que con%
tre le jeu. Lorsqu'on suit un pareil
\folio{iv}
plan, si les effets qu'on produit
sont moins brillans et moins vifs,
ils peuvent être plus sûrs et plus
durables.

Certes, je ne serai jamais l'apo%
logiste des goûts et des habitudes
du jeu ; mais il me semble que la
meilleure manière de les com%
battre, n'est pas d'annoncer d'a%
bord le vœu et l'intention de les
détruire ; et s'il est vrai que leur
destruction soit regardée comme
impraticable, n'y a-t-il pas quel%
\folio{v}
que chose de mieux à faire que
d'appeler sur les joueurs et sur les
lieux qui les rassemblent, le mé%
pris et la proscription ? On a su
extraire des sucs bienfaisans de
plantes vénéneuses ; ne peut-on
enlever au jeu ce qu'il a de plus
dangereux et de plus funeste ?
n'en peut-on du moins tirer quel%
ques fruits ? Ce désordre, ces
pertes sont-ils sans dédommage%
mens, sans compensations ? et n'y
a-t-il aucun bien à côté d'un si
grand mal ?

\folio{vi}
Il faut parler aux hommes éga%
rés par des passions un langage
qui leur soit familier, ou qu'ils
puissent entendre ; il faut comp%
ter, pour ainsi dire, avec eux,
dans leurs propres affaires ; ainsi
on se rend maître de leur atten%
tion, ce qui est déjà avoir beau%
coup obtenu : ils peuvent alors
apercevoir eux-mêmes le danger
qui les menaçait, le précipice
dans lequel ils allaient tomber ;
alors il est plus facile de leur faire
quitter la ligne sur laquelle ils
\folio{vii}
étaient placés, et de les attirer sur
un point qui concilie mieux leurs
intérêts et leurs goûts.

Je vais donc tâcher d'alléger
le poids énorme qu'un destin
aveugle fait peser sur les joueurs ;
et en examinant ce qui doit leur
être ôté, et ce qu'il convient en%
core de leur conserver, je m'ap%
plaudirai, si je peux aussi, par de
faibles, mais nouveaux aperçus,
aider l'administration publique à
remplir un de ses devoirs les plus
difficiles.

\folio{viii}
Je garderai le silence sur les
désagrémens et les défaveurs que
m'ont causé mes ouvrages contre
les jeux. Il en coûte souvent pour
faire connaître d'utiles vérités,
mais les écrivains moraux rem%
pliraient-ils leur devoir s'ils ne
savaient faire le sacrifice de leur
intéret personnel ?


\mainmatter

\chapter{
  De l'ouvrage intitulé : \emph{De la Passion
  du Jeu}, par \bsc{Dussaulx}%
}

\folio{1}
\lettrine{L}{e} bon, l'honnête Dussaulx a fait
sur la passion du jeu un traité histo%
rique et moral, qui est ce que nous
avons de plus complet, de mieux
pensé et de mieux écrit sur cette ma%
tière. On y remarque une érudition
\folio{2}
facile, des anecdotes curieuses et ins%
tructives, des réflexions originales,
piquantes et quelquefois profondes,
une sorte de verve poétique, et des
vues portant le cachet d'un bon es%
prit et d'un bon cœur : on partage
la juste indignation qu'excitent dans
l'âme de l'auteur les excès du jeu et
le crime de ceux qui les favorisent ;
mais souvent l'on sourit à sa con%
fiante bonhomie. A-t-il pu croire que
l'énergie des passions cupides céde%
rait à des moralités et à des citations ?
En le lisant avec attention, on doute
qu'il s'en soit flatté ; mais si on n'y
trouvait de ces pensées fortes, de ces
traits qui caractérisent la vraie sen%
sibilité et le besoin de la répandre,
on serait tenté de penser que l'au%
teur a voulu faire plutôt un ouvrage
\folio{3}
savant, curieux, orné des fleurs de
l'éloquence, qu'un ouvrage dont on
pût retirer beaucoup de fruit. On le
voit plus appliqué à peindre le mal
qu'à en indiquer le remède. Lui-%
même il parle de l'inutilité des lois et
des efforts des gouvernemens contre
cette fureur aveugle ; il dit et il prouve
que le jeu a dans tous les lieux et dans
tous les tems subjugué l'esprit des
hommes de toutes les classes. « Par%
courez la terre depuis le Japon
jusqu'à l'extrémité du nouveau
monde, quels que soient le culte,
les lois et les opinions, vous trouve%
rez des joueurs dans les climats
glacés et dans les climats brûlans ».

Voilà ce que dit Dussaulx ; et pour
démontrer par les faits que cette épi%
démie universelle est indestructible,
\folio{4}
je n'aurais besoin que de reproduire
ceux qu'il rapporte.

Je citerai plus d'une fois cet au%
teur, le seul qui chez nous se soit fait
entendre sur cette matière. N'ayant
pour but que d'offrir la vérité, je ne
dois rien négliger de ce qui me semble
propre à la faire connaître. C'est dans
cet esprit et dans cette obligation que
je relèverai aussi les défauts et les er%
reurs qui m'ont frappé dans l'ouvrage
dont il s'agit.

Dussaulx prononce d'abord trop
fortement son intention de peindre
les joueurs et leur manie avec les
couleurs les plus noires ; il commence
par les dévouer à la haine et à la pros%
cription ; et à ce sujet il s'exprime
ainsi : « Si les écrivains ont montré
les côtés séduisans du jeu, ils sont
\folio{5}
des corrupteurs ; s'ils n'en ont ex%
primé que la difformité, ils sont
les vrais amis de l'humanité ».

Aussi représente-t-il souvent les
joueurs moins tels qu'ils sont, que
tels qu'il faut qu'ils soient pour pa%
raître odieux. Voilà bien ce qui con%
vient pour faire briller le talent d'un
écrivain, pour qu'il puisse donner à
ses sentimens un développement éner%
gique ; mais ce n'est pas la meilleure
règle d'instruction. Sans doute c'est
en offrant aux hommes le flambeau
de la vérité, c'est en les éclairant qu'on
sert le mieux leurs intérêts. Si vous ne
leur montrez qu'un côté des choses,
et qu'ils viennent à découvrir celui
que vous voulez leur cacher \paren{ici ils
le découvriront ; ils l'ont même déjà
découvert, car sans cela vous n'au%
\folio{6}%
riez pas de leçons à leur faire}, ils
seront en droit de se plaindre de ce
que vous avez voulu les tromper ; ils
vous accuseront de mauvaise foi, et
n'auront plus de confiance dans ce
que vous leur direz.

Les écrivains moraux ont besoin,
pour fixer nos idées, d'un grand ca%
ractère d'impartialité et de désinté%
ressement. Dussaulx, par la sorte d'en%
gagement qu'il a pris dès le commen%
cement de son livre, manque une
partie des effets qu'il pouvait pro%
duire. On prévient, on devine sa pen%
sée ; on va jusqu'à suspecter la vérité
de ses tableaux ; ce n'est plus qu'un
avocat qui, dans un procès impor%
tant, rassemble tout ce qui lui paraît
favorable à sa cause, crie contre ses
adversaires, exagère ses accusations,
\folio{7}
ses reproches. On l'écoute ; il inté%
resse ; mais pour fixer son opinion,
on attend que ses adversaires lui aient
répondu.

Ainsi, Dussaulx, en voulant trop
prouver les dangers du jeu, a paru
mettre en question une vérité géné%
ralement sentie.

Pour prendre plus d'avantage sur
les joueurs, il en a trop simplifié le
caractère, et il a commis évidemment
une erreur, en considérant moins le
jeu comme une de ces passions inhé%
rentes, pour ainsi dire, à la faiblesse
humaine, et qu'il est plus facile d'é%
nerver, de diriger, que de détruire,
qu'en le considérant comme un vice
absolu, déterminé, sur lequel la loi
pouvait avoir une action directe, ou
auquel on pouvait appliquer, comme
\folio{8}
à des maux connus, des remèdes gé%
néraux. Je ferai voir que le caractère
du Joueur, extremement composé,
tient à différentes causes qui le mo%
difient et qu'il faudrait connaître,
pour être en état d'employer à la gué%
rison du mal des remèdes particuliers.

Mais comment concilier les diffé%
rentes idées que Dussaulx donne du
jeu et des joueurs ?

Il dit : « Il s'agit ici d'un vice pur
et sans mélange. Quel joueur a le
droit de s'estimer ? Un joueur ! ce
titre est une insulte ».

Et ailleurs : « La manie du jeu roule
sur trois pivots, la sottise, la fu%
reur et la fourberie ».

Et ailleurs : « On citerait moins de
joueurs sensibles que de bourreaux
compâtissans ».

\folio{9}
Et ailleurs : « Les joueurs manquent
de sensibilité comme de probité. »

Enfin, avec Aristote, il refuse aux
joueurs toutes les qualités du cœur.

Cependant il met au rang des plus
grands joueurs, les hommes doués
de plus d'imagination ; et parmi ces
grands joueurs, il cite d'excellens
hommes, tels que Caton, Henri~IV,
Montaigne, Descartes, Collardeau et
lui-même.

Il dit aussi que la fureur du jeu,
par un alliage monstrueux, se joint
quelquefois à de grands talens et à
de grandes vertus.

Et il rend encore moins effrayante
la laideur de ses portraits, en obser%
vant que l'ennui fait plus de joueurs
que la cupidité ; que le goût du jeu
\folio{10}
est quelquefois moins un symptôme
de cupidité que d'ambition.

Toutes ces contradictions sont-%
elles assez évidentes ? II est vrai que,
pour se mettre à l'abri du reproche
d'avoir désigné les joueurs par d'o%
dieuses qualifications, il applique,
vers la fin de son livre, ce qu'il en a
dit aux joueurs de profession ; mais
cette explication prudente et tardive
est loin d'être satisfaisante. Dussaulx
n'ignorait pas que les joueurs de pro%
fession n'ont pas de passions, n'ont
pas même de caractère, et que cette
classe est trop peu nombreuse, trop
peu importante, trop peu susceptible
d'impressions morales, pour qu'on
doive prendre la peine de faire pour
elle un gros livre.

Dussaulx, revenant au caractère de
\folio{11}
fourberie qu'il attribue injustement
aux joueurs en général, dit : « Vous
trouverez des joueurs suspects dans
tous les rangs ; parmi les gens de
lettres vous ne verrez que des vic%
times résignées aux caprices du
sort. »

J'observe d'abord que les joueurs
les plus nombreux, ceux du moins
qu'il faut le plus s'attacher à guérir
de leur frénésie, sont ceux qui jouent
aux jeux de hasard : or, d'après les
précautions prises ordinairement par
les banques, un joueur assez adroit
pour être fructueusement un fripon,
est une exception très-rare ; et encore
une fois ce n'est pas pour ceux qui
vont au jeu, avec l'intention d'y vo%
ler, qu'on fait des traités de morale :
ensuite si, par ces mots \emph{victimes ré%
\folio{12}
signées}, Dussaulx a entendu incapa%
bles de fourberie, je crois que d'autres
rangs ont également cet avantage ;
et s'il a entendu, disposées à souffrir
la perte avec patience, j'ai remarqué
que cette résignation se trouvait plus
chez les sots que chez les gens d'es%
prit, dont l'imagination est plus facile
à s'exalter, quoique la réflexion et la
philosophie les modèrent ensuite.

Dussaulx a trop confondu les rap%
ports sous lesquels le jeu peut être
considéré : il devait sans doute pré%
senter séparément l'influence qu'il a
sur les mœurs et sur la fortune pu%
blique, et celle qu'il a sur les mœurs
et sur la fortune des particuliers ;
mais il revient trop fréquemment aux
mêmes idées ; ce qu'on peut attribuer
au défaut d'ensemble de son ouvrage.
\folio{13}
Il me parait au moins que les parties
en sont trop détachées ; que ses ta%
bleaux ne sont pas liés de manière à
soutenir l'intérêt ; que ses raisonne%
mens ne sont pas assez suivis pour
porter dans les esprits cette convic%
tion dont plus de conversions et de
réformes auraient été les résultats.
Sa marche est quelquefois embarras%
sée : on voit qu'il veut dire tout ce
qu'il sait, tout ce qu'il a lu sur les
jeux ; et on peut lui appliquer ce qu'il
a dit de Barbeyrac : « II s'est jeté trop
souvent dans des discussions super%
flues ou étrangères à son sujet ; »
de manière qu'on jugerait difficile%
ment s'il a écrit pour les joueurs ou
pour ceux qui ne le sont pas.

La plupart de ces défauts, cet em%
barras, ces contradictions ne s'ex%
\folio{14}
pliquent-ils pas d'abord par l'incon%
venance de présenter à-la-fois, dans
un ouvrage moral, ce qui devait ne
contenter que la curiosité, avec ce
qui devait servir à l'instruction ? En
outre, l'espèce de nécessité dans la%
quelle l'auteur s'était mis de rendre
odieux et méprisables les joueurs et
les maisons de jeu, l'obligeait de s'é%
carter de tems en tems de la vérité, à
laquelle la franchise de son caractère
le ramenait bientôt.

Quant aux remèdes, Dussaulx,
après avoir cité une foule de traits
qui semblent en montrer l'ineffica%
cité ; après avoir dit qu'on n'a rien à
attendre des Gouvernemens, « tou%
si pauvres, qu'on ne saurait 
compter sur eux, lorsqu'il s'agit
d'argent,» ni sur l'experience, qui
\folio{15}
appartient plus à ceux qui méditent
qu'à ceux qui agissent, et qui est tou%
jours impuissante contre le désir et la
séduction, indique cependant quel%
ques moyens de réforme, tels que les
amusemens naturels, la suspension
de la pratique du jeu par un violent
effort sur soi-même, l'exercice de la
bienfaisance, des entreprises labo%
rieuses, le recours dans le sein d'une
sage et prudente amitié. Voilà ce qu'il
conseille aux joueurs eux-mêmes.
Sans doute ces moyens sont salutaires,
et leur indication seule prouve la
bonté, la candeur d'ame de l'auteur,
dans l'ouvrage de qui je ne relève qu'à
regret, parmi des beautés et des vé%
rités du premier ordre, ce que je
regarde comme des défauts ou des
erreurs ; mais plusieurs de ces moyens
\folio{16}
ne sont-ils pas hors du pouvoir des
uns ? Ne sont-ils pas insuffisans pour
les autres ? Et qui donnera aux joueurs
la force d'exécution sans laquelle la
force de volonté n'est rien ? Une
grande passion se guérit-elle avec des
calmans ? Et si ce joueur croit voir
dans ce qu'il va faire un avantage sûr
et prochain, l'en détournerez-vous
en lui offrant, dans ce que vous lui
proposez, un avantage douteux et éloigné ?

Dussaulx expose encore quelques
autres moyens de réforme ; mais
ceux-ci il les fait dépendre de l'auto%
rité. Ce sont les refus d'honneurs et
d'emplois aux joueurs incorrigibles,
l'obligation aux joueurs fortunés de
nourrir des vieillards, des pauvres,
des infirmes \paren{je n'ai pas besoin de
\folio{17}
dire quels maux pourraient résulter
de l'erreur ou de la mauvaise foi de
rapports faits à l'autorité sur la con%
duite des particuliers qui d'ailleurs
auraient tant de moyens de dérober
à ses agens la connaissance de leurs
gains ou de leurs excès}, la suppres%
sion des jeux d'état, l'abolition des
privilèges de jeux, la réformation des
mœurs, l'éducation.

Les lecteurs ayant de l'expérience
et de l'instruction, apprécieront, je
ne dirai pas les remèdes découverts,
mais les vœux formés par Dussaulx :
j'y reviendrai en m'occupant aussi
des moyens praticables de réforme
ou d'amélioration.

J'ai rempli un devoir pénible en 
manifestant mon opinion sur l'insuf%
fisance de l'ouvrage de Dussaulx con%
\folio{18}
tre la passion et les excès du jeu. Je
le répète, je n'en reconnais pas moins
le mérite supérieur de cet ouvrage,
et j'en recommande vivement la lec%
ture aux personnes malheureuses ou
trompées que le jeu entraîne ou sé%
duit. Je n'ai point l'orgueilleuse pré%
tention de le refaire ; mais en traitant
beaucoup moins d'objets, et me ren%
fermant dans un cadre étroit, je veux
rechercher s'il n'y aurait pas de nou%
velles lumières à répandre sur ce
sujet. D'ailleurs, l'état des jeux n'est
pas aujourd'hui ce qu'il était lorsque
Dussaulx a écrit.

\chapter[Du Jeu]{DU JEU}

\folio{19}
\lettrine{L}{e} jeu, fruit de l'amour et du plaisir,
et aussi variable, ne fut d'abord qu'un
exercice agréable ou salutaire de l'es%
prit ou du corps ; il n'est pas d'autre
chose pour beaucoup de personnes.

Si l'on fait attention à la manière
dont se développent les facultés intel%
lectuelles de l'homme ou de tout être
vivant, on verra que, presque dès sa
naissance, il joue avec des objects pu%
rement physiques ou avec des êtres
animés : s'il joue avec des objets pure%
ment physiques, il ne tarde pas à se
lasser de celui qui l'occupe ; il le quitte
pour en prendre un autre qu'il va quit%
\folio{20}
ter à son tour : s'il joue avec des êtres
animés, sur-tout ceux de même nature
que la sienne, son action est plus vive,
sa gaîté plus bruyante, son attache%
ment plus prolongé. Ces mouvemens,
d'abord vagues et irréguliers, lorsque
l'intelligence se forme, et que la joie est
partagée, acquièrent insensiblement
de la règle et de la mesure. L'attrait du
jeu n'est encore que l'attrait du plaisir.
Le talent du joueur est l'habileté, la
ruse, l'adresse ou l'industrie. Le jeu
consiste à faire des sons, courir, s'éle%
ver, atteindre un but, prévenir ou re%
pousser une attaque, saisir prompte%
ment un objet idéal ou matériel ; enfin,
il se compose suivant le goût de celui
qui s'y livre, et offre presque toujours
une difficulté à vaincre. Un prix est
donné à celui qui l'a vaincue ; c'est une
\folio{21}
fleur, un fruit, un sourire, un baiser :
ce prix tente celui qui ne l'a point ob%
tenu ; celui qui en a remporté un pre%
mier veut en remporter un second ;
l'amour-propre est piqué ; l'émulation
naît et est excitée ; les défis se propo%
sent ; on n'aspire plus après des baga%
telles ; la nature du prix a changé ; celle
du jeu n'a plus la même simplicité : elle
se varie, elle se complique ; les inéga%
lités de force ou de talent, le doute,
les diverses interprétations font naître
les disputes ; l'adresse, l'industrie ins%
pirent du découragement ou de la dé%
fiance ; on leur associe une puissance
aveugle, le sort, qui agit tantôt avec
elles, tantôt sans elles : l'ignorant
s'étonne de son savoir ; le faible de sa
force ; l'infortuné de ses ressources ; le
téméraire de son triomphe. Le joueur,
\folio{22}
dans sa joie, croit que le sort a des
yeux, puisqu'il le favorise. Bientôt ce
tyran, interrogé de toutes parts, ras%
semble autour de lui la foule de ses
favoris, même celle de ses victimes.
Ses arrêts sont prompts, ses faveurs
faciles. L'ennui, la paresse, l'ambition
assiégent ses portes ; et les plus aima%
ble enchanteresses, l'espérance et
l'imagination, sont là, qui rassurent
les timides, flattent les orgueilleux,
consolent les mécontens et ramènent 
les fugitifs.

Déjà l'ardeur du jeu, celle des pas%
sions cupides que la plupart des hom%
mes éprouvent la première, fait naître
ou met les autres en mouvement ; et
le monde habité est infecté d'un vice
d'autant plus funeste, d'autant plus
contagieux, qu'il s'embellit toujours
\folio{23}
du nom, de l'éclat et de la séduction
du plaisir.

Tels me paraissent être les commen%
cemens, les progrès, les variations de
ce qu'on appelle le jeu. Est-il donc né%
cessaire de fouiller dans les annales de
l'antiquité, pour découvrir son ori%
gine et étudier son histoire ? Si nous
consultons le livre de la nature, qui
nous est toujours ouvert, nous ne dou%
tons pas que les passions de l'homme
n'aient eu, ainsi que sa figure, dans
tous les lieux et dans tous les tems, à-%
peu-près les mêmes traits, le même
caractère. Les différens climats, les
lois, les mœurs, les usages, mettent
peu de différence dans leur dévelop%
pement et dans les excès auxquels
elles conduisent. C'est un fleuve ra%
pide dont on peut prévenir l'entière
\folio{24}
corruption, mais qu'il est aussi diffi%
cile d'épurer que d'en arrêter le cours.

Le jeu est pur dans sa source. Mais
de quoi n'abuse-t-on pas ! Les excès
ont lieu jusque dans le travail.

\frquote{Si la fureur du jeu, dit Dussaulx,
est universelle en France, c'est parce
qu'une corruption générale est im%
punie ; c'est parce que l'amour des
richesses l'emporte sur l'honneur,
à mesure que les empires vieillis%
sent \footnote{
  Grande vérité par laquelle s'expliquent
  les désordres dont nous avons tant à gémir.
}.

Le mal existe sans qu'on puisse
en accuser personne.}

Ce que j'ai dit du développement de
ce goût naturel qui nous porte vers le
jeu, a son application chez les peuples
sauvages comme chez les peuples civi%
\folio{25}
lisés. Le sauvage, en se mettant à la
merci du sort par des règles précises
et déterminées, en même tems qu'il 
prouve son ignorance, prouve qu'il a
fait un pas de plus vers la civilisation ;
et il est aisé de remarquer ici que la
passion du jeu réunit la sagacité à
l'aveuglement.

Presque partout le jeu a été la re%
présentation des combats. Les hom%
mes, naturellement imitateurs, et en%
clins à engager des luttes, se dédom%
magent, dans cette autre guerre, des
langueurs d'un honteux repos. On
peut régler leurs mouvemens ; mais
arrêtez-les dans leurs courses !

Ce besoin de jouer qui se manifeste
dès l'enfance, et fait contracter de
douces et fatales habitudes, a dans
la société bien d'autres effets que ceux
\folio{26}
qu'on leur attribue. Tous les jeux ne
sont pas ceux qui se pratiquent dans
les académies, dans les maisons de jeu ;
tous les joueurs ne sont pas désignés
par ce nom : il s'en trouve ailleurs, en
plus grand nombre, qui confient de
même au sort leurs plus grands inté%
rêts, et dont les calculs sont aussi faux,
les combinaisons aussi absurdes et les
espérances aussi chimériques. Une
ruine totale, la perte même de la vie,
est le résultat fréquent de ces autres
jeux : je veux parler de ce que, dans
les différens états de la vie sociale,
des hommes, égarés par leurs vœux
ou leurs désirs, exposent ou sacrifient
sans prudence, sans nécessité, ou sans
motifs raisonnables, dans la poursuite
des faveurs de la gloire, de l'amour et
de la fortune.

\folio{27}
Dans ces jeux, comme dans les pre%
miers, on est justifié par le succès ; et
l'opinion, toujours complice des vices
heureux, attribue à des calculs plus
médités, à une conduite plus sage, ce
qui n'est que l'effet d'un hasard favo%
rable ou d'une coupable audace, tan%
dis qu'on condamne et flétrit celui que
plus d'ordre et de modération n'a pas
garanti des revers du sort.



\chapter[Des joueurs]{Des Joueurs}

\folio{28}
\lettrine{L}{es} joueurs n'ont pas un caractère
unique, déterminé, susceptible d'être
traité avec les mêmes procédés, ou
combattu avec les mêmes armes. Leur
caractère a des nuances extrêmement
variées : en cela il diffèrent des ava%
res, des envieux, des jaloux, des ivro%
gnes, des débauchés, dont la passion
ou le vice a un principe connu, ou
commun à presque tous.

L'ambition, l'orgueil, la cupidité,
l'ennui, le besoin, font des joueurs de
différents espèces.

On joue par caprice ou par système,
\folio{29}
par occasion ou par habitude, aux jeux
de hasard ou de commerce.

Les différents manières de jouer
tiennent à la différence des motifs, de
l'esprit, du tempérament et de la po%
sition des joueurs.

A voir de l'audace et le sang-froid des
uns, la timidité et la turbulence des
autres : on juge aisément si les mêmes
leçons ou les mêmes mesures de ré%
pression leur conviennent.

Tel n'a joué que quelques jours, et
a joué un jeu considérable ; tel autre
ne peut se priver du jeu un seul jour,
qui ne joue qu'un jeu modéré. A qui le
nom de joueur convient-il davantage ?

Il me semble du moins qu'il n'est
pas juste de comprendre sous la même
dénomination le goût et la passion, le
caprice et l'habitude du jeu.

\folio{30}
Je me demande si ce sont ceux qui ai%
ment le jeu, ou ceux qui ne l'aiment
pas, qui dans la société ont exception ?

Le nombre des joueurs honteux est
plus considérable qu'on ne le croit.

Je rencontre un homme de ma con%
naissance peu favorisé de la fortune.
On parle des jeux de hasard : « C'est
une fureur, dit-il, et on ne songe
pas à y mettre un frein !» Le soir,
je le trouve dans une maison particu%
lière. Il jouait à la bouillotte ; la cave
était de cinq louis.

Il y a de la différence entre le ca%
ractère et les procédés des joueurs aux
jeux de commerce, et ceux des joueurs
aux jeux de hasard.

Il convient aussi de distinguer les 
joueurs d'habitude et les joueurs de
profession. Si on ne sait poser une
\folio{31}
ligne de séparation entre les diffé%
rentes espèces de joueurs, on est ex%
posé à commettre des erreurs et des
injustices.

Des joueurs aux jeux de commerce
peuvent tirer parti de leur expérience,
de leur savoir, de leur finesse ; d'au%
tres sont trop légers, trop distraits,
ou d'une ignorance trop présomp%
tueuse pour ne pas donner à ceux-ci
beaucoup d'avantages. Cependant,
comme le hasard y une part plus ou
moins grande, les plus habiles y sont
quelquefois maltraités ; mais ils ne
tardent pas à reprendre leur supério%
rité. Voilà pourquoi on cite des joueurs
presque constamment heureux ; et
ceux qui donnent au goût justifié de
ces joueurs un aliment facile, n'osent
avouer et ne s'avouent pas eux-mê%
\folio{32}
mes leur ignorance ou leur faiblesse.
Mais imaginez qu'une perfide adresse,
qu'une coupable industrie vienne en%
core seconder l'art et l'expérience,
vous connaîtrez mieux les motifs pour
lesquels certains hommes font du jeu
leur unique occupation. Aussi ce n'est
pas aux jeux de hasard que si livrent
ces joueurs, dont l'honnête Dussaulx,
a fait, avec raison, un épouvantail ;
car je ne parle point encore de ces
banquiers de société, à qui l'art de
mettre en défaut les regards les plus
attentifs, réussit d'autant plus, que
là on s'en défie le moins : là on crain%
drait, par une accusation directe, de
paraître impudent ou grossier ; et il se
trouverait difficilement quelqu'un qui
oserait vérifier et constater le délit.

On conçoit que les joueurs de pro%
\folio{33}
fession ont pour la plupart les doigts
agiles, la tête froide, et cette absence
de passions favorable aux calculs. Mal%
heureusement ils ne prennent pas le
titre de joueurs, et ils souffriraient
impatiemment qu'on le leur donnât.
Ils sont toujours surchargés d'autres
affaires, d'autres soins, et le jeu est
le moindre sujet de leur conversation.
De qui les favorise surtout, c'est qu'il
est aisé de les confondre avec ceux qui
ne sont que des joueurs d'habitude.
Ceux-ci ont besoin de jouer, comme le
besoin de manger et de boire : l'heure
du jeu est marquée pour eux comme
celle de leur repas, de leur sommeil,
de leur dévotion. Il leur arrive sou%
vent de bâiller, de s'assoupir au jeu.
Ils jouent à un jeu de hasard comme
à tout autre, et ne sont étonnés que
\folio{34}
du coup qui les ruine sans ressource ;
c'est leur maison qui vient de s'é%
crouler. S'ils survivent à cet accident,
ils passeront le reste de leur vie à ra%
conter ce qu'il avait d'extraordinaire.

Les grands joueurs, j'entends ceux
qui ont la fureur du jeu, sont en gé%
néral des hommes à caractère et à
grandes passions, c'est-à-dire, qu'ils
ont le sang vif, la tête sulfureuse,
l'âme brulante, l'imagination exaltée,
la sensibilité profonde, et en cela, je
ne diffère pas autant qu'on le croirait
d'opinion avec Dussaulx, qui, tout
en refusant cette qualité aux joueurs,
prouve, par beaucoup de traits sail%
lans sur leur compte, qu'ils la pos%
sèdent à un degré éminent. Il est vrai
qu'un sort très-heureux ou très-mal%
heureux paraît rendre, même rend
\folio{35}
quelquefois les hommes insensibles ;
mais l'état d'enivrement ou d'apathie
ne dure pas long-tems, et la nature
ne tarde pas à reprendre son cours et
son énergie.

C'est de la classe de ces joueurs dont
je viens de parler, que sort la plus
grande partie de ceux que la ruine et
le désespoir portent au suicide.

Parmi les hommes de mérite que
présente Dussaulx comme ayant été
de grands joueurs, j'ai déjà nommé
Caton, Henri~IV, Montaigne, Des%
cartes, Collardeau, et lui-même, qui
en a fait l'aveu : je dois ajouter les
noms célèbres de Duguesclin, le Gui%
de, Rotrou, Voiture, Cardan, Halli%
fax, Schafsterbury, même du méde%
cin allemant Ponchasius Justus, aussi
auteur d'un livre contre le jeu.

\folio{36}
Qu'on ne s'y méprenne pas, quand
je dis que les grands joueurs sont des
hommes à passions, je ne dis pas que
les hommes à passions éprouvent né%
cessairement celle du jeu. Parmi ces
être peu communs, il en est qui ont
parcouru le cercle entier des passions ;
il en est dont une seule a consommé
la vie entière.

Les états sui laissent le plus de loisir,
sont ceux qui fournissent le plus grand
nombre de joueurs. C'est parmi les
ecclésiastiques que j'ai pris le goût du
jeu ; c'est parmi les militaires que je
l'ai vu régner avec le plus d'éclat. En
jouant aux jeux de hasard, ils ne sor%
tent pour ainsi dire, ni de leur profes%
sion, ni de leurs habitudes.

Mais quel long chapter il y aurait
à faire sur les joueurs ! …



\chapter[De la Fureur du Jeu]{De la fureur du jeu}

\folio{37}
\lettrine{L}{a} fureur du jeu a des symptômes
effrayans : ils se manifestent, soit lors%
que, semblable à une épidémie elle
a gagné la majeure partie des habi%
tans d'un pays, dont les jeux de ha%
sard sont devenus la principale oc%
cupation ; soit lorsqu'on fait à l'envi
des mises de jeu considérables, soit
lorsqu'un joueur, irrité de ses pertes,
ne suivant plus de règle, et obéissant
à une aveugle impulsion, s'expose à
perdre en peu de tems tout ce qu'il
possède ; soit enfin lorsqu'ayant per%
du son argent, on joue ses effets ou
autres choses, qui, par leur nature,
\folio{38}
semblent ne devoir par être mises à la
disposition du sort.

Voilà les abus, les excès du jeu ;
ceux contre lesquels la raison, l'hu%
manité invoquent des précautions et
des mesures, mais qu'il semble qu'au%
cun pouvoir ne saurait atteindre.

Quoique j'aie déjà mis le lecteur
en état d'examiner les déplorables
effets de cette fureur, je vais tâcher
de les rendre plus sensibles par des
exemples, en jetant, avec Dussaulx
et d'autres écrivains philosophes, un
coup-d'œil rapide sur ce qui a signalé
la passion du jeu en différens lieux
et en différens tems.

Chez les Gentous, le plus ancien
des peuples connus, le jeu avait causé
un tel désordre, qu'il fut nécessaire
de faire des lois pour en réprimer
\folio{39}
les excès. Un magistrat était payé
pour surveiller les rendez-vous de
jeu ; il avertissait des fautes, et fai%
sait couper les doigts aux prévarica%
teurs.

Les Romains, même dans l'état
républicain, qui suppose des vertus
plus pures, ont été des joueurs déter%
minés. Ovide, en parlant des joueurs
qu'il avait vus en action, dit : \frquote{On
sèche de désir, on frémit de colère,
on se meurt de rage. Que d'injures !
Quels cris ! Les malheureux ! ils
invoquent les dieux !}

On lit dans Juvénal : \frquote{On ne se
contente pas de porter sa bourse
au lieu de la séance, on y traîne
son coffre-fort.

On perd cent mille sesterces, et
on ne peut vêtir un esclave !

\folio{40}
Tous, jusqu'à la populace, sont
en proie à la fureur du jeu.}

On lit dans Tacite : \frquote{Quand les
Germains s'étaient ruinés au jeu,
ils se jouaient eux-mêmes. Le
vaincu, quoique plus jeune et plus
fort, se laissait garotter et vendre.}

Saint-Ambroise nous apprend que
chez les Huns, peuple grossier, mais
fidèle à sa parole, celui qui jouait sa
vie et perdait, se tuait quelquefois,
malgré son vainqueur.

Les nègres de Juida, les Chinois,
les Vénitiens jouaient leurs femmes
et leurs enfans.

Les Indiens jouaient jusqu'aux
doigts de leurs mains, et s'ils les per%
daient, ils se les coupaient eux-mêmes.

En Russie, on joue ses esclaves. Il
n'est pas rare de voir, soit à Moscow,
\folio{41}
soit à Pétersbourg, de pauvres fa%
milles appartenir successivement à
dix maîtres en un jour.

A Naples, et dans divers endroits
d'Italie, les bateliers jouent leur li%
berté pour un certain nombre d'an%
nées.

Aucun peuple n'a porté plus loin
la manie du jeu que les Anglais. Chez
eux, c'est presque l'esprit national.
Leurs factions, leurs affaires, leur
commerce, ils ont tout mis en jeu,
tout soumis au calcul. Navigateurs
insatiables, dit Dussaulx, ils se sont
familiarisés avec les dangers et le ha%
sard. Excepté quelques philosophes
et quelques-unes de ces ames que la
contagion ne saurait infecter, le reste
n'a étudié ses devoirs que sur des
tables de probabilités, dressées pour
\folio{42}
apprendre à faire des fortunes ra%
pides.

Mais, depuis le commencement de
notre monarchie, nous n'avons guè%
res, sur cet article, montré plus de
raison et de sagesse.

On voit dans nos annales, que ces
seigneurs hautains et fainéans qui ne
savaient guères que tourmenter leurs
vassaux, boire et se battre, étaient
pour la plupart des joueurs effrénés ;
qu'ils bravaient impunément la dé%
cence et les lois. Le frère de Saint-%
Louis jouait aux dés malgré les dé%
fenses réitérées de ce prince vertueux.
Duguesclin lui-même perdit, dans sa
prison, tout ce qu'il possédait ; le
duc de Touraine, frère de Charles~VI,
\emph{se mettait volontiers en peine,} dit
Froissard, \emph{pour gagner l'argent du
\folio{43}
Roi.} Transporté de lui avoir gagné
cinq mille livres, son premier cri fut :
\emph{Monseigneur, faites-moi payer.}

On jouait dans les camps et en
présence de l'ennemi. Des généraux,
après avoir ruiné leurs propres af%
faires, ont compromis le salut de la
patrie.

Philibert de Châlons, prince d'O%
range, commandant au siède de
Florence pour l'empereur Charles-%
Quint, perdit de l'argent qui lui avait
été compté pour la paye des soldats,
et fut contraint, après onze mois de
travaux, de capituler avec ceux qu'il
aurait pu forcer.

On parle dans le manuscrit d'Eus%
tache Deschamps, d'un hôtel de
Nesle, fameux par de sanglantes ca%
tastrophes : des acteurs y ont perdu,
\folio{44}
les uns la vie, les autres l'honneur.

Sous Henri II, dit Brantôme, un
capitaine français, nommé la Roue,
jouait cinq à six mille écus d'un
coup ; ce qui alors était exorbitant.
Il proposa de jouer vingt mille écus
contre l'une des galères de Jean-%
André Doria.

Un fils naturel du duc de Belle\-%
garde fut en état de lui compter, sur
ses gains, cinquante mille écus pour
s'en faire reconnaître juridiquement.
Il est vrai que la plus forte partie de
cette somme avait été gagnée en An%
gleterre.

Le peuple s'essayait déjà. Des fri%
pons s'étant concertés avec des Ita%
liens qu'ils avaient appelés à leur aide,
gagnèrent trente mille écus à Hen%
ri~III, \emph{qui avait}, dit un journaliste,
\folio{45}
\emph{dressé en son Louvre un déduit de
cartes et de dés.}

C'est surtout sous notre bon Henri~
IV, qui, jeune encore et peu fortu%
né, empruntait, pour jouer, de l'ar%
gent à tous ceux qu'il croyait de ses
amis, que la fureur du jeu a éclaté.
Qu'on en juge par quelques traits. En
une année, Bassompierre gagna cinq
cent mille livres, Pimentel deux cent
mille écus ; et le duc de Biron perdit
seul plus de cinq cent mille écus.
Qu'on considère le prix que l'argent
avait alors.

Henri IV, dit Péréfixe, n'était pas
beau joueur, mais âpre au gain, ti%
mide dans les grands coups, et de
mauvaise humeur dans la perte.

Comme les joueurs vulgaires, il
jouait tantôt avec audace, tantôt avec
\folio{46}
faiblesse. Le duc de Savoye jouant
avec lui et sachant qu'il aimait à ga%
gner, dissimula sont jeu, et, par po%
litique, renonça volontairement à
quatre mille pistoles.

On ne l'abandonnait pas impuné%
ment lorsqu'il perdait.

L'amour même ne pouvait le dis%
traire de sa malheureuse habitude.
On lui annonce qu'une princesse
qu'il aiamait va lui être ravie : \frquote{Prends
garde à mon argent, dit-il à Bas%
sompierre, et entretiens le jeu pen%
dant que je vais savoir des nou%
velles plus particulières.}

Lorsque, sous son règne, la Na%
tion, long-tems agitée par la guerre
civile, put enfin se reposer au sein de
la paix, presque toutes les professions
éprouvèrent la fureur du jeu. Des
\folio{47}
magistrats vendaient la permission
de jouer. Les joueurs avaient à la
cour un grand crédit, et jouissaient
de privilèges particuliers.

Paris se remplissait de joueurs : il
s'y forma, pour la première fois, des
académies de jeu, où \emph{la bourgeoisie,
les artisans et le peuple se précipi%
taient en foule.} Tous les jours il y
avait quelqu'un de ruiné.

On rapporte que Louis~XIII, celui
de nos Rois qui a le plus sévi contre
le jeu, aimait tant les échecs, que
pour qu'il y eût pas de temps perdu,
et qu'il pût y jouer en voiture, on
fit pour lui ce qu'on avait fait pour
l'empereur Claude : on plaça sans sa
voiture un échiquier bourré, sur le%
quel s'adaptaient les pièces montées
sur des aiguilles.

\folio{48}
Mazarin, dit l'abbé de St.-Pierre, in%
troduisit le jeu à la Cour de Louis~XIV
en 1648. Il engagea le roi et le reine
régente à jouer, et l'on préféra les jeux
de hasard. Le jeu passa de la Cour à la
ville, et de la capitale dans toutes les
petites villes de province.

Dès-lors on ne vit que des joueurs
d'un bout de la France à l'autre ; ils se
multipliaient rapidement dans toutes
les professions et même dans la robe,
qui se piquait encore d'une certaine
décence.

Le cardinal de Retz rapporte dans
ses Mémoires, qu'en 1650, le magis%
trat le plus âgé du parlement de Bor%
deaux, et qui passait pour être le plus
sage, ne rougissait pas de risquer tout
son bien dans une soirée, et cela,
ajoute-t-il, sans que sa réputation
\folio{49}
en souffrît, tant cette fureur était
générale !

Les États n'offraient plus, lors%
qu'ils étaient convoqués, que des as%
semblées de joueurs.

\frquote{J'ai vu, dit madame de Sévigné,
mille louis répandus sur le tapis ; il
n'y avait plus d'autres jetons ; les
poules étaient au moins de cinq, six
ou sept cents louis, jusqu'à mille,
douze cents…. On joue des jeux im%
menses à Versailles…. Le \emph{hoca} est
défendu à Paris, \emph{sous peine de la
vie, et on le joue chez le roi.} Cinq
mille pistoles avant le dîner, ce n'est
rien. C'est un vrai coupe-gorge !}

Dans les soupers clandestins et dans
les maisons de campagne du surinten%
dent Fouquet, vingt joueurs qualifiés,
tels que les maréchaux de Richelieu,
\folio{50}
de Clairembaut, etc., se rassemblaient
avec un peu de mauvaise compagnie
pour y jouer des terres, des maisons,
des bijoux, et jusqu'à des points de
Venise, jusqu'à des rabats ; on s'y avi%
lissait au point de circonvenir quel%
ques dupes opulentes, toujours in%
vitées les premières.

Les trois quarts de la nation ne sou%
pirèrent plus qu'après le jeu, qui, lui-%
même, devint un objet de spéculation
pour le gouvernement.

On connaît le fameux jeu auquel
Law, joueur étranger, devenu contrô%
leur général, entreprit de faire jouer la
nation, sous la minorité de Louis~XV.
On sait comme il séduisit ceux même
qui s'étaient garantis de l'épidémie
des jeux de hasard.

Vers le même tems, des ministres
\folio{51}
et des magistrats permirent des jeux
publics, parmi lesquels on distingue
ceux des hôtels de Gesvres et de Sois%
sons, où l'on a tant fait de victimes.

C'est là que le jeu de la roulette a
paru pour la première fois en France,
les joueurs de bonne foi ayant enfin
voulu jouer à un jeu où ils pouvaient
hasarder leur argent en toute sûreté.
En effet, il n'y a pas de jeu où les
chances soient plus égales \paren{je ne
parle pas de l'avantage du banquier}.

\frquote{Nous avons encore, dit Dussaulx,
qui m'a fourni presque tout ce que je
viens d'exposer, indépendamment de
cent maisons connues où l'on se ruine
tous les jours, dix fois plus de réduits
subalternes que l'on n'en comptait
sous Henri~IV, sous Louis~XIV et du 
tems de la régence.}

\folio{52}
On ne rougit plus, à l'exemple de
Caligula, de jouer au retour des funé%
railles de ses parens ou de ses amis.

La plupart de ceux qui vont aux
eaux sous prétexte de santé, n'y cher%
chent que des joueurs.

Aux États, c'est moins l'intérêt du
peuple qui rassemble une partie de la
noblesse, que l'attrait d'un jeu ter%
rible.

Tout est en feu au moment où j'écris,
ajoute Dussaulx ; sans parler des bas%
sesses, depuis deux jours je compte
quatre suicides et un grand crime.

Louis~XVI n'aimait pas le jeu. On
profitait de son absence pour se livrer
à la Cour aux jeux de hasard : on y
jouait sur-tout le pharaon et le biribi.
Il aurait été extraordinaire qu'au
milieu de la corruption des mœurs,
\folio{53}
corruption que ce roi vertueux n'a
jamais encouragée par son exemple,
l'épidémie du jeu ne se fût pas fait
sentir. On jouait gros chez des fi%
nanciers, des grands seigneurs, des am%
bassadeurs étrangers, et chez quel%
ques princes. La police autorisait ces
jeux, moyennant dix louis par maison.je

A l'hôtel d'Angleterre, tripot des
plus fréquentés, les jeux de commerce
étaient encore plus dangereux que les
jeux de hasard. De ceux-ci, on n'y a
guère joué que \emph{la belle}.

Madame de Polignac rassemblait
chez elle les personnes de distinction
les plus atteintes de la fureur du jeu.

On faisait des mises considérables
au trente-un, qui se jouait chez la res%
pectable et trop malheureuse prin%
cesse de Lamballe ; elle avait la fai%
\folio{54} 
blesse d'y faire une martingale de cent
louis.

On jouait aussi le plus gros jeu chez
la maréchale de Luxembourg, le duc
de la Trimoille, etc.

Il y avait un certain nombre d'hom%
mes habiles à diriger les jeux qui, au
premier mot, se rendaient dans les
maisons où ils étaient demandés, avec
les ustensiles de jeu et les fonds né%
cessaires. A la Cour, c'était toujours
les mêmes : on les appelait \emph{banquiers
de la société suivant la Cour.}

Dans quelques maisons de jeux les
mieux famées, on jouait le biribi, le
pharaon, le passe-dix, le pair et l'im%
pair, et le creps.

Il ne faut pas confondre ces mai%
sons avec les obscurs tripots, domaine
particulier de quelques inspecteurs.

\folio{55}
Le lieutenant de police appliquait
la rétribution de ces maisons de jeu
au soulagement de familles pauvres,
mais honnêtes ; et au soutien d'éta%
blissemens de bienfaisance, tels que
l'hospice des chevaliers de St.-Louis,
à la barrière d'Enfer.

Mais le centre des excès du jeu se
trouvait chez le dernier duc d'Orléans,
tantôt à son palais, tantôt à sa jolie
retraite de Mousseaux. Les étrangers
de distinction, de grands seigneurs,
des financiers, des commerçans, en
un mot, tous les gens à argent, étaient
recrutés pour ces parties, où l'on as%
sure qu'à l'insu du prince ne ré%
gnaient pas une exacte probité et une
extrëme délicatesse. Des Anglais y
étaient le plus remarqués. Les pertes
considérables qui s'y faisaient la nuit,
\folio{56}
étaient le jour le sujet des conversa%
tions.

Ces parties n'avaient plus lieu, lors%
qu'on établit, au Palais-Royal, une
maison de jeu de hasard autorisée par
la police. Là, de nombreuses victimes
ont encore été dépouillées.

Jusqu'au 10 août 1792, la plus bril%
lante et la plus forte partie des jeux de
hasard était à l'hôtel Massiac, place
des Victoires. Là se rendaient les mem%
bres les plus marquans de l'assemblée
constituante, et des personnes con%
nues par leur opulence.

Vers ce tems, la police municipale
autorisait des jeux au cirque du Pa%
lais-Royal, dans quelques autres par%
ties de ce palais, et dans divers quar%
tiers de Paris.

Le jeu n'avait jamais causé plus de
\folio{57}
ravages dans Paris que dans les deux
années qui ont précédé le 10 août ;
mais avant le 9 thermidor, c'est-à-%
dire, sous le règne affreux de la ter%
reur, sa fureur a paru céder à des
fureurs plus sanglantes.

Dès l'hiver de l'an 3, il a repris une
activité effrayante.

Je borne à cette époque le tableau
des principaux désordres attribués à
la passion du jeu, laissant à d'autres
le soin de porter dans les esprits une
terreur salutaire, par le recit de sui%
cides et d'autres accidens, résultats
inévitables de ces excès. Je ne pour%
rais signaler les traits qui, depuis ce
tems ont caractérisé la fureur des
jeux de hasard, sans accuser ou affli%
ger des personnes encore vivantes,
ce qui est loin de mon esprit et de
mes intentions.

\chapter
    [De la Théorie des Jeux de hasard]
    {De la théorie des jeux de hasard}

\folio{58}
\lettrine{C}{es} vœux, ces espérances qui gui%
dent et animent les joueurs dans la
poursuite des faveurs de la fortu%
ne, sont-ils donc sans fondement ?
La recherche de la vérité, les ef%
forts de la raison, les lumières, la
prudence, ne peuvent-ils procurer
des avantages certains dans les jeux
de hasard ? Voilà ce qu'il me convient
le plus d'examiner, car si des béné%
fices assurés peuvent être le résultat
de calculs et de combinaisons, la con%
duite des joueurs instruits est presque
justifiée ; et si ni lumières, ni calculs,
ne peuvent rendre le sort plus favo%
\folio{59}
rable aux joueurs qui les possèdent,
qu'à ceux qui ne les possèdent pas,
il résultera de cette certitude, ou de
cette véritée démontrée, que la pra%
tique du jeu dans l'espérance du gain,
est une folie à la fois la plus ridicules
et des plus funestes.

Il est incontestable que, dans les
jeux mêlés de science et de hasard, un
joueur peut avoir pour lui la faveur
des chances.

« Un joueur habile, dit l'abbé Du%
bos, pourrait faire tous les jours un
gain certain, en ne risquant son
argent qu'aux jeux où le succès dé%
pend plus de l'habileté des tenans,
que du hasard des cartes et des
dés ».

Aussi c'est à de pareils jeux que se
livrent le plus ceux qu'on nomme
\folio{60}
\emph{joueurs de métier}, ou ces hommes
dits de \emph{bonne compagnie}, qui veulent
au moins tirer parti de leur réputa%
tion. \paren{Il ne manque à cette conduite
que la moralité}. C'est pour l'instruc%
tion d'élèves dans ces jeux qu'on a
multiplié les méthodes et les traités.

Les gens honnêtes qui aiment à in%
téresser leur jeu, font donc très-bien,
s'ils ne veulent pas être dupes, de ne
s'engager que dans des parties où la
probité et l'égalité de talent puissent
au moins être présumées.

Quant aux jeux de pur hasard, pour
savoir s'il y a une manière plus ou
moins avantageuse d'y engager son
argent, indépendamment de ce qu'on
livre aux profits de la banque, je n'au%
rais pas eu besoin de recourir à des
études de mathématiques, et de consul%
\folio{61}
ter les auteurs qui ont écrit sur cette
matière ; la seule dénomination de
\emph{jeux de hasard} suffit pour dispenser 
d'une pareille recherche ; elle exclut
toute idée de science et de calculs ;
mais j'ai été curieux de connaître par
quels moyens on était parvenu à faire
croire qu'il était possible de fixer l'in%
constance du sort, et de donner l'exis%
tence au néant.

J'ai donc jeté un coup-d'œil sur
quelques écrits des auteurs qu'on cite
le plus en faveur de cette singulière
doctrine. Nommer les célèbres mathé%
maticiens Pascal, Bernouilli, Mont-%
Mort, d'Alembert, Fermat, Euler,
Ozanam, c'est dire que le charlata%
nisme avec lequel on annonce la so%
lution de leurs problèmes, ne peut
être mis que sur le compte de leurs
\folio{62}
éditeurs. Les écrits de peu de ces au%
teurs ont ce caractère : on trouve
dans leurs ouvrages des recherches
savantes et curieuses sur les jeux :
ils ont fait la décomposition et l'ana%
lyse des plus connus, et en ont pré%
senté les résultats ; mais il est aisé de
se convaincre qu'ils n'ont eu pour
but, dans leurs travaux, que d'éta%
blir des rapports du joueur avec le
joueur dans des chances inégales, ou
du joueur avec le banquier dans les
avantages accordés à celui-ci. Pour
cela, ils ont fait l'énumération des
différentes combinaisons résultantes
des différentes manières de faire ses
mises et ses paris ; ils ont déterminé
les proportions dans lesquelles les
paiemens devaient se faire ; ils ont dit
ce qui rendait égales ou inégales les
\folio{63}
conditions des joueurs dans les di%
verses conventions du jeu, ils ont
même indiqué, d'après des mises
faites, les degrés de probabilités de la
perte et du gain.

Voici, par exemple, un des pro%
blèmes qu'ils exposent.

« Lorsqu'on jour à croix ou pile,
la probabilité que croix ou pile arri%
vera sur un coup, est égale à ½. Il y a
également à parier pour ou contre ;
mais si l'on joue deux coups, et que
quelqu'un parie un écu d'amener
croix dès les deux coups, quelle somme
son adversaire doit-il mettre au pari,
pour que la condition des joueurs soit
égale » ?

Voici un autre de leurs problèmes :

« Dans une partie liée, un des
joueurs a gagné une partie ; on pro%
\folio{64}
pose de quitter ; la mise de chaque
joueur est d'un écu : quel est le droit
sur le fond du jeu de celui qui, sur
trois parties, a gagné la première » ?

De problèmes simples et facile à
résoudre, les mathématiciens passent
à d'autres plus difficiles ; les chances
se varient, se multiplient : les combi%
naisons se compliquent, et vous vous
égarez dans un labyrinthe de calculs,
si vous quittez un instant la ligne que
vous tracée l'esprit d'analyse.

Peu de joueurs sont capables de se
livrer à des études aussi abstraites ; et,
ne nous y trompons pas, le seul fruit
qu'on peut en retirer est d'apercevoir
dans le jeu ou dans la condition des
joueurs, c'est-à-dire, dans les mises et
dans les paiemens, des disproportions
dont ils seraient dupes, ou enfin de
\folio{65}
connaître les moyens de diminuer à
plusieurs de ces jeux, l'avantage du
banquier.

Mais comme en général les jeux qui
se jouent par entreprise sont calculés
de manière que le plus grand avantage
est toujours pour le banquier, la con%
naissance de cette vérité ne peut être
utile qu'à ceux qu'elle porte à renon%
cer à ces jeux ; car lorsque votre mise
est faite, à quoi vous servirait de sa%
voir que si telle carte sort, ou si les
dés donnent tel nombre, ou si la boule
tombe sur tel numéro, telle devra être
votre condition comparativement à
celle de la banque ? Irez-vous repro%
cher au banquier de ne pas payer 
votre gain en proportion de la perte à
laquelle vous étiez exposé ?

Il est bon cependant de savoir dis%
tinguer les inégalités de chances qui
résultent de la nature même du jeu,
indépendamment de l'avantage du 
banquier, et celles qui ne résultent
que de cet avantage, car alors on peut
ne jouer qu'aux jeux qui n'ont que
cette défaveur, ou on profite de la 
connaissance qu'on a des autres iné%
galités pour rendre son sort préfé%
rable à celui de son adversaire.

Les jeux qui ont d'autres chances
inégales que celles qui font l'avantage
de la banque, ne sont point des jeux
de pur hasard : c'est dans les sociétés
particulières que ces autres jeux se
jouent le plus et font le plus de vic%
times ; mais je considère ici que
les jeux de hasard, tels qu'ils se jouent
dans les maison de jeu.

L'avantage de la banque étant dif%
\folio{66}
férent pour les différens jeux, c'est
aux joueurs, s'ils sont susceptibles
de prudence, à préférer ceux qui
donnent à la banque le moindre avan%
tage ; mais cet avantage que néces%
sitent pourtant les dépenses et les 
frais d'administration, est désastreux
pour les personnes qui jouent fré%
quemment ou long-tems de suite. En
effet, je suppose que l'entreprise des
jeux ait le droit de deux pour cent 
sur l'argent exposé au tapis, en pre%
nant le terme moyen de son avan%
tage aux différens jeux qu'elle fait
jouer, celui qui joue un écu de cinq
francs chaque coup, après cinquante
coups, a nécessairement donné son
écu à l'entreprise ; et à ces jeux cin%
quante coups sont joués en bien peu
de tems.

\folio{68}
J'invite les joueurs à porter toute
leur attention sur cette vérité simple
et aussi facile à saisir que cette autre
vérité de calcul, \emph{un et un font deux.}
Elle est seule l'arrêt de leur ruine ; et
ce triste résultat de l'avantage inévi%
table de la banque, est ce que pré%
sente de plus clair et de plus certain
la solution des problèmes des célèbres
mathématiciens que j'ai nommés.

On chercherait inutilement dans
leurs écrits des règles et des méthodes
pour obtenir un gain assuré, même
des probabilités de gain aux jeux de 
hasard, dont les chances sont égales.

Cependant on voit de tems en tems
paraître de petits livrets dont le titre
annonce qu'on a enfin trouvé le
moyen d'enchaîner le sort aux jeux
de hasard, sur-tout à la roulette et
\folio{69}
au trente-un. Ceux de ces ouvrages
qui ne vous offrent pas des certi%
tudes de gain, vous offrent au moins
des probabilités.

Ces petits livrets sont écrits avec
une risible assurance : tout y est en
assertions, rien en preuves. On y in%
dique, par des colonnes de chiffres,
la manière de faire des mises ; on vous
dit gravement que tel numéro amène
tel autre ; on vous conseille de ne pas
mettre sur telle chance, parce qu'elle
est ingrate ; et on vous met à portée
de vérifier la justesse de ses combi%
naisons, en vous renvoyant à des ta%
bleaux qui contiennent la série des
numéros sortis ou des cartes tirées
pendant de nombreuses séances, où
l'on a eut la patience de les recueillir,
au lieu de les écrire chez soi en un
\folio{70}
instant, ce qui serait revenu au même.

Il n'est pas besoin de dire que le
style de ces petites brochures répond
à la justesse et à la lucidité des idées.
On reproche à leurs auteurs de ne
point mettre à profit, pour eux-%
mêmes, ces secrets de fortune ; mais
pour ne point trop m'écarter du ton
qui convient à mon sujet, j'avoue
que, soupçonnant à ces prétendus
écrivains plus de besoins que de ma%
lice, je crois qu'ils sont moins dignes
de mépris que de pitié.

Je déclare ici que les joueurs qui
fondent des soupçons ou des espé%
rances sur des inégalités ou des er%
reurs, dans la combinaison de chan%
ces des jeux de hasard auxquels font
jouer les banques publiques, se trom%
pent également : l'usage n'a consacré,
\folio{71}
pour ainsi dire, ces jeux, que parce
qu'ils sont calculés dans une exacte
proportion, et indépendamment de
ce qui forme l'avantage de la banque,
qui peut en être facilement distrait.
Ils sont, s'il m'est permis de hasarder
cette définition, la mise en action de
vérités de calcul mathématiquement
démontrées. Séparez-en l'avantage de
la banque, je donne publiquement le
défi aux plus habiles calculateurs de
prouver qu'aucune manière d'enga%
ger son argent, que des mises faites
dans un ordre progressif ou diminutif,
simples ou compliquées sur diffé%
rentes chances, puissent rompre cette
égalité, et donner aux joueurs le moin%
dre avantage.

« Les conjectures des joueurs, dit
Dussaulx, portent sur le néant ».
\folio{72}
Des listes de perte et de gain, faites
par des joueurs, avaient convaincu
cet excellent observateur, qu'aux
jeux de hasard il n'y a point de for%
tunes qui ne s'épuisent en peu de
tems.

Lorsqu'il reste aux hommes quel%
que raison, ils doivent s'en servir
pour régler leurs actions sur ce qui
leur paraît utile ou préférable.

Que les joueurs tâchent donc de
donner à la réflexions sur la nature et
les effets du jeu, une faible partie du
tems qu'ils destinaient à sa pratique !
Que sur-tout ils ne soient pas dupes
des noms ! On parle dans l'Encyclo%
pédie méthodique d'un jeu de dés ap%
pelé \emph{la parfaite égalité.} A ce jeu, six
chances, qui sont celles de raffles,
font perdre le banquier : et quatre-%
\folio{73}
vingt-dix autres, qui sont les dou%
blets, le font gagner.

Ainsi, dans le cours de deux cent
seize coups, où les pontes auront mis
un écu sur chaque case, le ban%
quier devra, toutes choses égales,
perdre trente écus, et en gagner
quatre-vingt-dix.

Son bénéfice sera donc de soixante
écus. Qu'on juge de la chose par le 
nom !

\chapter{Des Illusions des Joueurs}

\folio{74}
\lettrine{L}{orsqu'on} pense qu'il est aussi im%
possible de trouver le moindre moyen,
la moindre probabilité de gain à un
jeu dont les chances pour le gain et
pour la perte sont parfaitement éga%
les, qu'il est impossible de tracer sur
l'eau des caractères, ou de construire
une maison dans l'air, lorsqu'on
pense que cette égalité de chances
n'est jamais rompue qu'en faveur de
celui qui tient le jeu, et dont l'avan%
tage suffit pour opérer la ruine du
joueur d'habitude, on ne conçoit pas
l'empressement avec lequel des per%
sonnes, qui ne sont pas dépourvues
\folio{75}
de raison, vont ainsi exposer leur
fortune, troubler leur repos, user
leur tems et leur esprit, en un mot,
consumer toutes leurs facultés physi%
ques et morales. Imaginez un homme
qui sort de chez lui avec tout ce qu'il
possède d'argent, et le joue à croix
ou pile contre pareille somme que
met au jeu un premier venu, ne le
croiriez-vous pas atteint de folie ? Eh
bien, voilà en beau l'histoire de tous
les joueurs. Et n'est-ce pas ce qu'un
joueur pourrait faire de mieux ? Ici il
n'y a pour lui aucune perte de tems ;
il s'épargne de longues transes et
une pénible contention d'esprit ; il
est bientôt délivré de l'incertitude, le
plus cruel de tous les états. Ici le jeu
est égal, et on n'a pas contre soi un
avantage ruineux.

\folio{76}
Faisons une autre observation.

L'amour, le vin, la table, les jeux
d'exercice, affectent du moins agréa%
blement les sens, et si leurs excès
nous causent aussi de grands maux,
nous jettent dans de grands désor%
dres, ce n'est qu'après nous avoir
procuré des plaisirs : il n'en est pas
ainsi des jeux de hasard, et, pour
s'en convaincre, il suffit de jeter un
coup-d'œil sur les figures groupées
autour d'une table de roulette ou de
trente-un. J'invoque le témoignage
des joueurs ; le gain même donne de
la gravité, et laisse une vague inquié%
tude : on n'est point ainsi lorsqu'on
tient dans sa main le prix de son
travail.

Comment ce fait-il donc que la cu%
pidité, qui donne de l'esprit aux plus
\folio{77}
sots aveugle ici les hommes qui ail%
leurs montrent le plus de lumières et
de finesse ? Et quel est donc ce charme
puissant qui attire dans un précipice
ceux-même qui en connaissent toute
la profondeur ?

Je reconnais là un des effets les
plus merveilleux de l'imagination : oui
c'est l'imagination, trompée par tout
ce qui est capable de la séduire, qui
nous fait voir les choses moins telles
qu'elles sont, que telles que nous dé%
sirons qu'elles soient ; et c'est le plus
souvent le besoin d'être remué, d'être
jeté, pour ainsi dire, hors de soi par de
fortes sensations, qui nous fait obéir
à l'impulsion donnée par notre imagi%
nation, et rechercher une position
qui, par son attrait et son danger, met
en mouvement toutes nos facultés.

\folio{78}
\frquote{En général, le jeu nous plaît, dit
Montesquieu, parce qu'il attache
à l'espérance d'avoir plus ; il flatte
notre vanité par l'idée de la préfé%
rence que la fortune nous donne,
et de l'attention qu'ont les autres
sur notre bonheur ; il satisfait no%
tre curiosité, en nous procurant un
spectacle ; il nous donne les diffé%
rens plaisirs de la surprise.}

On ne peut caractériser l'amour
du jeu d'une manière plus vraie, plus
piquante et plus précise. Voilà bien
la manière des grands maîtres. Je vais
tâcher de donner du mouvement à ce
caractère.

L'ardeur du jeu vous presse : il n'y
a qu'un instant vous éprouviez un
mal-aise, une anxiété dont vous vou%
liez être délivré ; vous aviez pensé à
\folio{79}
une situation dans laquelle vous vous
trouveriez beaucoup mieux : ce mal%
aise, c'était ou le poids de l'ennui, ou
celui du besoin, ou le désir importun
de jouissances que vous ne pouvez
vous procurer : cette meilleure situa%
tion, c'est le jeu ; car lorsque vous
serez au jeu, vous aurez bientôt ce
qui vous manque. Vous y voyez d'ail%
leurs une agréable distraction ; vous
êtes frappé par l'idée d'un succès peu
ordinaire ; vous calculez déjà les heu%
reux effets de votre savoir et de votre
prudence : si vous aviez en votre pou%
voir une ressource plus sûre contre
l'ennui, ou plus conforme à la nature
de vos vœux, vous l'adopteriez de
préférence.

Cependant vous êtes loin du lieu où
vous pourrez essayer des parolis et des
\folio{80}
martingales, que vous avez négligés
jusqu'ici, et que vous ayez vu réussir
à d'autres. Qu'importe ? la route s'a%
brège par vos réflexions sur votre
prochain bonheur, et par l'emploi
idéal des bénéfices que vous allez faire.

Vous rencontrez un ami, confident
ordinaire de vos projets : il trouve que
vos calculs manquent de base, et ne
se rapportent à rien ; mais quel est le
joueur qui ne repousse pas, comme
les efforts d'un démon jaloux de son
bonheur, toute idée qui contrarie ses
espérances ? Est-il possible d'ailleurs
que ce que vous désirez si fortement,
ayant pris dans votre tête ardente le
caractère de la certitude, ne vous
paraisse pas infaillible ? Vous quittez
votre ami pour aller donner par le
fait un démenti à son opinion. Enfin
\folio{81}
vous respirez l'air qui vous est favo%
rable ; vous êtes aux salons de jeu :
l'accueil qu'on vous y fait vous sem%
ble d'un bon augure ; tout vous y sou%
rit ; vous y souriez à tout. La brillante
lumière que des lustres nombreux ré%
pandent dans toutes les parties de ces
salons dorés, les valets qui s'empres%
sent de venir vous offrir leurs soins,
ces monceaux d'or et d'argent dont
des joueurs, habiles sans doute, at%
tirent à eux des débris, l'absence to%
tale des images de gêne et de misère,
cet air de fête, cette aimable hilarité
qui épanouit toujours le visage des
joueurs aux commencemens des
parties, tous les objets enfin pren%
nent à vos yeux une teinte douce et
flatteuse. Votre esprit, déjà calmé, se
repose mollement dans la contempla%
\folio{82}
tion des tableaux de jeu ; votre œil
est recréé par la variété des chances
qu'ils présentent. Quelle abondante
moisson est offerte à l'art des combi%
naisons ! Vous ne pouvez vous défen%
dre d'une émotion légère, mais vous
vous gardez de montrer un empres%
sement qu'on prendrait pour de l'a%
vidité. Vous êtes un instant témoin
de la lutte qui vient de s'engager au
trente-un entre les pontes et le ban%
quier. --- Oh ! quelle faute, dites-vous
en vous-même, vient de faire cette
dame en mettant dix louis sur la cou%
leur qui a passé sept fois ! Il est bon
de mettre sur la gagnante ; mais au
huitième coup, quelle folie ! Cette
dame gagne ; vous en êtes étonné ; elle
fait paroli ; vous haussez les épaules :
elle gagne et laisse tout au jeu ; elle
\folio{83}
réussit encore. Alors elle retire, avec
sa mise, son bénéfice de soixante-dix
louis : vous ne lui reprochez pas moins
d'avoir joué contre toutes les règles
et toutes les probabilités. Au coup
suivant la couleur perd. \frquote{Voyez, lui
dites-vous, ce qui vous serait ar%
rivé si vous aviez fait un pas de
plus ! --- Oh ! je m'en étais douté,
vous répond-elle, mes pressenti%
mens ne me trompent guères.}

Vous portez votre attention sur un
joueur qui attendait ce moment pour
commencer une martingale à la con%.
tre-couleur : vous êtes tenté de jouer
le même jeu, mais vous vous êtes fait
la règle de ne vous engager que sur
la noire, et lorsque la rouge aura pas%
sé cinq fois ; vous savez que rien ne
porte plus malheur au jeu que de ne
\folio{84}
pas tenir à sa première idée. « Ce mon%
sieur, dit un voisin, est le joueur
le plus sage que je connaisse ; il ne
manque jamais son coup, et se re%
tire chaque jour avec une vingtaine
de louis de bénéfices. »

Cependant la couleur sort six fois ;
le martingale saute. « Ma faute,
dit-il, est de n'avoir pas pris plus
d'argent sur moi. »

Tandis que vous vous applaudissez
d'avoir su résister à la tentation, un
autre joueur murmure. « Je ne perds,
dit-il, que dans cette infernale mai%
son. Puis reprenant sa sérénité :
Je suis sur d'être plus heureux dans
une autre ; » et il sort. « Il n'est point
étonnant qu'il ait perdu, » observe
un homme âgé, à qui l'on attribue une
longue expérience du jeu; » il s'ob%
\folio{85}
stine à suivre une chance qui, de%
puis quinze jours, n'a réussi à per%
sonne. »

Enfin la rouge a passe cinq fois :
vous commencez vos parolis ; mais
les intermittences durent un quart%
d'heure ; mais le refait du trente-un
est fréquent : vous renoncez aux pa%
rolis, et augmentant vos mises, vous
ne jouez plus que sur la rouge. Il vient
une longue série de noires. Dix fois
vous vous êtes dit : \emph{La prudence veut
que je me retire ;} des combinaisons
qui vous ont paru plus sûres, vous
ont tenu enchaîné au jeu : vous êtes
d'ailleurs encouragé et cousolé par
les éloges de vos voisins, qui, applau%
dissant à la manière dont vous avez
coutume de faire vos mises, rejettent
vos pertes sur une fatalité dont il y a
\folio{86}
peu d'exemples. Rien ne vous réussit.
Vous remarquez dans la foule une fi%
gure qui vous parait sinistre ; vous
donneriez une partie de ce qui vous
reste pour être débarrassé de cette
vue qui vous occupe malgré vous, et
à laquelle vous attribuez toutes les
fautes que vous avez faites. Incapable
de réflexions, vous jouez sans plan,
sans ordre ; alors tout vous prospère ;
vos fonds sont rentrés ; ils se doublent ;
c'est l'instant de la retraite. En vous y
disposant, vous promenez d'un air a%
vantageux vos regards sur la galerie,
étonnée sans doute de l'habileté avec
laquelle vous avez su faire fléchir le
sort ; vous croyez lire sur tous les vi%
sages que la joie de votre triomphe
est partagée, elle en devient plus
vive.

\folio{87}
En vous retirant, vous traversez
une salle de roulette. Un homme qui
vous a toujours porté bonheur, vous
salue et vous attire auprès de lui :
vous lui annoncez votre gain et votre
intention de vous y tenir. Cependant
d'heureuses tentatives de quelques
joueurs vous rappellent votre plan
favori de martingale. Après quel%
ques hésitations, vous en faites l'es%
sai aux petits ecus : vous avez pour
vous les cinq-sixièmes des numéros ;
mais votre martingale est déjà portée
assez haut pour que vous ayez sur
le tapis toute la somme que vous avez
gagnée au trente-un. Les zéros tom%
bent ; la terreur s'empare de vous :
vous avez dans la main, pour un der
nier coup, tout l'argent que vous avez
apporté au jeu ; vous n'osez vous en
\folio{88}
dessaisir : laboule se câse; vous auriez
gagné ; votre cœur se comprime ;
vos idées s'égarent ; vous restez long%
tems stupéfait ; mais vous savez que
les yeux sont fixés sur vous ; vous
craignez qu'on ne vous accuse de
timidité, ou qu'on attribue votre re%
tenue à l'épuisement de votre bourse.
Vous vous avisez alors de réaliser un
projet de distribution de mises iné%
gales sur une partie du tableau de
la roulette. Combien de fois n'avez%
vous pas reconnu l'avantage de ces
mises compliquées ! Infortuné ; vo%
tre imagination vous promène d'er%
reurs en erreurs, jusqu'à la fatale
catastrophe où vous ne verrez plus
que la vérité. En vain vous variez ves
mises, en vain vous vous emparez
des trois-quarts tableau, votre
\folio{89}
opération n'est point compliquée ;
elle est simple ; vous jouez un contre
un, et vous avez toujours contre vous
les zéros, terrible ascendant qui se
fait une fois moins sentir à celui qui
joue sur de simples chances, comme
la rouge, le passe, 1 impair, car il a
le privilége des refaits auxquels, par
vos mises concentrées sur les numé%
ros, vous avez renoncé. Vous pros%
pérez un instant, mais bientôt votre
chute est rapide, et vous payez
cher l'erreur funeste qui vous a fait
préférer le moyen le moins propre
à écarter de vous le besoin et les en%
nuis, ou à vous procurer des jouis%
sances.

Ainsi finit le jeu ; vous avez d'a%
bord été épris de l'éclat et des traits
d'un beau visage ; vos regards s'y
\folio{90}
sont fixés ; mais insensiblement vous
avez vu cet éclat s'affaiblir, ces traits
s'altérer, et vous n'avez plus eu de%
vant les yeux qu'une affreuse tête
de mort qui a porté dans vos sens
l'horreur et l'épouvante.

Tandis que vous retournez d'un
pas lent dans vos obscurs foyers, vo%
tre ame est plongée dans une morne
tristesse, et votre esprit s'occupe en
vain de la recherche des moyens de
remplir le lendemain des engage%
mens sacrés, et d'alimenter vos mat%
heureux enfans qui vous attendent.

Cruel aveuglement ! ce que vous
regrettez davantage, c'est de n'avoir
pu porter au jeu une plus forte som%
me. Vous croyez que s'il vous avait
été possible de jouer quelques coups
de plus, le sort aurait cessé de vous
\folio{91}
être contraire ;  et si une espérance
vous ranime par intervalles, c'est
celle de recouvrer le lendemain ce
que vous avez perdu en livrant aux
hasards du jeu le prix de là vente des
effets qui vous sont le plus nécéssaires.

Voilà une faible esquisse des illu%
sions du joueur : mais comment les
attaquer ? Comment les détruire ? Je
le répète ; je doute qu'on y réussisse
par le langage austère de la morale,
et par la touchante effusion du sen%
timent : je n'en suis pas moins per%
suadé, je prouverai peut-être que les
lois prohibitives, que des mesures de
rigueur ne feraient aujourd'hui que
rendre l'incendie plus considérable.
Encore une fois, les joueurs calcu%
lent ; il faut compter avec eux ; il faut
que les vérités de calcul les plus sim%
\folio{92}
ples, les plus claires, les poursuivent,
les atteignent, frappent leurs oreilles
et leurs yeux, s'emparent de tous
leurs sens dans les maisons de jeu,
et que leurs pertes leur paraissent
inévitablement la preuve de la règle
dont vous leur avez donné la leçon.

Et nous, écrivains moralistes, n'ou%
blions pas que l'homme du peuple
qu'on n'avertit point assez, par des
écrits à sa portée, des dangers aux%
quels l'expose sa cupidité, se rendra
plutôt à la démonstration du préju%
dice que cause aux joueurs honnêtes
l'inégal contrat du jeu, que l'homme
du monde qui, livré à ses désirs et
à ses passions, est entouré de plus de
prestiges.

\chapter
    [De la Conduite du Jeu]
    {De la conduite du jeu}

\folio{93}
\lettrine{C}{'est} se mal conduire que de jouer
aux jeux de hasard, et le meilleur
conseil que puisse donner un homme
raisonnable à ceux qui s'y livrent,
est d'y renoncer sans délai. S'ils
n'ont pas ce courage, la conduite
qu'ils doivent tenir consiste à tirer
le meilleur parti possible de leur po%
sition, c'est-à-dire à éviter ce qui
peut leur nuire davantage, en n'ou%
bliant pas néanmoins que, dans tou%
tes les circonstances de la vie, l'équité
doit être la première règle de con%
duite.

Il y a une grande imprudence à
prendre pour son jeu sur son néces%
\folio{94}
saire ; car, loin de se mettre ainsi en
état de réparer sa mauvaise fortune,
on adopte précisément le moyen le
plus sûr de l'accroître et de la porter
à son comble.

Si la raison permettait de croire
aux pressentimens, je dirais qu'il
semble que la crainte des joueurs
qui exposent au jeu leur nécessaire,
crainte qui les fait jouer de la ma%
nière la plus désordonnée, se vérifie
presque toujours. Dans la vie ordi%
naire, ce qu'on appelle malheur, est
le plus souvent l'effet du défaut de
conduite : cela est encore plus vrai
au jeu. La moindre perte est un grand
malheur pour les personnes qui
jouent ce que réclament leurs be%
soins ou ceux de leur famille, ou
l'acquit de leurs engagemens ; aussi
\folio{95}
les entendez-vous se plaindre dou%
loureusement à chaque coup qui leur
est contraire, tandis que l'homme
riche, ou celui qui n'a pris ce qu'il
met au jeu que sur son superflu,
perd avec tranquillité, ignore même
quelquefois s'il perd ou s'il gagne.

Le petit jeu et le gros jeu ne sont
tels que relativement aux facultés
des joueurs. Celui qui porte au jeu
une forte somme, quand il n'a d'au%
tre intention que celle de jouer quel%
ques pièces, sans augmenter ses
mises pour courir après celles qu'il
aura perdues, commet aussi une
grande imprudence : il faut qu'il soit
bien maître de lui, pour ne pas aug%
menter son jeu, s'il est en perte. Les
exemples d'une pareille modération
sont très-rares, et les exemples de
\folio{96}
ceux qui, en pareil cas, ont exposé
et perdu leur somme entière, ne le
sont point. Celui-là même qui, por%
teur d'une forte somme, ne joue que
petit jeu, mais qui joue sans inter%
ruption, fait encore une grande faute,
car sa somme s'altère et diminue in%
sensiblement par l'avantage du ban%
quier.

Il ne convient de porter au jeu que
la somme qu'on peut perdre, sans dé%
ranger ses affaires et s'exposer à de
grandes privations.

Je citerai ici ce que la Bruyère, le
moraliste français que j'aime davan%
tage, et que j'ai lu avec le plus de
plaisir, dit de la conduite dans les
choses auxquelles le hasard participe
davantage.

\frquote{Le guerrier et le politique, non
\folio{97}
plus que le joueur habile, ne font
pas le hasard, ils l'attirent \emph{et sem%
blent presque le déterminer.} Non-%
seulement ils savent ce que le sot et
le poltron ignorent, je veux dire, se
servir du hasard quand il arrive,
ils savent même profiter, par leurs
précautions et leurs mesures, d'un
tel hasard et de plusieurs à-la-fois.
Si ce point arrive, ils gagnent ; si
c'est un autre, ils gagnent encore.
Ces hommes sages peuvent être
loués de leur bonne fortune, comme
de leur bonne conduite, et le ha%
sard doit être récompensé en eux
comme la vertus}.

Ces réflexions sont plus applicables
aux jeux dans lesquels le savoir et
l'habileté entrent pour quelque chose,
qu'aux jeux de pur hasard, auxquels
\folio{98}
cependant elles ne sont point entière%
ment étrangères.

Les joueurs ne se ruinent guères
que par leur inconduite ou par leur
imprudence.

La sagesse et la conduite ont des
effets bien plus sûrs aux jeux de com%
merce, c'est-à-dire, aux jeux mêlés
de hasard et de science, qu'aux sim%
ples jeux de hasard, parce que ces
qualités suffisent souvent pour y
faire pencher le sort du côté de ceux
qui les possèdent, sur-tout s'ils y joi%
gnent l'habileté ; mais je dois dire
aussi que pour la plupart des joueurs,
les jeux de commerce sont bien plus
à redouter que les jeux de hasard ;
car la faveur des chances y cède  pres%
que toujours à la supériorité du ta%
rent, supériorité d'autant plus dange%
\folio{99}
reuse, qu'il est aisé de la dissimuler.

Divisez la somme que vous consa%
crez au jeu en un nombre de portions
que vous fixez suivant les coups que
vous voulez jouer, ou les mises que
vous voulez faire.

Les mises au jeu doivent être faites
en proportion de ce qu'on craint de
perdre ou de ce qu'on aspire à ga%
gner.

Il vaut mieux, lorsqu'on perd,
diminuer sa mise que de l'augmen%
ter. Alors, comme le jeu a ses varia%
tions, si vous gagnez, vous pouvez
ressaisir votre argent ; si vous conti%
nuez de perdre, vous vous félicitez
de n'avoir pas, par de plus fortes
mises, rendu votre perte plus consi%
dérable ;  et cette réflexion contribue
à vous consoler.

\folio{100}
Êtes-vous en gain, doublez pro%
gressivement vos mises ; faites des
parolis, n'exposez point de votre ar%
gent avec celui que vous gagnez ; au
contraire, retirez à mesure partie de
ce gain : fixez le nombre des coups
de gain après lesquels vous retirerez
tout ce qui vous appartiendra au ta%
pis, sauf à recommencer le même
jeu.

Voici un exemple de cette manière
de jouer :

Vous avez sur vous vingt écus que
vous voulez hasarder successivement :
vous les divisez en dix portions, afin
que votre mise simple soit de six
francs. Vous jouerez donc d'abord six
francs ; si vous perdez, vous jouerez
encore six francs : il est indifférent
que ce soit au coup suivant ou que
\folio{101}
ce soit après que vous aurez laissé
passer plusieurs coups sans jouer ;
si tous gagnez, vous retirerez votre
mise ; si vous gagnez le second coup,
vous laisserez au jeu les douze francs
qui vous reviennent ; si vous gagnez
le troisième coup, vous retirez six
francs sur les vingt-quatre qui vous
reviennent, et vous laisserez au jeu
les dix-huit francs. Si vous gagnez
le quatrième coup, comme les séries
au-delà sont rares, et qu'au jeu que
vous jouez, une grande ambition ne
vous est pas permise, retirez douze
francs et n'en laissez que six. Con%
duisez-vous comme si vous commen%
ciez la première partie que je vous
indique plutôt pour vous faire con%
naître l'esprit dans lequel vous devez
jouer, que pour vous indiquer une
méthode déterminée.

\folio{102}
Gardez-vous sur-tout de penser que
vous serez plus près du gain en pour%
suivant une chance à laquelle vous
venez de perdre ou qui a été long-%
tems sans paraître. C'est un préju%
gé qui a été funeste à beaucoup de
joueurs que celui de croire qu'une
chance ou qu'un numéro en retard
est plus prêt qu'un autre à arriver.
Peu de personnes, il est vrai, ont
l'esprit assez sain et la tête assez froide
pour se garantir de ce préjugé, et ne
pas suivre les mouvements qu'excite
en eux leur imagination étrangement
trompée. Les hommes les plus sensés
et les plus spirituels ne me nieront
pas qu'en pareille circonstance ils
n'ont pas su résister à l'empire de ces
mouvemens. On augmente sa mise
avec une effrayante progression ;
\folio{103}
bientôt le jeu est plus de la fu%
reur ; c'est un inconcevable aveugle%
ment, c'est une rage. J'ai vu alors des
joueurs obstinés, jusques-là tran%
quilles en apparence, mettre au jeu
en rugissant, or, billets, monnaie,
tout ce qu'ils avaient sur eux. Une
pareille erreur a causé plus d'une
ruine et plus d'un suicide ; elle ne
vous jeterait pas dans ces excès, mais
elle pourrait vous faire perdre en un
instant l'argent que vous aviez ré%
parti de manière à vous occuper pen%
dant une partie de la séance.

Persuadez-vous donc que chaque
coup étant isolé, étant indépendant
de ceux qui le précédent, n'étant que
le résultat de mouvemens plus ou
moins rapides, mais toujours incer%
tains, le retard de sortie d'une chance
\folio{104}
ou d'un numéro ne peut fonder au%
cune probabilité en sa faveur.

D'autres préjugés rendent les crain%
tes des joueurs aussi peu raisonnables
que leurs espérances. Pour suivre le
petit plan que vous vous êtes fait, et
dont l'exécution serait du moins un
amusement pour vous, puisque l'é%
vènement du jeu ne pourrait ni trou%
bler votre repos, ni compromettre
votre fortune, vous auriez besoin de
mettre sur une chance ; mais cette
chance est proscrite ; tout le monde
l'abandonne ; vous l'évitez comme on
évite quelquefois un ami dans le mal%
heur. Elle vient à gagner, et au regret
d'avoir écouté une fausse prévention,
se joint l'embarras de se faire un au%
tre plan. Comme on ne doit jouer
qu'à un jeu dont on connaît bien la
\folio{105}
théorie, et qu'on sait n'avoir d'autres
perfidies que celles dont le sort est
capable, et dont aucune réflexion,
aucune prudence ne peut vous ga%
rantir, il faut y apporter cette assu%
rance, j'allais dire cette confiance
qui donne au jeu plus de charme
lorsque le succès la justifie, et que
du moins la raison ne condamne
pas, si elle est trompée par l'évè%
nement.

Aux jeux de hasard, dont les mou%
vemens sont si prompts, si variés,
où les tentations renaissent à toutes
les minutes, et où les sens sont agités
si puissamment, même par d'autres
intérêts que les nôtres, il est bien
difficile de suivre le plan qu'on s'est
lait d'abord, et de se maintenir dans la
contention d'esprit qu'il exige : rien
\folio{106}
n'est cependant plus nécessaire, si
on veut éviter l'embarras de volon%
tés incertaines ou contraires l'une à
l'autre, et le défaut de règle à la suite
duquel vient toujours le défaut de
conduite.

Le sang-froid et la résignation font
mériter le titre de beau joueur, qui
est encore au jeu une sorte de séduc%
tion : s'ils ne rendent pas le hasard
plus favorable, ils mettent plus en
état de profiter des avantages qu'on ob%
tient, et de ne pas accroître ses revers.

Lorsqu'on a peu d'argent et qu'on est
jaloux de le conserver, ou lorsqu'on
veut s'occuper au jeu quelque tems,
il faut jouer à des chances où le gain
et la perte puissent être en proportion
égale ; par exemple, si ne jouant à la
roulette qu'une pièce à la fois, vous la
\folio{107}
mettez ou à une transversale ou à un
carré, votre argent, à moins d'une fa%
veur du sort peu ordinaire, sera bien%
tôt épuisé. J'ai vu des joueurs, ou
plutôt des joueuses avides \paren{car cette
imprudence est plus commune aux
femmes}, n'ayant qu'une somme mé%
diocre, jouer sur deux numéros, ou
jouer sur un plein. Leurs lamenta%
tions, lorsqu'elles perdaient, n'étaient
assurément pas fondées.

Quoi qu'on en dise, j'ai reconnu
que le moyen d'obtenir au jeu les fa%
veurs de la fortune, n'était pas de la
brusquer.

Lorsqu'on est en gain, on est pres%
qu'excusable de tenter des coups plus
hasardeux, et d'aspirer même à une
forte somme. Le succès a quelquefois
couronné cette hardiesse, qui s'allie
avec la prudence. Perd-on son béné%
\folio{108}
fice ? on fait preuve de sagesse et de
conduite en revenant à ses mises sim%
ples, après avoir joué beaucoup d'ar%
gent.

Mais s'il est pour des joueurs qui
ont sur eux de fortes sommes, un sys%
tème désastreux et qu'il faille s'atta%
cher à proscrire, c'est celui des mar%
tingales portées à un grand nombre
de coups. Quelque faible que soit la
somme qui les commence, et quelque
régulière que soit la progression qu'on
y observe, elles menacent toujours du
plus grand danger. Le joueur imite
alors celui qui porte du feu dans un
magasin à poudre ; il peut y entrer cin%
quante fois sans qu'il lui arrive d'acci%
dent ; mais l'instant fatal de l'explosion
vient tôt ou tard ; il pouvait venir à la
première imprudence.

\chapter
  [Des avantages et des dangers du Jeu : Bonheur et malheur]
  {Des avantages et des dangers du jeu : bonheur et malheur}

\folio{109}
\lettrine{Q}{uoi} qu'en dise une philosophie trop
austère, le jeu n'est pas sans quelques
avantages pour ceux qui n'y portent
point de passions désordonnées ; et si
nos mœurs ne nous avaient accoutu%
més à ne voir que des excès dans les
choses qui peuvent nous procurer des
jouissances, nous reconnaîtrions que
les personnées douées d'un caractère
de modération et capables de se sous%
traires à de vicieuses habitudes, peu%
vent mettre, même les jeux de ha%
sard, au nombre de leurs plaisirs.
Ces jeux, il est vrai, disent peu de
choses à l'esprit, mais quelquefois on
\folio{110}
leur est redevable d'une distraction
dont on a besoin, et qu'on ne cherche%
rait pas dans les jeux de commerce,
lesquels, comme les autres, ne tien%
nent pas l'âme dans cette légère émo%
tion que cause l'incertitude de l'évè%
nement ; lesquels aussi exigent, pour
qu'on soit en état de s'y défendre,
une science ou une expérience qu'on 
n'a point acquise. Les premiers ont
encore sur ceux-ci le grand avantage
de pouvoir être pris et quittés à vo%
lonté ; ils donnent aux sens une agita%
tion salutaire, pour ceux qui languis%
sent dans de trop fortunés loisirs. En%
fin, ce n'est guère qu'en exposant à
ces jeux ce dont on peut se passer, ou
ce à quoi on est peut attaché, qu'il est
possible que ceux qui sont dénués de
talens, ou d'industrie, ou d'amour du
\folio{111}
travail, obtiennent légitimement les
moyens d'accroître leurs jouissances.
Qu'on ne perde point de vue que je
ne pardonne pas aux joueurs de mettre
au jeu leur nécessaire ou celui de leur
famille. Je conviens qu'on ferait beau%
coup mieux d'appliquer son superflu
à des actes de bienfaisance, qui sont
pour certaines ames des objects de né%
cessité; mais je dois considérer ici les
hommes tels qu'ils sont en général,
tels qu'on n'empêchera pas qu'ils
soient, et non tels qu'ils devraient être.
C'est trop accorder que de trop exiger
en morale : la difficulté de faire tout
le bien qui est demandé, semble dis%
penser de faire aucun bien : c'est ainsi
qu'on peut entendre la maxime sou%
vent citée, malgré sa fausseté : \emph{le
mieux est l'ennemi du bien.}

\folio{112}
Cette observation me porte à m'ex%
pliquer sur la légimité que j'attribue
aux gains des joueurs, et que Dussaulx
leur conteste, plutôt d'après des con%
sidérations, que d'après un principe
juste. Le contrat du jeu peut n'être
pas moral dans son esprit, et cepen%
dant être légitime dans ses effets : il
peut n'être pas légitime relativement
à la loi, et cependant l'être relative%
ment aux contractans. S'il n'avait pas
ce caractère de légitimité, il ne serait
pas obligatoire ; et alors il y aurait au
moins deux maux pour un. Ce qui lui
donne ce caractère est l'égalité des
mises, des risques et des espérances.
Si aux jeux de hasard, tels qu'ils se
pratiquent aujourd'hui, l'égalité pa%
raît être rompue, c'est en faveur du
banquier ; mais on ne peut dire qu'elle
\folio{113}
le soit réellement, puisque son avan
tage reconnu et consenti par le joueur,
n'est regardé que comme le paiement
des travaux et des dépenses de l'admi%
nistration : quant au joueur, qui a eu
plus de chances contre lui qu'il n'en
a eues pour lui, il est au moins singu%
lier qu'on l'accuse de recueillir injus%
tement ce que le sort lui accorde, et
qu'on compare ses bénéfices à des ra%
pines et aux pillages, tels que ceux
qu'excercent les Arabes sur les cara%
vanes. Ces bénéfices sont, dit-on, la
dépouille de pères de famille, réduits
au désespoir ; cet argent qu'on vous
donne, sort des mains de celui qui l'a
volé : je le veux ; mais cette dépouille
et cet argent volés, vous ne les recevez
pas des joueurs malheureux ou fri%
pons ; ils sont devenus la propriété
\folio{114}
des banquiers, les seuls avec lesquels
vous ayez contracté : l'unique effet
de votre gain ne se fait sentir qu'à la
banque, et quel argent aurait-on si
on se faisait scrupule de recevoir ce%
lui qu'on croit avoir passé par des
mains impures ?

Voici les erreurs et les exagérations
dans lesquelles on est entraîné par
l'amour du bien. Je ne reconnais pas
plus la justesse de l'idée de Dussaulx,
lorsque contestant au jeu ses avan%
tages, il dit : \frquote{Quand vous jouez la
moitié de votre bien, si vous ga%
gnez, votre capital n'augmente que
d'un tiers ; si vous perdez, il dé%
croît de moitié}. Qui ne serait pas
étonné de voir que ce savant Dus%
saulx ne connaît rien de plus fort
contre la séduction du jeu que cet
\folio{115}
argument qui n'est que spécieux ?
Celui qui joue la moitié de son bien,
s'il gagne, l'augmente réellement de
la moitié et non d'un tiers : à la vé%
rité, lorsque cette valeur de la moitié
de son bien y est jointe, elle n'est
plus comptée que comme le tiers de
la totalité ; mais il n'est pas moins
vrai que, par l'effet de son gain, ses
jouissances sont augmentées de moi%
tié, comme elles seraient diminuées
de moitié s'il eût perdu. Il n'y a donc
là qu'un jeu de mots ou une fausse
image propre à donner une fausse idée
à ceux qui lisent superficiellement et
non cette inégalité de condition dont
Dussaulx a voulu frapper les sens.

S'agit-il, au surplus, d'établir des
proportions entre les privations et
les jouissances ? Je dirai qu'il est pos%
\folio{116}
sible que celui qui hasarde la moitié
de sa fortune, ait assez de philosophie
pour la perdre sans peine, mai qu'il
peut trouver dans l'augmentation
d'un tiers de son bien, la jouissance
de ce qui était le terme de ses vœux.

J'ai dit les avantages que paraissent
avoir les jeux de hasard, qui devraient
être abandonnées aux grands et aux
riches, pour lesquels il semblent faits ;
dirai-je leurs dangers ? On a déjà pu
s'en former une idée, par les désordres
dont j'ai précédemment esquissé le
tableau. Peu de personnes sont ca%
pables de se maintenir dans l'esprit de
modération et dans les règles de con%
duite que j'ai recommendées : le nom%
bre de celles qui n'ont jamais su résis%
ter à l'attrait de ces jeux, après l'avoir
une fois connu, est considérable.
\folio{117}
Les désirs immodérés, les vaines es%
pérances, les trompeuses illusions,
abandonnent difficilement les joueurs
dont ils se sont emparés : ils les pour%
suivent jusques dans leurs songes, les
dégoütent insensiblement des plaisirs
simples et honnêtes, et leur rendent
bientôt l'acquit de leurs devoirs et
toutes leurs occupations ennuyeux
et insupportables. Quelques faibles
que soient les pertes qu'on ait faites,
on est préssé de les réparer, et l'on se
flatte de s'en tenir là lorsqu'elles se%
ront réparées ; mais, ou l'on tâche
avec plus d'ardeur de ressaisir ce
qu'on a perdu de nouveau, ou l'on
trouve agréable et facile d'ajouter
d'autres gains à ceux qu'on a faits.
Ainsi le tems se perd, l'esprit se con%
sume, le cœur s'endurcit ou se cor%
\folio{118}
rompt, le sang s'aigrit, la santé s'al%
tère et la fortune s'épuise dans des ha%
bitudes qu'on a bientôt contractées,
et auxquelles on ne renonce point
sans une force plus qu'humaine.

Voilà, non une peinture exagérée
des dangers du jeu, mais des vérités
qu'atteste l'expérience de presque
tous les joueurs, et qui doivent ren%
dre bien faibles à des yeux éclairés,
les rares avantages qu'on peut trouver
dans les jeux de hasard.

Quelques-uns observent sérieuse%
ment qu'il faut laisser ces jeux aux
personnes à tout réussit, parce
qu'elles sont nées heureuses, et ils
vous citeront beaucoup d'exemples
d'un bonheur constant ; d'autres vous
disent qu'ils ne sont point étonnés
de ce qu'ils perdent toujours, parce
\folio{119}
qu'ils sont nés malheureux, et ils n'en
jouent pas moins. Pour se faire en%
tendre de ceux qui tiennent de bonne
foi un pareil langage, dicté souvent
par l'humeur ou l'impatience, il fau%
drait leur donner quelques facultés
intellectuelles que la nature leur a
refusées, ou la première instruction
que reçoit le jeune âge ; pour moi qui
n'ai pas entrepris cette tâche, je me
contenterai de leur dire qu'ils pren%
nent des effets pour des causes ; qu'il
n'y a de bonheur que pour ceux qui 
ont gagné, de malheur que pour ceux
qui ont perdu ; que le gain et la perte
viennent de causes que personne ne
peut ni prévoir, ni éviter, et qu'il ne
manque rien à l'aveuglement de cette
puissance à la disposition de laquelle
les joueurs mettent leurs destinées.

\chapter
  [De l'action du Gouvernement sur les Jeux. Des lois prohibitives]
  {De l'action du gouvernement sur les jeux : des lois prohibitives}

\lettrine{L}{'ordre} public, l'intérêt des mœurs
et la sureté des fortunes, rendent
nécessaire l'action du gouvernement
sur les jeux de hasard, en quelques
lieux qu'ils se jouent.

Observons qu'il ne s'agit point de
ces simples amusemens, qui se con%
centrant dans le sein des familles et
des amis, sont hors du domaine de
l'autorité : on doit sans doute en être
affranchi lorsqu'on ne fait rien qui
porte préjudice à la chose commune
et aux droits d'autrui ; et les parti%
culiers qui ne s'écartent pas de ces
\folio{121}
justes règles, peuvent se livrer li%
brement à leurs goûs, à leurs vo%
lontés, même à leurs passions ; mais
ici l'abus est si près de la chose, les
infidélités sont si aisées à commettre
et auraient de telles conséquences ;
tant de désordres sont nés de ces
sources impures, que le gouverne%
ment, gardien et conservateur de la 
morale et de la fortune publique,
trahirait un de ses principaux de%
voirs, s'il se montrait étranger à des
actions qui ont sur elles une si grande
influence. C'est ici sur-tout que des
étincelles inaperçues ou négligées
causeraient un violent incendie.

Je lis dans Barbeyrac, un des écri%
vains qui ont traité les joueurs avec
le plus d'indulgence : \frquote{La faculté
d'avoir à point nommé le moyen
\folio{122}
de satisfaire un désir innocent,
mais sujet à mener au crime, est
une tentation très-dangereuse, et
contre laquelle on ne saurait trop
se précautionner.}

Mais quelle sera la nature de cette
action du gouvernement sur les jeux
de hasard ?

Il me semble qu'elle doit être l'exer%
cice habituel d'une surveillance ac%
tive, et que, s'adaptant aux localités
et aux circonstances, elle doit admet%
tre plutôt des règles particulières d'or%
dre et des mesures de répression, que
des lois générales et prohibitives.

Je m'expliquerai sur le danger que
je vois dans ces lois, sur-tout sur
celui que je crois qu'elles auraient
aujourd'hui, et je dirai pourquoi je
regarde comme nécessaire que le
\folio{123}
Gouvernement étende dans tous les
lieux son action sur les jeux de ha%
sard, lorsque j'aurai fait connaître
l'état actuel de ces jeux, soit dans
les maisons publiques, soit dans les
maisons particulières. Je me conten%
terai de jeter ici un coup-d'œil sur
une partie des lois et des réglemens
de cette nature, portés contre les jeux
sous différens règnes, et sur les effets
qu'elles ont produits.

Chez les Grecs, les joueurs étaient
flétris, et il était enjoint aux citoyens
de dénoncer ceux qui jouaient furti%
vement.

A Rome, il y eut un sénatus-con
sulte qui ne défendit de jouer de l'ar%
gent qu'aux jeux qui avaient pour
objet l'exercice du corps et qui étaient
utiles pour la guerre. Il n'était per%
\folio{124}
mis d'y jouer que son écot dans un
festin, ou des raffraîchissements. De%
puis, Charles~IX, Roi de France,
voulut qu'on ne jouât que des oublis ;
et un duc de Savoie, que des épin%
gles.

Chez les Romains, dont on se plaît
tant à citer les lois, quiconque don%
nant à jouer perdait le droit de citoyen,
et restait à la merci de ceux à qui il
avait gagné de l'argent.

Sous Cicéron, ceux qui étaient
reconnus pour joueurs, n'étaient pas
admis à se plaindre des insultes qu'on
leur faisait ou du dommage qu'on leur
causait.

Les pères de l'Eglises et les Conciles
ont constamment lancé leurs foudres
contre le jeu. Le cardinal Pierre Da%
mien, au onzième siècle, condamna
\folio{125}
un évêque de Florence, pour avoir
joué dans une auberge, à réciter
trois fois de suite le Psautier, à laver
les pieds de douze pauvres, et à leur
compter un écu par tête.

Justinien ordonna qu'on ne pût
jouer plus d'un écu par partie. Il
accorda pendant cinq années le droit
de réclamer juridiquement contre les
gains des joueurs ; et lorsque personne
ne se présentait, le trésor public pro%
fitait de la confiscation.

Les Rois de France, d'Espagne,
d'Angleterre, et tous les potentats
de l'Europe ont sévi contre les jeux.
Charlemagne, Louis le Débonnaire,
S.~Louis et Charles~V, se sont signa%
lés dans cette tâche honorable, mais
difficile.

Pour seconder les intentions de
\folio{126}
Charles~V, le prévôt de Paris rendit,
en janvier~1397, une ordonnance
dans laquelle il déclarait qu'en inter%
rogeant les criminels, il avait décou%
vert que la plupart des crimes ve%
naient du jeu.

Charles~VIII, par une ordonnance
de 1485, permit seulement aux per%
sonnes de distinction arrêtées pour
des causes légères, de jouer au tric-
trac et aux échecs.

En 1532, François~I\ier, instituteur
d'une loterie qui ne fut pas tirée,
parce que le peuple ne fut pas dupe
des motifs qui la faisaient créer, se
contenta, par un édit donné à Châ%
teaubriant en 1532, de condamner
quiconque jouerait contre des comp%
tables, à restituer le double de ce
qu'il leur aurait gagné.

\folio{127}
Louis~XIII ne fut pas plutôt sur
le trône, qu'il fit une déclaration
énergique contre les brelans, les aca%
démies de jeu, etc. Il déclara \emph{infa%
mes}, intestables, et incapables de
tenir jamais offices royaux, quicon%
que se livrait aux jeux de hasard.
Il voulut en outre que l'argent et les
effets mis au jeu, fussent saisis au
profit des pauvres. Cette déclaration
fut suivie de plusieurs ordonnances
sur le même objet.

Sous Louis~XIV, plus de vingt
ordonnances, déclarations ou édits,
furent publiés contre les jeux de
hasard. \emph{La bassette et le hoca} furent
sur-tout défendus sous les peines les 
plus graves.

Sous Louis~XV, les permissions
de jeux éprouvèrent seulement quel%
ques restrictions ; et ce qu'on a le
plus remarqué, est une ordonnance
rendue le 6~mai~1760, par les maré%
chaux de France, portant qu'ils n'au%
raient aucun égard aux demandes 
qui leur seraient adressées pour des
créances procédant de pertes faites
au jeu, excédant la somme de mille
livres.

Dans les Etats voisins de la France,
des lois de même nature ont été por%
tées contre le jeu.

En Angleterre, pour arrêter ce
désordre dans les derniers rangs de
la société, Henri~VIII défendit aux
artisans, sous peine de prison, de se
livrer, excepté pendant les fêtes de
Noël, aux jeux qui de son tems étaient
en vogue.

Georges~III, par un statut qui con%
\folio{129}
firme cette défense, inglige les mêmes
peines à ceux qui donnent publique%
ment à jouer aux domestiques.

\frquote{Si quelqu'un, dit Charles~III, soit
en jouant, soit en pariant, perd
plus de cent livres, je le dispense
du paiement : je condamne son ad%
versaire à compter le triple de la
somme gagnée, moitié à la cou%
ronne, moitié au dénonciateur.}

La reine Anne déclare nuls et de
nul effet les billets, l'argent prêté,
et tous les engagemens contractés au
jeu : elle donne action au perdant
contre le gagnant, et, au défaut de
ce dernier, à quiconque voudra pour%
suivre le délit, adjugeant à celui-ci
le quintuple de la somme perdue. Ce
qu'il y a de plus remarquable, c'est
qu'elle permet à ceux qui sollicitent
\folio{130}
la confiscation des gains faits au jeu,
de prendre à serment l'infracteur,
de quelque qualité qu'il soit, voulant
que les actions de cette nature sus%
pendent les privilèges des membres
du parlement. Si des joueurs infidèles
gagnent plus de dix livres, soit en
argent, soit en effets, elle les con%
damne à rendre le quintuple, les sou%
mettant d'ailleurs à des notes d'in%
famie et à des peines afflictives.

Georges~II condamna les moteurs
de différens jeux, par lesquels on
cherchait à éluder les défenses, à cinq
cents livres d'amende, leurs dupes à 
cinquante. Tout ce qui équivalait aux
loteries, comme le pharaon, la bas%
sette, etc., fut défendu par un grand
nombre de statuts.

Georges~II défendit aussi, sous peine
\folio{131}
de deux cents livres d'amende, de
parier plus de cinquante livres aux
courses de chevaux.

Le gouverneur de Rome, en~1776,
a rendu une ordonnance contre les
jeux de hasard.

Le roi de Prusse, en~1777, a re%
nouvelé les anciens édits contre les
joueurs. Ils étaient condamnés à trois
cents ducats d'amende ; faute de paie%
ment, ils devaient être détenus à la
forteresse de Landau pendant trois
mois, et n'y vivre que de pain et d'eau.

Au Japon, quiconque risque de
l'argent aux jeux de hasard, doit être
\emph{puni de mort!\ldots}

Je reviens aux lois rendues en
France.

Suivant l'article~15 du titre~19, et
l'art.~28 du titre~20 de l'ordonnance
\folio{132}
du roi du 1\ier~mars~1768, les officiers
généraux et commandans de place
sont tenus d'empêcher que les troupes
qui sont sous leurs ordres jouent aux
jeux de hasard.

Tout officier qui joue malgré cette
défense, doit être mis la première fois
en prison pour trois mois ; la seconde
fois pour six mois ; la troisième, il
doit être cassé et renfermé dans une
citadelle.

Suivant un autre article, les sol%
dats, cavaliers ou dragons tenant des
jeux défendus, doivent être condam%
nés suivant la rigueur des lois, et les
joueurs doivent subir quinze jours
de prison ; et suivant l'art.~16, un des
plus importans, puisqu'il concerne
toutes les classes de la société, les com%
mandans doivent s'informer quels ha%
\folio{133}
bitans donnent à jouer aux jeux dé%
fendus, les faire arrêter et punir sui%
vant l'exigeance des cas.

Par arrêt du 16~décembre~1780,
le parlement de Paris a défendu les
jeux de hasard et les académies de jeu,
à peine de trois mille livres d'amende.

Il avait déjà ordonné l'exécution
des anciens arrêts et ordonnances sur
les mêmes objets.

Le 1\ier mars 1781, le roi en a aussi
rappelé les dispositions, et en a or%
donné l'exécution rigoureuse.

Sont réputés prohibés les jeux dont
les chances sont inégales, et qui pré%
sentent des avantages certains à l'une
des parties au préjudice de l'autre : les
banquiers condamnés et par corps, en
trois mille livres d'amende, les joueurs
en mille livres ; après deux condam%
\folio{134}
nations, punis de peines afflictives
et infamentes.

Presque toutes ces lois ont déclaré
nuls les billets, promesses et autres
actes ayant le jeu pour cause.

L'assemblée nationale, par un dé%
cret du 22~juillet~1791, a formé le der%
nier état de la jurisprudence des jeux.

Les jeux de hasard où l'on admet ou
le public ou les affiliés, sont défendus.

Des amendes sont prononcées contre
les propriétaires ou principaux loca%
taires des maisons de jeu qui n'ont
pa averti la police : elles sont la pre%
mière fois de 200~liv., la seconde de 
1,000 livres.

Les officiers de police peuvent en%
trer en tout tems dans les maisons de
jeu, ur la désignation donnée par
deux citoyens domiciliés.

\folio{135}
Suivant l'art.~36 du titre~2 du Code
de police correctionnelle, les teneurs
de maison où le public est admis,
sont punis d'une amende de 1,000 à
3,000~liv., de la confiscation des fonds
trouvés exposés au jeu, et d'un em%
prisonnement qui ne peut excéder une
année. En cas de récidive, l'amende
est de 5,000 à 10,000~liv., et l'em%
prisonnement peut être de deux an%
nées.

Les teneurs de jeux pris en flagrant
délit, doivent être arrêtés et conduits
devant le juge-de-paix.

Voilà beaucoup de lois : que rap%
porte-t-on des effets qu'elles ont
produits ? et quel sont ceux que nous
avons vus nous-mêmes ?

On a déjà pu s'en faire quelqu'idée,
par ce que j'ai dit des progrès de la
\folio{136}
fureur du jeu chez les peuples où ces
lois ont été portées.

A Athènes et à Rome, lorsque
l'aéropage et le sénat se signalaient
de part et d'autre par la censure des
vices, les magistrats en donnaient
eux mêmes l'exemple.

Les Grecs, pour éviter les dénon%
ciations, partaient d'Athènes et al%
laient à Scyros dans le temps de
Minerve.

Le clergé, qui s'est tant déchaîné
contre le jeu, a le plus participé à
ses excès.

Les grands vassaux, la noblesse et
le clergé rendaient nulles toutes les
mesures prises par Charles~V pour
enchaîner le jeu et les joueurs.

Les lois de St.-Louis, de Charles~V,
de Henri~VIII, de la reine Anne, etc.,
\folio{137}
se sont abolies, parce que la difficulté
ou l'impossibilité de leur exécution
les a fait négliger.

Le frère de Saint-Louis bravait, en
jouant, des lois impuissantes.

Les grands seigneurs, sous Louis~%
XIII, r'ouvrirent les jeux que ce roi
était parvenu à faire fermer. Les lois
ne portèrent que sur quelques plé%
béiens obscurs. Les joueurs, compri%
més quelque tems, devinrent bientôt
plus effrénés et plus nombreux.

Sous le règne suivant, pour éluder 
impunément les termes de la loi, il a
suffi de déguiser les jeux prohibés sous
d'autres noms et sous d'autres for%
mes.

\emph{On se cachait}, dit Dussaulx, \emph{mais
on n'en jouait que plus gros jeu.}

Les princes, les ministres, fatigués
\folio{138}
de contredire \emph{un penchant si général,}
ont fait grace à cette antique manie.
Souvent ils en ont fait un objet de
spéculation.

Suivant Blackston, qui a fourni à
Dussaulx ce que j'ai cité des lois ren%
dues en Angleterre pour la repres%
sion des jeux, les joueurs anglais ont
aussi trouvé le moyen de se soustraire
aux châtimens. Les jeux, les loteries
se multipliaient impunément dans les
tems même où l'on paraissait le plus 
sévir contre eux.

Qu'on n'oublie point le trait que
j'ai rapporté du vieux et plus sage
magistrat du parlement de Bordeaux,
dont la réputation ne souffrait pas de 
ce qu'il risquait au jeu, dans une
soirée, tout ce qu'il possédait : et le
parlement de Bordeaux s'était dis%
\folio{139}
dingué par des ordonnances contre
les jeux !

Lorsque les vingt déclarations ou
ordonnances de Louis~XIV eurent
été portées, les trois quarts de la na%
tion ne respirèrent plus qu'après le
jeu : et il est bon de remarquer que
c'est toujours à la suite des lois pro%
hibitives, que les jeux ont repris le
plus d'activité.

Comme les différens symptômes
d'immoralité et de corruption se ma%
nifestent à la fois, tandis que le sys%
tème de \emph{Jean-Law} se propageait,
des ministres, des magistrats, loin de
faire exécuter les lois contre les jeux
publics, les permirent généralement.
Qu'on ne s'étonne donc pas de la
prospérité des jeux dans les hôtels de
Gêvres et de Soissons !

\folio{140}
Comprimait-on réellement les
joueurs, ils s'expatriaient : ainsi des
Vénitiens, à l'exemple des joueurs
émigrés d'Athènes et retirés à Scyros,
se sont réfugiés en France et en An%
gleterre, pour y savourer sans con%
trainte l'horrible volupté des jeux.

S'il y avait à l'époque où Dussaulx
a écrit son livre sur le jeu, \emph{cent mai%
  sons connues où l'on se ruinait tous
  les jours, et dix fois plus de réduits
  subalternes où l'on se ruinait, que
  l'on n'en comptait sous Henri~IV,
  sous Louix~XIV, et du tems de la
régence,} on est dispensé de recher%
cher davantage si les lois prohibitives
sur les jeux ont des effets salutaires.
Ce que nous avons vu après que l'as%
semblée nationale eut aussi participé
à la gloire de faire des lois aussi bril%
\folio{141}
lantes en motifs que stériles en résul%
tats, n'achève-t-il pas d'éclairer cette
question ?

Il est inutile d'observer que des lois
que j'ai citées, les unes sont ridicules
et bizarres, les autres injustes ou in%
convenantes ; plusieurs sont aussi
cruelles ou aussi immorales que les ha%
bitudes qu'elles tendaient à détruire ;
d'autres enfin, n'étaient que des pri%
vilèges exclusifs accordés à une classe
particulière d'hommes, pour qu'ils
pussent porter impunément, par
l'exemple du vice le plus séduisant, les
dangereuses tentations du jeu dans la
partie la plus saine et la plus utile du
corps social.

\chapter{État du jeu en 1803}

\folio{142}
\lettrine{L}{orsque} les français, n'étant plus
comprimés par la terreur, commen%
cèrent à reprendre avec leur caractère
une partie de leurs anciennes habitu%
des, lorsque l'argent rentra dans la
circulation, par-tout on se dédomma%
gea, comme à l'envi, de la privation de
presque toutes les jouissances ; et la
foule des joueurs s'empressa de venir
se ranger autour des tapis de jeux de
hasard. Ceux dont l'état était perdu
ou la fortune altérée, ceux qui n'a%
vaient pas réussi dans leurs spécula%
tions ou dans leurs intrigues, espé%
raient y trouver un prompt moyen de
\folio{143}
réparer leurs pertes ou leurs fautes,
ou d'alimenter leur soif de l'or : des
hommes nouveaux, embarassés de
l'emploi de ce qu'ils avaient gagné dans
un facile agiotage, et aussi peu pro%
pres aux plaisirs ordinaires de la so%
ciété, qu'aux travaux qui demandent
des connaissances, étaient jaloux, par
de fortes mises, d'acquérir aux jeux
publics une sorte d'importance. Cette
fureur de jeu effaça tous les excès dont
Dussaulx avait fait l'effrayante pein%
ture. Le sol français avait été couvert
de bastilles et de tombeaux ; il le fut
de théâtres, de salons de bals et de
maisons de jeu. Les principales villes
des départemens ont eu, comme à
Paris, des parties considérables qui se
sont tenues soit dans des cafés, soit aux
lieux connus auparavant par le nom
\folio{144}
d'académies, soit dans des maisons
bourgeoises, dont les maîtres, réduits
par la révolution à ces propriétés et
à un stérile mobilier, n'ont pas cru
pouvoir trier un meilleur parti. Com%
munément on n'entrait là que par ca%
chets, ou par invitations, ou par pré%
sentations d'affiliés ; mais il n'était pas
difficile d'obtenir ce droit d'entrée. A
la tête de ces maisons étaient des
dames, dont le nom était une sorte de
garantie qu'on trouverait chez elles
de la probité ! Elles attiraient l'habi%
tant et l'Etranger par des repas et des
bals. La police civile ou militaire leur
accordait appui et protection, moyen%
nant une faible rétribution, destinée
au soulagement des pauvres, ou à des
actes de bienfaisance.

Le pharaon, le biribi, le trente-un,
\folio{145}
le passe-dix, ou le pair et l'impair,
sont devenus successivement les jeux
à la mode. Des banquiers ambulans
se transportaient où ils étaient de%
mandés, avec les instruments de jeu,
et le plus souvent en faisaient les
fonds. Mais Paris est toujours resté
le grand théâtre des jeux ; et certes,
à l'époque où ils ont repris leur acti%
vité, à laquelle la rentrée des troupes
dans l'intérieur a beaucoup contri%
bué, il aurait été aussi impolitique
que difficile de les prohiber, ou d'y
mettre de trop fortes entraves. Ils ont
été tolérés, surveillés, et, comme ce%
la devait être, mis à contribution.

Au milieu de ces jeux dont je viens
de parler, s'est élevé le plus impo%
sant et le plus séducteur de tous, \emph{la
roulette.} Les tables s'en sont multi%
pliées, et ont fait déserter une foule
de maisons particulières et de tri%
pots établis ouvertement ou clandes%
tinement, mais qui n'offraient pas
ou autant de liberté, ou autant de
sureté.

L'administration publique, forcée,
pour ainsi dire, de capituler avec les
passions, a tâché de donner de la
fixité aux maisons en possessions d'at%
tirer le plus grand nombre de joueurs,
en leur accordant des privilèges, et
prenant des mesures pour que les
vols, les friponneries, les querelles n'y
fussent point impunis. Si cet ordre et
cette sûreté devaient avoir pour effet
d'accroître le nombre de joueurs,
ils devaient aussi avoir l'avantage
d'assembler sur quelques points ce
peuple cupide, inquiet et efferves%
\folio{147}
cent, jusques-là inégalement répandu
dans des repaires.

Mais bientôt la France ayant joui
d'un système mieux combiné d'admi%
nistration intérieure, et une unité de
mesures ayant dû être adoptée pour
qu'on pût rapprocher davantage de
l'{\oe}il du Gouvernement ce qui inté%
resse la tranquilité et les m{\oe}urs, les
jeux dans Paris ont été affermés, et la
ferme a été chargée de les administrer
de manière qu'en conservant aux ci%
toyens la liberté de leurs actions, et des
habitudes dont la subite réforme au%
rait pu être mise au nombre des rêves
politiques, cette antique manie fût
purgée de beaucoup d'abus et de 
beaucoup d'excès que l'autorité n'au%
rait pu directement atteindre.

\folio{148}
De cette manière, si la fureur du
jeu n'a point été altérée, du moins les
jeux ont été administrés. Leur état
est aujourd'hui à-peu-près tel que je
viens de le décrire
\footnote{Il faut se reporter à cette année de 1803, dans le
  cours de laquelle j'ai composé et publié ces Considé%
rations.} ; on joue beau%
coup, parce qu'on sent fortement le
besoin de jouer, parce que c'est un
goût né du caractère indestructible
et de la position de beaucoup d'indi%
vidus. L'ardeur du jeu a depuis quel%
ques années gagné dans Paris presque
toutes les classes de la société, et ce
n'est pas dans les maisons publiques
que se joue le plus gros jeu. Chez des
particuliers, on joue entre prétendus
ami ce qu'on appelle un jeu infernal.

La roulette, le trente-un, le pair et
\folio{149}
l'impair, se jouent dans les maisons
publiques de jeu. Depuis quelque
tems, on a réduit de plus de moitié
le nombre de ces maisons : on asure 
que le nombre des joueurs est aussi
diminué, et que les parties ne sont
plus aussi fortes ; mais on ne dit pas,
ou on ne sait pas que depuis la réduc%
tion du nombre des maisons de jeu,
la bouillotte seule, dont les parties se
sont formées librement dans des mai%
sons particulières \paren{à la plupart des%
quelles le nom de tripot pourrait fort
bien convenir} absorbe peut-être plus
d'argent qu'il ne va s'en perdre dans
les maisons qui ont le droit exclusif
d'attirer le public.

Mais il me semble que je ferais mal
connaître l'état actuel du jeu, si je
n'entrais séparément dans des détails
particuliers aux différens lieux où
l'on joue. Ils peuvent être rangés sous
trois dénominations : maisons publi%
ques, maisons particulières, tripots.

\chapter
  [Maisons publiques de Jeu]
  {Maisons publiques de jeu \footnote{En 1803.}}

\folio{151}
\lettrine{O}{n} a eu, ou on doit avoir du pour
principal but, en établissant des mai%
sons publiques de jeu, de donner un
contre-poids aux dangers et aux dé%
sordres auxquels le jeu expose dans
des tripots et des maisons particu%
lières.

L'emplacement et le nombre des
maisons publiques sont réglés par
l'autorité surveillante.

Il y a différentes maisons de jeu
pour différentes classes de citoyens ;
la plus marquante est le salon de la
Paix, rue Grange-Batelière. Outre
les jeux dont j'ai parlé et les jeux
de commerce, on y joue aux krabs,
fameux jeu anglais. Ce salon, monté
sur le ton des maisons les plus opu%
lentes, est fréquenté par une société
choisie, dans laquelle se trouvent des
personnes jouissant d'une bonne ré%
putation, et beaucoup d'étrangers
distingués par leur nom, leur rang
ou leur fortune. On y entre par ca%
chets, ainsi que dans la maison des
arcades du Palais Royal, et dans
quelques autres.

La précédente administration avait 
la faculté d'étendre le nombre de ces
maisons, et elle en usait : elle avait
sous elle une administration ambu%
lante, et envoyait des missionnaires
dans les départemens, en Italie, etc.

L'administration des jeux tient pour
son compte, et avec ses propres fonds,
un petit nombre de maisons. D'autres
maisons subalternes ont seulement
des permissions ou privilèges, pour
lesquels une rétribution est payée à
l'entreprise générale, suivant la loca%
lité et la nature du jeu.

On a réduit de plus de moitié le
nombre des maisons où va joueur la
classe ouvrières.

Lorsque dans les maisons particu%
lières, des jours de fête ou de bal,
on veut faire jouer des jeux de hasard,
l'administration y envoie des tables et
ustensiles du jeu demandé, des tail%
leurs, et employés, et même des
fonds. Le minimum de la mise pour
chaque coup dans presque toutes les
\folio{154}
maisons, est de trente sous
\footnote{
  Le minimum de la mise dans les maisons publiques
  \emph{subalternes}, était de trente sous. Depuis quelques années
  il a été fixé à deux francs.
}. Anté%
rieurement, plusieurs avaient la fa%
culté désastreuse de recevoir de moin%
dres mises. Depuis cette rigoureuse
fixation, on voit aux jeux publics
moins d'artisans et de cultivateurs.

La ferme des jeux n'est que pour
Paris ; son bail n'est que pour une
année, et peut se résilier
\footnote{
  Présentement le privilège de la ferme des jeux
  s'étend dans plusieurs lieux où les eaux se prennent.
}.

Pour son bénéfice ou son avantage,
l'administration a, au trente-un, les 
refaits du trente-un ; à la roulette, le
zéro et le double zéro ; au biribi, une
petite colonne particulière ; au pair
et impair, un certain nombre de
\folio{155}
points donnés par les dés. Là-dessus,
je n'ai pas besoin de m'expliquer
mieux ; ceux qui ont le bonheur d'être
étrangers au jeu n'ont pas besoin de
me comprendre : je ne le serai que
trop par ceux qui ont le malheur de
les pratiquer.

Dans les maisons publiques, les fri%
pons n'ont rien à faire contre la ban%
que. Les yeux des surveillans et des
spectateurs tiennent lieu de cons%
cience à ceux qui n'en ont pas.

Les joueurs aussi n'ont rien à re%
douter de la banque ; outre que les
ruses et les fraudes n'y seraient guè%
res au pouvoir des tailleurs les plus
adroits, les joueurs ont une garantie
de leur fidélité dans l'intérêt des pon%
tes, la haine et la jalousie des spec%
tateurs bénévoles qui inclinent ordi%
\folio{156}
nairement contre la banque, et les
regards de tous habituellement fixés
sur les mouvemens de ces tailleurs.
Je ne hasarde rien, sans doute, en
assurant qu'on peut aussi compter sur
la loyauté et la probité des employés
actuels dans les jeux. En général,
ces employés sont scrupuleusement
choisis d'après de rigoureuses infor%
mations. Leur air, leur politesse, leur
langage, annoncent qu'ils sont bien
nés, et ont reçu de l'éducation. La
plupart sont fils de familles ruinées
ou mutilées dans le cours des dé%
sastres publics, ou ont été employés 
dans les armées et les administrations,
et se sont trouvés inoccupés par l'effet
des réformes nécessaires, ou sont
victimes de malheurs particuliers.
L'entreprise des jeux a un si grand
\folio{157}
intérêt à n'y placer que des personnes
sur la probité desquelles elle puisse
compter, qu'à cet égard on ne peut
la soupçonner de négligence. C'est
par de pareils choix sans doute qu'elle
a tâché de se sauver en partie de la
défaveur dont l'opinion frappe des
opérations telles que les siennes. Je
crois donc qu'on pourrait avoir un
motif de plus de confiance dans un
employé qui, après quelque tems
d'exercice, serait sorti sans reproche
d'une administration telle que celle
des jeux. Ce ne serait pas par des
exceptions qu'on serait fondé à ne
pas trouver de la vérité dans mes
observations.

Les maisons publiques, telles qu'el%
les sont aujourd'hui, ont encore d'au%
tres avantages qu'on ne trouverait
pas ailleurs.

\folio{158}
Il y a évidemment moins de dangers
que dans des lieux où se jouent des
jeux qui tiennent beaucoup de l'a%
dresse et de l'habileté. On sait du
moins à quoi on s'expose.

Non-seulement on n'y dépouille
personne de dessein prémédité, mais
ke joueur qui entre et sort librement,
absolument maître de son sort, peut
quitter à volonté sa partie adverse,
sans craindre ni reproches, ni mur%
mures.

Le joueur, s'il est en perte, peut
prendre la revanche contre le ban%
quier, et la refuser s'il est en gain.

Dussaulx paraît attacher du prix à
cet avantage.

Pour quelqu'un qui n'est pas riche
et qui a le malheureux goût du jeu,
ces maisons sont encore préférables
\folio{159}
aux maisons particulières, où l'on
sonde pour ainsi dire la bourse, où
l'on règle sur cette connaissance la
manière dont on doit se comporter 
avec vous, où les égards, d'ailleurs,
sont toujours proportionnés à la for%
tune, et où la fausse honte et la crainte
du discrédit vous font souvent jouer
plus gros jeu que vous n'y seriez na%
turellement porté.

Si on gagne aux jeux publics, on
n'a pas le remords d'avoir soi-même
plongé des malheureux dans la mi%
sère et le désespoir.

Ce n'est pas là aussi que se forment
des liaisons souvent plus nuisibles
que de fortes pertes. Les pontes n'y
ont affaire qu'aux tailleurs, et tout
le tems y est pris par le jeu, qui exige 
le silence.

\folio{160}
A la vérité, les banques y font de
forts gains ; mais les maîtres de mai%
sons particulières ou de tripots ne
lèvent-ils pas sur vous de plus fortes
contributions ? Il faut encore le re%
connaître, ce maisons ne sont pas
sans agrémens pour des personnes
qui, sans manquer de raison et de con%
duite, puevent ne pas aimer les com%
mérages, le cérémonial ennuyeux,
et toutes ces petites malignités qu'on
n'évite point dans les sociétés ordi%
naires. Et, je le répète, il est des 
peronnes à qui la distraction d'un
faible jeu est nécessaire.

Si les maisons publiques ne sont 
pas sans avantage pour les amateurs
du jeu, elles en ont aussi pour les
mœurs et pour l'ordre public. Là,
peu de joueuses osent se montrer, et
\folio{161}
elles ont peu de communication avec
les joueurs. Je crois être dispensé de
dire ce que produit dans d'autres lieux
le mélange de joueurs des deux sexes :
on peut s'en former une idée ; mais
voici des considérations d'une autre
importance. N'est-il pas, je le de%
mande, et moral et politique d'amener
les grands joueurs à comparaître de%
vant le public, pour le rendre le té%
moin de leur audace et le surveillant
de leur loyauté ? N'est-ce pas en for%
cer un grand nombre à la probité et
et à la modération ? Là, du moins,
les fortes pertes, les ruines, sont des
leçons qui profitent et aux joueurs
et à ceux qui seraient tenté de le
devenir. Il n'en est point ainsi des au%
tres rendez-vous de jeu.

Enfin, je dois rendre justice à l'ex%
\folio{162}
cellente police qui s'exerce, pour
ansi dire sans se montrer, dans les
maisons de jeu. Jusqu'ici des maisons
de ce genre n'avaient point donné
l'exemple d'autant d'ordre, de calme
et de décence.

Il y a un commissaire du gouver%
nement près les jeux.

Si mon attachement à la vérité et
à la justice me fait dire ici des mai%
sons de jeux publics ce qui est connu
et ne peut m'être contesté ; si je leur
donne hautement la préférence sur
les autres lieux où l'on joue, je n'en
pense pas moins qu'en y entrant on
s'expose aux plus grands malheurs ;
et je regarderais comme un beau
jour pour la chose publique, celui
où la fureur du jeu s'altérant par
dégrés, et ne trouvant plus d'aliment
\folio{163}
dans les antres qui lui sont ouverts
de tous côtés, il n'y aurait plus ni
obstacle ni danger à les détruire.

\chapter{Maisons particulières où l'on joue gros jeu}

\folio{164}
\lettrine{I}{l} y a à Paris une classe d'hommes
de qui les lumières et l'étude ne sont
pas le partage, et qui exagèrent les
modes, dans la crainte qu'on ne les
soupçonne de ne pas les connaître.
On pense bien que la plupart de ces
gens-là, sachant à peine d'où leur est
venue la fortune, après la renais%
sance des jeux de hasard, n'ont pas
été les derniers à s'y livrer. Ils y ont
joué d'abord par ton, ensuite par
cupidité, enfin par habitude.

Peu de modernes enrichis vont
dans les maisons publiques de jeu :
\folio{165}
ils croient avoir un rang à tenir et
une réputation à conserver ; mais
on en connaît plusieurs dont le jeu,
soit là, soit ailleurs, a vu finir la
métamorphose, et qui sont deve%
nus pauvres aussi rapidement qu'ils
étaient devenus riches.

J'ai dit comment s'étaient établies
des maisons particulières de jeu dans
les départemens : il s'en est formé et
s'en forme encore beaucoup dans
Paris du même genre ; mais celles-ci,
pour n'être point soumises aux re%
gards de la police, n'ouvrent point
leurs portes à tout le monde, et ne
font point distribuer d'adresses ni de
cartes d'invitation à un grand nom%
bre de personnes ; seulement l'ami y
mène son ami ; et ces amis, d'autant
plus attachés l'un à l'autre qu'ils ne
\folio{166}
se connaissent pas, se lient subite%
ment par de tels rapports, qu'ils n'ont
d'autre désir, d'autre but, d'autre 
soin que de s'enlever les uns aux
autres tout ce qu'ils possèdent.

On n'a point, dans ces maisons,
de grandes tables garnies de machines
de différentes formes, qui trahiraient
les intentions des maîtres, ou plutôt
des maîtresses, car ce sont le plus
souvent des dames qui sont à leur tête
ou en font les honneurs : il n'y a que
de simples tables de bouillotte, ou
d'autres qui servent au besoin pour
un vingt-un, un trente-un, un loto
à fortes mises, etc. Il vous serait
difficile d'y échapper aux différens
moyens de séduction qui vous envi%
ronnent. Les jeux à argent s'y con%
fondent parmi d'autres jeux ; souvent
\folio{167}
ils n'y paraissent qu'un amusement
accessoire : ils ne tardent pas à deve%
nir une grande occupation ; et les
cris de joie qui partent d'un salon
voisin accompagnent les cris de dé%
tresse des victimes qui tombent l'une
après l'autre dans un précipice dont
les bords n'en restent pas moins cou%
verts de fleurs.

Vous dites que ce ne sont pas là des
maisons honnêtes : le nom des maîtres
ne s'est-il pas rendu recommandable
par des emplois dans la robe, dans la
finance, dans le militaire ? N'avez-%
vous pas trouvé chez eux, comme on
vous l'avait annoncé, des gens distin%
gués par leurs places, leurs richesses
ou leurs talens ? Le goût, la décence,
la délicatesse n'y brillaient-ils pas
également dans les discours et dans
\folio{168}
la parure des femmes ? Ah ! il n'est
pas moins vrai que ce sont de ces
maisons honnêtes que sortent confu%
sément, comme Dussaulx l'a observé,
le parjure, la misère, l'opprobre, le
duel et la mort.

Les maisons publiques où la plu%
part des joueurs considérés dans le
monde, craignent d'être aperçus, fa%
vorisent certainement moins les ex%
cès du jeu que ces brillans rendez-%
vous de société, où, avec de pareils
goûts, on serait bien fâché de n'être
pas admis.

Il n'y a pas encore de long-tems, on
citait des pertes considérables faites
par des personnes connues, \emph{sur-tout
par des étrangers}, dans ces société
du bon ton.

Les maisons particulières où l'on
\folio{169}
joue gros jeu à Paris, sont de diffé%
rens genres : il y en a, comme des
maisons publiques de jeu, pour les
différentes classes de citoyens. Les
unes reçoivent tous les soirs, les au%
tres donnent un grand dîner un jour
fixe de la semaine. Après le dîner les
parties s'arrangent : ce ne sont pres%
que que des parties carrées, les sim%
ples jeux de commerce. Une bouil%
lotte cependant rassemble ceux qui
n'aiment pas le petit jeu. De ceux-là
quelques-uns, pour faire un double
emploi de leur tems, parient de fortes
sommes à la queue ou aux marqués
d'un piquet qui se joue à côté d'eux à
cinq sols la fiche.

A mesure que les petits jeux finis%
sent, les belles dames se rassemblent
autour de la bouillotte ; elle encou%
\folio{170}
ragent des yeux les joueurs de leur 
connaissance ; celui qui ayant devant
lui une forte somme d'argent, fait ou
tient le tout contre une masse à-peu-%
près égale, s'il gagne, est compli%
menté sur son bonheur, et s'il perd,
est consolé par le titre de beau joueur
qu'on ne peut lui contester.

On vante le sang-froid et la témé%
rité d'un joueur qui se ruine, comme
on vante le calme et le courage d'un
homme condamné qui marche au
supplice.

Il est nuit : les personnes raisonnables
se retirent peu-à-peu ; le nombre des
rentrans à la bouillotte a contraint d'en
former plusieurs tables. Cette dame,
qui au reversis a eu, pour une fiche
à deux sols, une contestation d'un
quart d'heure, s'est cavée de dix louis.

\folio{171}
La bouillotte ne se quitte pas aussi
promptement qu'un autre jeu ; d'ail%
leurs, des perdans trouveraient fort
mauvais qu'on les abandonnât de 
bonne heure. Le jeu se prolonge
donc dans la nuit : il s'échauffe ; les
caves se centuplent ; le tems n'a plus
d'heures ; le jour vient, et les dames
de la maison songent enfin qu'il se%
rait bon de prendre quelque repos.
Hélas ! il n'en est plus pour de nou%
velles victimes que vient de faire ce
jeu de société, qui, plus que tous les
jeux publics, a porté depuis quelque
tems dans les familles, la misère et
la désolation.

Oui, on assure que plus de banque%
routes, de duels, de suicides ont été
causés par ce jeu dans des maisons
particulières, que par tous les jeux
\folio{172}
de hasard dans les maisons publiques.
Dans plusieurs, la première cave est
de cinq louis. On assure que chez des
parvenus, on se cave le plus souvent
de quatre à cinq cents louis : on en
cite même où les caves ont été portées
à mille.

Ce n'est pas seulement à la ville
qu'on joue un jeu énorme chez des
particuliers ; pour être plus à l'aise,
et pour que les femmes ne soient pas
toujours sur les épaules des maris,
on se donne rendez-vous dans des
maisons de campagne, que des par%
venus et des enrichis appellent \emph{leurs
petites maisons}, à l'imitation des 
grands seigneurs et des financiers des
derniers règnes. Là, l'ivresse du vin
accroît l'ivresse du jeu ; là, on con%
vient de jouer jusqu'à extinction de
\folio{173}
bourse ; mais on va plus loin, on joue
jusqu'à épuisement de crédit ; car un
maître de maison , ou un des joueurs,
trouve souvent son compte à prêter
à celui qui est encore dans l'usage et
le pouvoir de rendre, ressource bien
fatale pour celui qui la possède ! Le 
crédit, qui dans le commerce a des
effets bienfaisans, a au jeu les effets
les plus désastreux. Mais des négo%
cians, des banquiers, des agens de
change, qui se garderaient bien de se
montrer dans des jeux publics, jouent
volontiers dans ces parties de cam%
pagne, qu'ils appellent \emph{des parties
fines.} Leurs conventions au jeu se
font comme celles de la bourse ; le
mot suffit.

C'est ainsi que certains, qui avaient
fait quelque tems grand bruit dans la
\folio{174}
banque ou le commerce, \emph{ont joué de
leur reste.}

On sait ce qui à la bouillotte peut
résulter d'une facile intelligence éta%
blie entre quelques joueurs, et s'il t
a des joueurs honnêtes et délicats, on
ne croira pas que c'est parmi ceux
qui jouent le jeu le plus considérable
qu'ils se trouvent le plus.

\emph{Le flambeau}, à la bouillotte, est
d'un produit si abondant, qu'on ne
doit pas être étonné que tant de mai%
sons particulières cherchent à en éta%
blir une. On ne découvre pas le secret
de sa spéculation : ce sont ses an%
ciennes et ses nouvelles connaissances
qu'on est bien aise de réunir de tems
en tems. Les profits, au surplus, ne re%
gardent, dit-on, que les domestiques.

Il y a dans ces spéculations, comme
au jeu, bonheur et malheur.


\backmatter

\tableofcontents

\end{document}
