\documentclass[11pt,twoside,openany]{book}

%%% PACKAGES %%%%%%%%%%%%%%%%%%%%%%%%%%%%%%%%%%%%%%%%%%%%%%%%%%%%%%%%%%%%%%%%%%

\usepackage[utf8]{inputenc}
\usepackage[T1]{fontenc}
\usepackage[osf]{ebgaramond}
\usepackage[francais]{babel}
\usepackage{geometry}     % Configuration de la mise en page
\usepackage{microtype}    % Améliorations typographiques
\usepackage{hyperref}     % Liens hypertextes & métadonnées

%%% CONFIGURATION %%%%%%%%%%%%%%%%%%%%%%%%%%%%%%%%%%%%%%%%%%%%%%%%%%%%%%%%%%%%%

\geometry{
  papersize={9.33cm,15cm}, % In-18 carré
  textwidth=5.3cm,
  lines=20,
  headsep=5pt,
  marginparsep=7mm,
}

\frenchsetup{
  AutoSpacePunctuation=false,
  og=«, fg=»,
}

\hypersetup{
    colorlinks=true,
    linkcolor=black,
    pdftitle={Considérations sur le jeu, les joueurs, la théorie des jeux, etc.},
    pdfauthor={Lablée, Jacques (1751-1841)},
    pdfproducer={Éric Guirbal},
}

%%% MACROS POUR LA SAISIE DU TEXTE %%%%%%%%%%%%%%%%%%%%%%%%%%%%%%%%%%%%%%%%%%%%

% Folio du document source dans la marge
\newcommand{\folio}[1]{\marginpar[\raggedleft\scriptsize#1]{\scriptsize#1}}

% Lettrines
\newcommand*{\lettrine}[2]{\noindent{\Large#1}\textsc{#2}}

% Texte entre parenthèses
\newcommand*{\paren}[1]{(~#1~)}


%%% DÉBUT DU DOCUMENT %%%%%%%%%%%%%%%%%%%%%%%%%%%%%%%%%%%%%%%%%%%%%%%%%%%%%%%%%

\begin{document}

\frontmatter

\chapter{Introduction}

\folio{i}
\lettrine{D}{ans} le roman de la Roulette
\footnote{
  La sixième édition de cet ouvrage, avec
  un tableau gravé et colorié, représentant un
  jeu de roulette, se vend chez Rapet, commis%
  sionnaire en librairie, rue Saint-André-des-%
  Arcs, n.~41. Prix : 1~franc 5o~centimes ;  et
  franc de port, 1~fr. 75 cent.},
j'ai parlé le langage du sentiment ;
j'ai tâché d'offrir des tableaux qui
pussent émouvoir l'imagination ;
j'ai peint un joueur en action,
\folio{ii}
entraîné, comme ils le sont pres%
que tous, par des erreurs de cal%
culs, et livrés aux illusions d'une
passion désastreuse. Quel écrivain,
ami de l'humanité et des mœurs,
n'est pas pressé par le besoin d'ar%
rêter dans leur cours les vices et les
excès qui leur portent le plus d'at%
teinte, et d'attirer tous les regards
sur les pièges tendus à la crédulité,
à l'ignorance et à la faiblesse ?

Je vais encore m'occuper du
même sujet : j'en parlerai dans les
\folio{iii}
mêmes sentimens et dans les
mêmes principes ; mais je le con%
sidérerai sous d'autres rapports.
Je réprimerai tout mouvement
passionné ; et, recherchant plus
ce qui est vrai que ce qui peut
faire sensation, je donnerai à mes
idées un développement plus mé%
thodique. Moins jaloux d'émou%
voir, que d'éclairer et de con%
vaincre, je m'adresserai moins au
cœur qu'à l'esprit ; en un mot,
j'écrirai plutôt sur le jeu que con%
tre le jeu. Lorsqu'on suit un pareil
\folio{iv}
plan, si les effets qu'on produit
sont moins brillans et moins vifs,
ils peuvent être plus sûrs et plus
durables.

Certes, je ne serai jamais l'apo%
logiste des goûts et des habitudes
du jeu ; mais il me semble que la
meilleure manière de les com%
battre, n'est pas d'annoncer d'a%
bord le vœu et l'intention de les
détruire ; et s'il est vrai que leur
destruction soit regardée comme
impraticable, n'y a-t-il pas quel%
\folio{v}
que chose de mieux à faire que
d'appeler sur les joueurs et sur les
lieux qui les rassemblent, le mé%
pris et la proscription ? On a su
extraire des sucs bienfaisans de
plantes vénéneuses ; ne peut-on
enlever au jeu ce qu'il a de plus
dangereux et de plus funeste ?
n'en peut-on du moins tirer quel%
ques fruits ? Ce désordre, ces
pertes sont-ils sans dédommage%
mens, sans compensations ? et n'y
a-t-il aucun bien à côté d'un si
grand mal ?

\folio{vi}
Il faut parler aux hommes éga%
rés par des passions un langage
qui leur soit familier, ou qu'ils
puissent entendre ; il faut comp%
ter, pour ainsi dire, avec eux,
dans leurs propres affaires ; ainsi
on se rend maître de leur atten%
tion, ce qui est déjà avoir beau%
coup obtenu : ils peuvent alors
apercevoir eux-mêmes le danger
qui les menaçait, le précipice
dans lequel ils allaient tomber ;
alors il est plus facile de leur faire
quitter la ligne sur laquelle ils
\folio{vii}
étaient placés, et de les attirer sur
un point qui concilie mieux leurs
intérêts et leurs goûts.

Je vais donc tâcher d'alléger
le poids énorme qu'un destin
aveugle fait peser sur les joueurs ;
et en examinant ce qui doit leur
être ôté, et ce qu'il convient en%
core de leur conserver, je m'ap%
plaudirai, si je peux aussi, par de
faibles, mais nouveaux aperçus,
aider l'administration publique à
remplir un de ses devoirs les plus
difficiles.

\folio{viii}
Je garderai le silence sur les
désagrémens et les défaveurs que
m'ont causé mes ouvrages contre
les jeux. Il en coûte souvent pour
faire connaître d'utiles vérités,
mais les écrivains moraux rem%
pliraient-ils leur devoir s'ils ne
savaient faire le sacrifice de leur
intéret personnel ?


\mainmatter

\chapter{
  De l'ouvrage intitulé : \emph{De la Passion
  du Jeu}, par \bsc{Dussaulx}%
}

\folio{1}
\lettrine{L}{e} bon, l'honnête Dussaulx a fait
sur la passion du jeu un traité histo%
rique et moral, qui est ce que nous
avons de plus complet, de mieux
pensé et de mieux écrit sur cette ma%
tière. On y remarque une érudition
\folio{2}
facile, des anecdotes curieuses et ins%
tructives, des réflexions originales,
piquantes et quelquefois profondes,
une sorte de verve poétique, et des
vues portant le cachet d'un bon es%
prit et d'un bon cœur : on partage
la juste indignation qu'excitent dans
l'âme de l'auteur les excès du jeu et
le crime de ceux qui les favorisent ;
mais souvent l'on sourit à sa con%
fiante bonhomie. A-t-il pu croire que
l'énergie des passions cupides céde%
rait à des moralités et à des citations ?
En le lisant avec attention, on doute
qu'il s'en soit flatté ; mais si on n'y
trouvait de ces pensées fortes, de ces
traits qui caractérisent la vraie sen%
sibilité et le besoin de la répandre,
on serait tenté de penser que l'au%
teur a voulu faire plutôt un ouvrage
\folio{3}
savant, curieux, orné des fleurs de
l'éloquence, qu'un ouvrage dont on
pût retirer beaucoup de fruit. On le
voit plus appliqué à peindre le mal
qu'à en indiquer le remède. Lui-%
même il parle de l'inutilité des lois et
des efforts des gouvernemens contre
cette fureur aveugle ; il dit et il prouve
que le jeu a dans tous les lieux et dans
tous les tems subjugué l'esprit des
hommes de toutes les classes. « Par%
courez la terre depuis le Japon
jusqu'à l'extrémité du nouveau
monde, quels que soient le culte,
les lois et les opinions, vous trouve%
rez des joueurs dans les climats
glacés et dans les climats brûlans ».

Voilà ce que dit Dussaulx ; et pour
démontrer par les faits que cette épi%
démie universelle est indestructible,
\folio{4}
je n'aurais besoin que de reproduire
ceux qu'il rapporte.

Je citerai plus d'une fois cet au%
teur, le seul qui chez nous se soit fait
entendre sur cette matière. N'ayant
pour but que d'offrir la vérité, je ne
dois rien négliger de ce qui me semble
propre à la faire connaître. C'est dans
cet esprit et dans cette obligation que
je relèverai aussi les défauts et les er%
reurs qui m'ont frappé dans l'ouvrage
dont il s'agit.

Dussaulx prononce d'abord trop
fortement son intention de peindre
les joueurs et leur manie avec les
couleurs les plus noires ; il commence
par les dévouer à la haine et à la pros%
cription ; et à ce sujet il s'exprime
ainsi : « Si les écrivains ont montré
les côtés séduisans du jeu, ils sont
\folio{5}
des corrupteurs ; s'ils n'en ont ex%
primé que la difformité, ils sont
les vrais amis de l'humanité ».

Aussi représente-t-il souvent les
joueurs moins tels qu'ils sont, que
tels qu'il faut qu'ils soient pour pa%
raître odieux. Voilà bien ce qui con%
vient pour faire briller le talent d'un
écrivain, pour qu'il puisse donner à
ses sentimens un développement éner%
gique ; mais ce n'est pas la meilleure
règle d'instruction. Sans doute c'est
en offrant aux hommes le flambeau
de la vérité, c'est en les éclairant qu'on
sert le mieux leurs intérêts. Si vous ne
leur montrez qu'un côté des choses,
et qu'ils viennent à découvrir celui
que vous voulez leur cacher \paren{ici ils
le découvriront ; ils l'ont même déjà
découvert, car sans cela vous n'au%
\folio{6}%
riez pas de leçons à leur faire}, ils
seront en droit de se plaindre de ce
que vous avez voulu les tromper ; ils
vous accuseront de mauvaise foi, et
n'auront plus de confiance dans ce
que vous leur direz.

Les écrivains moraux ont besoin,
pour fixer nos idées, d'un grand ca%
ractère d'impartialité et de désinté%
ressement. Dussaulx, par la sorte d'en%
gagement qu'il a pris dès le commen%
cement de son livre, manque une
partie des effets qu'il pouvait pro%
duire. On prévient, on devine sa pen%
sée ; on va jusqu'à suspecter la vérité
de ses tableaux ; ce n'est plus qu'un
avocat qui, dans un procès impor%
tant, rassemble tout ce qui lui paraît
favorable à sa cause, crie contre ses
adversaires, exagère ses accusations,
\folio{7}
ses reproches. On l'écoute ; il inté%
resse ; mais pour fixer son opinion,
on attend que ses adversaires lui aient
répondu.

Ainsi, Dussaulx, en voulant trop
prouver les dangers du jeu, a paru
mettre en question une vérité géné%
ralement sentie.

Pour prendre plus d'avantage sur
les joueurs, il en a trop simplifié le
caractère, et il a commis évidemment
une erreur, en considérant moins le
jeu comme une de ces passions inhé%
rentes, pour ainsi dire, à la faiblesse
humaine, et qu'il est plus facile d'é%
nerver, de diriger, que de détruire,
qu'en le considérant comme un vice
absolu, déterminé, sur lequel la loi
pouvait avoir une action directe, ou
auquel on pouvait appliquer, comme
\folio{8}
à des maux connus, des remèdes gé%
néraux. Je ferai voir que le caractère
du Joueur, extremement composé,
tient à différentes causes qui le mo%
difient et qu'il faudrait connaître,
pour être en état d'employer à la gué%
rison du mal des remèdes particuliers.

Mais comment concilier les diffé%
rentes idées que Dussaulx donne du
jeu et des joueurs ?

Il dit : « Il s'agit ici d'un vice pur
et sans mélange. Quel joueur a le
droit de s'estimer ? Un joueur ! ce
titre est une insulte ».

Et ailleurs : « La manie du jeu roule
sur trois pivots, la sottise, la fu%
reur et la fourberie ».

Et ailleurs : « On citerait moins de
joueurs sensibles que de bourreaux
compâtissans ».

\folio{9}
Et ailleurs : « Les joueurs manquent
de sensibilité comme de probité. »

Enfin, avec Aristote, il refuse aux
joueurs toutes les qualités du cœur.

Cependant il met au rang des plus
grands joueurs, les hommes doués
de plus d'imagination ; et parmi ces
grands joueurs, il cite d'excellens
hommes, tels que Caton, Henri~IV,
Montaigne, Descartes, Collardeau et
lui-même.

Il dit aussi que la fureur du jeu,
par un alliage monstrueux, se joint
quelquefois à de grands talens et à
de grandes vertus.

Et il rend encore moins effrayante
la laideur de ses portraits, en obser%
vant que l'ennui fait plus de joueurs
que la cupidité ; que le goût du jeu
\folio{10}
est quelquefois moins un symptôme
de cupidité que d'ambition.

Toutes ces contradictions sont-%
elles assez évidentes ? II est vrai que,
pour se mettre à l'abri du reproche
d'avoir désigné les joueurs par d'o%
dieuses qualifications, il applique,
vers la fin de son livre, ce qu'il en a
dit aux joueurs de profession ; mais
cette explication prudente et tardive
est loin d'être satisfaisante. Dussaulx
n'ignorait pas que les joueurs de pro%
fession n'ont pas de passions, n'ont
pas même de caractère, et que cette
classe est trop peu nombreuse, trop
peu importante, trop peu susceptible
d'impressions morales, pour qu'on
doive prendre la peine de faire pour
elle un gros livre.

Dussaulx, revenant au caractère de
\folio{11}
fourberie qu'il attribue injustement
aux joueurs en général, dit : « Vous
trouverez des joueurs suspects dans
tous les rangs ; parmi les gens de
lettres vous ne verrez que des vic%
times résignées aux caprices du
sort. »

J'observe d'abord que les joueurs
les plus nombreux, ceux du moins
qu'il faut le plus s'attacher à guérir
de leur frénésie, sont ceux qui jouent
aux jeux de hasard : or, d'après les
précautions prises ordinairement par
les banques, un joueur assez adroit
pour être fructueusement un fripon,
est une exception très-rare ; et encore
une fois ce n'est pas pour ceux qui
vont au jeu, avec l'intention d'y vo%
ler, qu'on fait des traités de morale :
ensuite si, par ces mots \emph{victimes ré%
\folio{12}
signées}, Dussaulx a entendu incapa%
bles de fourberie, je crois que d'autres
rangs ont également cet avantage ;
et s'il a entendu, disposées à souffrir
la perte avec patience, j'ai remarqué
que cette résignation se trouvait plus
chez les sots que chez les gens d'es%
prit, dont l'imagination est plus facile
à s'exalter, quoique la réflexion et la
philosophie les modèrent ensuite.

Dussaulx a trop confondu les rap%
ports sous lesquels le jeu peut être
considéré : il devait sans doute pré%
senter séparément l'influence qu'il a
sur les mœurs et sur la fortune pu%
blique, et celle qu'il a sur les mœurs
et sur la fortune des particuliers ;
mais il revient trop fréquemment aux
mêmes idées ; ce qu'on peut attribuer
au défaut d'ensemble de son ouvrage.
\folio{13}
Il me parait au moins que les parties
en sont trop détachées ; que ses ta%
bleaux ne sont pas liés de manière à
soutenir l'intérêt ; que ses raisonne%
mens ne sont pas assez suivis pour
porter dans les esprits cette convic%
tion dont plus de conversions et de
réformes auraient été les résultats.
Sa marche est quelquefois embarras%
sée : on voit qu'il veut dire tout ce
qu'il sait, tout ce qu'il a lu sur les
jeux ; et on peut lui appliquer ce qu'il
a dit de Barbeyrac : « II s'est jeté trop
souvent dans des discussions super%
flues ou étrangères à son sujet ; »
de manière qu'on jugerait difficile%
ment s'il a écrit pour les joueurs ou
pour ceux qui ne le sont pas.

La plupart de ces défauts, cet em%
barras, ces contradictions ne s'ex%
\folio{14}
pliquent-ils pas d'abord par l'incon%
venance de présenter à-la-fois, dans
un ouvrage moral, ce qui devait ne
contenter que la curiosité, avec ce
qui devait servir à l'instruction ? En
outre, l'espèce de nécessité dans la%
quelle l'auteur s'était mis de rendre
odieux et méprisables les joueurs et
les maisons de jeu, l'obligeait de s'é%
carter de tems en tems de la vérité, à
laquelle la franchise de son caractère
le ramenait bientôt.

Quant aux remèdes, Dussaulx,
après avoir cité une foule de traits
qui semblent en montrer l'ineffica%
cité ; après avoir dit qu'on n'a rien à
attendre des Gouvernemens, « tou%
si pauvres, qu'on ne saurait 
compter sur eux, lorsqu'il s'agit
d'argent,» ni sur l'experience, qui
\folio{15}
appartient plus à ceux qui méditent
qu'à ceux qui agissent, et qui est tou%
jours impuissante contre le désir et la
séduction, indique cependant quel%
ques moyens de réforme, tels que les
amusemens naturels, la suspension
de la pratique du jeu par un violent
effort sur soi-même, l'exercice de la
bienfaisance, des entreprises labo%
rieuses, le recours dans le sein d'une
sage et prudente amitié. Voilà ce qu'il
conseille aux joueurs eux-mêmes.
Sans doute ces moyens sont salutaires,
et leur indication seule prouve la
bonté, la candeur d'ame de l'auteur,
dans l'ouvrage de qui je ne relève qu'à
regret, parmi des beautés et des vé%
rités du premier ordre, ce que je
regarde comme des défauts ou des
erreurs ; mais plusieurs de ces moyens
\folio{16}
ne sont-ils pas hors du pouvoir des
uns ? Ne sont-ils pas insuffisans pour
les autres ? Et qui donnera aux joueurs
la force d'exécution sans laquelle la
force de volonté n'est rien ? Une
grande passion se guérit-elle avec des
calmans ? Et si ce joueur croit voir
dans ce qu'il va faire un avantage sûr
et prochain, l'en détournerez-vous
en lui offrant, dans ce que vous lui
proposez, un avantage douteux et éloigné ?

Dussaulx expose encore quelques
autres moyens de réforme ; mais
ceux-ci il les fait dépendre de l'auto%
rité. Ce sont les refus d'honneurs et
d'emplois aux joueurs incorrigibles,
l'obligation aux joueurs fortunés de
nourrir des vieillards, des pauvres,
des infirmes \paren{je n'ai pas besoin de
\folio{17}
dire quels maux pourraient résulter
de l'erreur ou de la mauvaise foi de
rapports faits à l'autorité sur la con%
duite des particuliers qui d'ailleurs
auraient tant de moyens de dérober
à ses agens la connaissance de leurs
gains ou de leurs excès}, la suppres%
sion des jeux d'état, l'abolition des
privilèges de jeux, la réformation des
mœurs, l'éducation.

Les lecteurs ayant de l'expérience
et de l'instruction, apprécieront, je
ne dirai pas les remèdes découverts,
mais les vœux formés par Dussaulx :
j'y reviendrai en m'occupant aussi
des moyens praticables de réforme
ou d'amélioration.

J'ai rempli un devoir pénible en 
manifestant mon opinion sur l'insuf%
fisance de l'ouvrage de Dussaulx con%
\folio{18}
tre la passion et les excès du jeu. Je
le répète, je n'en reconnais pas moins
le mérite supérieur de cet ouvrage,
et j'en recommande vivement la lec%
ture aux personnes malheureuses ou
trompées que le jeu entraîne ou sé%
duit. Je n'ai point l'orgueilleuse pré%
tention de le refaire ; mais en traitant
beaucoup moins d'objets, et me ren%
fermant dans un cadre étroit, je veux
rechercher s'il n'y aurait pas de nou%
velles lumières à répandre sur ce
sujet. D'ailleurs, l'état des jeux n'est
pas aujourd'hui ce qu'il était lorsque
Dussaulx a écrit.

\chapter[Du Jeu]{DU JEU}

\folio{19}
\lettrine{L}{e} jeu, fruit de l'amour et du plaisir,
et aussi variable, ne fut d'abord qu'un
exercice agréable ou salutaire de l'es%
prit ou du corps ; il n'est pas d'autre
chose pour beaucoup de personnes.

Si l'on fait attention à la manière
dont se développent les facultés intel%
lectuelles de l'homme ou de tout être
vivant, on verra que, presque dès sa
naissance, il joue avec des objects pu%
rement physiques ou avec des êtres
animés : s'il joue avec des objets pure%
ment physiques, il ne tarde pas à se
lasser de celui qui l'occupe ; il le quitte
pour en prendre un autre qu'il va quit%
\folio{20}
ter à son tour : s'il joue avec des êtres
animés, sur-tout ceux de même nature
que la sienne, son action est plus vive,
sa gaîté plus bruyante, son attache%
ment plus prolongé. Ces mouvemens,
d'abord vagues et irréguliers, lorsque
l'intelligence se forme, et que la joie est
partagée, acquièrent insensiblement
de la règle et de la mesure. L'attrait du
jeu n'est encore que l'attrait du plaisir.
Le talent du joueur est l'habileté, la
ruse, l'adresse ou l'industrie. Le jeu
consiste à faire des sons, courir, s'éle%
ver, atteindre un but, prévenir ou re%
pousser une attaque, saisir prompte%
ment un objet idéal ou matériel ; enfin,
il se compose suivant le goût de celui
qui s'y livre, et offre presque toujours
une difficulté à vaincre. Un prix est
donné à celui qui l'a vaincue ; c'est une
\folio{21}
fleur, un fruit, un sourire, un baiser :
ce prix tente celui qui ne l'a point ob%
tenu ; celui qui en a remporté un pre%
mier veut en remporter un second ;
l'amour-propre est piqué ; l'émulation
naît et est excitée ; les défis se propo%
sent ; on n'aspire plus après des baga%
telles ; la nature du prix a changé ; celle
du jeu n'a plus la même simplicité : elle
se varie, elle se complique ; les inéga%
lités de force ou de talent, le doute,
les diverses interprétations font naître
les disputes ; l'adresse, l'industrie ins%
pirent du découragement ou de la dé%
fiance ; on leur associe une puissance
aveugle, le sort, qui agit tantôt avec
elles, tantôt sans elles : l'ignorant
s'étonne de son savoir ; le faible de sa
force ; l'infortuné de ses ressources ; le
téméraire de son triomphe. Le joueur,
\folio{22}
dans sa joie, croit que le sort a des
yeux, puisqu'il le favorise. Bientôt ce
tyran, interrogé de toutes parts, ras%
semble autour de lui la foule de ses
favoris, même celle de ses victimes.
Ses arrêts sont prompts, ses faveurs
faciles. L'ennui, la paresse, l'ambition
assiégent ses portes ; et les plus aima%
ble enchanteresses, l'espérance et
l'imagination, sont là, qui rassurent
les timides, flattent les orgueilleux,
consolent les mécontens et ramènent 
les fugitifs.

Déjà l'ardeur du jeu, celle des pas%
sions cupides que la plupart des hom%
mes éprouvent la première, fait naître
ou met les autres en mouvement ; et
le monde habité est infecté d'un vice
d'autant plus funeste, d'autant plus
contagieux, qu'il s'embellit toujours
\folio{23}
du nom, de l'éclat et de la séduction
du plaisir.

Tels me paraissent être les commen%
cemens, les progrès, les variations de
ce qu'on appelle le jeu. Est-il donc né%
cessaire de fouiller dans les annales de
l'antiquité, pour découvrir son ori%
gine et étudier son histoire ? Si nous
consultons le livre de la nature, qui
nous est toujours ouvert, nous ne dou%
tons pas que les passions de l'homme
n'aient eu, ainsi que sa figure, dans
tous les lieux et dans tous les tems, à-%
peu-près les mêmes traits, le même
caractère. Les différens climats, les
lois, les mœurs, les usages, mettent
peu de différence dans leur dévelop%
pement et dans les excès auxquels
elles conduisent. C'est un fleuve ra%
pide dont on peut prévenir l'entière
\folio{24}
corruption, mais qu'il est aussi diffi%
cile d'épurer que d'en arrêter le cours.

Le jeu est pur dans sa source. Mais
de quoi n'abuse-t-on pas ! Les excès
ont lieu jusque dans le travail.

\frquote{Si la fureur du jeu, dit Dussaulx,
est universelle en France, c'est parce
qu'une corruption générale est im%
punie ; c'est parce que l'amour des
richesses l'emporte sur l'honneur,
à mesure que les empires vieillis%
sent \footnote{
  Grande vérité par laquelle s'expliquent
  les désordres dont nous avons tant à gémir.
}.

Le mal existe sans qu'on puisse
en accuser personne.}

Ce que j'ai dit du développement de
ce goût naturel qui nous porte vers le
jeu, a son application chez les peuples
sauvages comme chez les peuples civi%
\folio{25}
lisés. Le sauvage, en se mettant à la
merci du sort par des règles précises
et déterminées, en même tems qu'il 
prouve son ignorance, prouve qu'il a
fait un pas de plus vers la civilisation ;
et il est aisé de remarquer ici que la
passion du jeu réunit la sagacité à
l'aveuglement.

Presque partout le jeu a été la re%
présentation des combats. Les hom%
mes, naturellement imitateurs, et en%
clins à engager des luttes, se dédom%
magent, dans cette autre guerre, des
langueurs d'un honteux repos. On
peut régler leurs mouvemens ; mais
arrêtez-les dans leurs courses !

Ce besoin de jouer qui se manifeste
dès l'enfance, et fait contracter de
douces et fatales habitudes, a dans
la société bien d'autres effets que ceux
\folio{26}
qu'on leur attribue. Tous les jeux ne
sont pas ceux qui se pratiquent dans
les académies, dans les maisons de jeu ;
tous les joueurs ne sont pas désignés
par ce nom : il s'en trouve ailleurs, en
plus grand nombre, qui confient de
même au sort leurs plus grands inté%
rêts, et dont les calculs sont aussi faux,
les combinaisons aussi absurdes et les
espérances aussi chimériques. Une
ruine totale, la perte même de la vie,
est le résultat fréquent de ces autres
jeux : je veux parler de ce que, dans
les différens états de la vie sociale,
des hommes, égarés par leurs vœux
ou leurs désirs, exposent ou sacrifient
sans prudence, sans nécessité, ou sans
motifs raisonnables, dans la poursuite
des faveurs de la gloire, de l'amour et
de la fortune.

\folio{27}
Dans ces jeux, comme dans les pre%
miers, on est justifié par le succès ; et
l'opinion, toujours complice des vices
heureux, attribue à des calculs plus
médités, à une conduite plus sage, ce
qui n'est que l'effet d'un hasard favo%
rable ou d'une coupable audace, tan%
dis qu'on condamne et flétrit celui que
plus d'ordre et de modération n'a pas
garanti des revers du sort.



\chapter[Des joueurs]{Des Joueurs}

\folio{28}
\lettrine{L}{es} joueurs n'ont pas un caractère
unique, déterminé, susceptible d'être
traité avec les mêmes procédés, ou
combattu avec les mêmes armes. Leur
caractère a des nuances extrêmement
variées : en cela il diffèrent des ava%
res, des envieux, des jaloux, des ivro%
gnes, des débauchés, dont la passion
ou le vice a un principe connu, ou
commun à presque tous.

L'ambition, l'orgueil, la cupidité,
l'ennui, le besoin, font des joueurs de
différents espèces.

On joue par caprice ou par système,
\folio{29}
par occasion ou par habitude, aux jeux
de hasard ou de commerce.

Les différents manières de jouer
tiennent à la différence des motifs, de
l'esprit, du tempérament et de la po%
sition des joueurs.

A voir de l'audace et le sang-froid des
uns, la timidité et la turbulence des
autres : on juge aisément si les mêmes
leçons ou les mêmes mesures de ré%
pression leur conviennent.

Tel n'a joué que quelques jours, et
a joué un jeu considérable ; tel autre
ne peut se priver du jeu un seul jour,
qui ne joue qu'un jeu modéré. A qui le
nom de joueur convient-il davantage ?

Il me semble du moins qu'il n'est
pas juste de comprendre sous la même
dénomination le goût et la passion, le
caprice et l'habitude du jeu.

\folio{30}
Je me demande si ce sont ceux qui ai%
ment le jeu, ou ceux qui ne l'aiment
pas, qui dans la société ont exception ?

Le nombre des joueurs honteux est
plus considérable qu'on ne le croit.

Je rencontre un homme de ma con%
naissance peu favorisé de la fortune.
On parle des jeux de hasard : « C'est
une fureur, dit-il, et on ne songe
pas à y mettre un frein !» Le soir,
je le trouve dans une maison particu%
lière. Il jouait à la bouillotte ; la cave
était de cinq louis.

Il y a de la différence entre le ca%
ractère et les procédés des joueurs aux
jeux de commerce, et ceux des joueurs
aux jeux de hasard.

Il convient aussi de distinguer les 
joueurs d'habitude et les joueurs de
profession. Si on ne sait poser une
\folio{31}
ligne de séparation entre les diffé%
rentes espèces de joueurs, on est ex%
posé à commettre des erreurs et des
injustices.

Des joueurs aux jeux de commerce
peuvent tirer parti de leur expérience,
de leur savoir, de leur finesse ; d'au%
tres sont trop légers, trop distraits,
ou d'une ignorance trop présomp%
tueuse pour ne pas donner à ceux-ci
beaucoup d'avantages. Cependant,
comme le hasard y une part plus ou
moins grande, les plus habiles y sont
quelquefois maltraités ; mais ils ne
tardent pas à reprendre leur supério%
rité. Voilà pourquoi on cite des joueurs
presque constamment heureux ; et
ceux qui donnent au goût justifié de
ces joueurs un aliment facile, n'osent
avouer et ne s'avouent pas eux-mê%
\folio{32}
mes leur ignorance ou leur faiblesse.
Mais imaginez qu'une perfide adresse,
qu'une coupable industrie vienne en%
core seconder l'art et l'expérience,
vous connaîtrez mieux les motifs pour
lesquels certains hommes font du jeu
leur unique occupation. Aussi ce n'est
pas aux jeux de hasard que si livrent
ces joueurs, dont l'honnête Dussaulx,
a fait, avec raison, un épouvantail ;
car je ne parle point encore de ces
banquiers de société, à qui l'art de
mettre en défaut les regards les plus
attentifs, réussit d'autant plus, que
là on s'en défie le moins : là on crain%
drait, par une accusation directe, de
paraître impudent ou grossier ; et il se
trouverait difficilement quelqu'un qui
oserait vérifier et constater le délit.

On conçoit que les joueurs de pro%
\folio{33}
fession ont pour la plupart les doigts
agiles, la tête froide, et cette absence
de passions favorable aux calculs. Mal%
heureusement ils ne prennent pas le
titre de joueurs, et ils souffriraient
impatiemment qu'on le leur donnât.
Ils sont toujours surchargés d'autres
affaires, d'autres soins, et le jeu est
le moindre sujet de leur conversation.
De qui les favorise surtout, c'est qu'il
est aisé de les confondre avec ceux qui
ne sont que des joueurs d'habitude.
Ceux-ci ont besoin de jouer, comme le
besoin de manger et de boire : l'heure
du jeu est marquée pour eux comme
celle de leur repas, de leur sommeil,
de leur dévotion. Il leur arrive sou%
vent de bâiller, de s'assoupir au jeu.
Ils jouent à un jeu de hasard comme
à tout autre, et ne sont étonnés que
\folio{34}
du coup qui les ruine sans ressource ;
c'est leur maison qui vient de s'é%
crouler. S'ils survivent à cet accident,
ils passeront le reste de leur vie à ra%
conter ce qu'il avait d'extraordinaire.

Les grands joueurs, j'entends ceux
qui ont la fureur du jeu, sont en gé%
néral des hommes à caractère et à
grandes passions, c'est-à-dire, qu'ils
ont le sang vif, la tête sulfureuse,
l'âme brulante, l'imagination exaltée,
la sensibilité profonde, et en cela, je
ne diffère pas autant qu'on le croirait
d'opinion avec Dussaulx, qui, tout
en refusant cette qualité aux joueurs,
prouve, par beaucoup de traits sail%
lans sur leur compte, qu'ils la pos%
sèdent à un degré éminent. Il est vrai
qu'un sort très-heureux ou très-mal%
heureux paraît rendre, même rend
\folio{35}
quelquefois les hommes insensibles ;
mais l'état d'enivrement ou d'apathie
ne dure pas long-tems, et la nature
ne tarde pas à reprendre son cours et
son énergie.

C'est de la classe de ces joueurs dont
je viens de parler, que sort la plus
grande partie de ceux que la ruine et
le désespoir portent au suicide.

Parmi les hommes de mérite que
présente Dussaulx comme ayant été
de grands joueurs, j'ai déjà nommé
Caton, Henri~IV, Montaigne, Des%
cartes, Collardeau, et lui-même, qui
en a fait l'aveu : je dois ajouter les
noms célèbres de Duguesclin, le Gui%
de, Rotrou, Voiture, Cardan, Halli%
fax, Schafsterbury, même du méde%
cin allemant Ponchasius Justus, aussi
auteur d'un livre contre le jeu.

\folio{36}
Qu'on ne s'y méprenne pas, quand
je dis que les grands joueurs sont des
hommes à passions, je ne dis pas que
les hommes à passions éprouvent né%
cessairement celle du jeu. Parmi ces
être peu communs, il en est qui ont
parcouru le cercle entier des passions ;
il en est dont une seule a consommé
la vie entière.

Les états sui laissent le plus de loisir,
sont ceux qui fournissent le plus grand
nombre de joueurs. C'est parmi les
ecclésiastiques que j'ai pris le goût du
jeu ; c'est parmi les militaires que je
l'ai vu régner avec le plus d'éclat. En
jouant aux jeux de hasard, ils ne sor%
tent pour ainsi dire, ni de leur profes%
sion, ni de leurs habitudes.

Mais quel long chapter il y aurait
à faire sur les joueurs ! …




\end{document}
