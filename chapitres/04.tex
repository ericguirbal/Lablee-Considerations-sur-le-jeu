\chapter[De la Fureur du Jeu]{De la fureur du jeu}

\folio{37}
\lettrine{L}{a} fureur du jeu a des symptômes
effrayans : ils se manifestent, soit lors%
que, semblable à une épidémie elle
a gagné la majeure partie des habi%
tans d'un pays, dont les jeux de ha%
sard sont devenus la principale oc%
cupation ; soit lorsqu'on fait à l'envi
des mises de jeu considérables, soit
lorsqu'un joueur, irrité de ses pertes,
ne suivant plus de règle, et obéissant
à une aveugle impulsion, s'expose à
perdre en peu de tems tout ce qu'il
possède ; soit enfin lorsqu'ayant per%
du son argent, on joue ses effets ou
autres choses, qui, par leur nature,
\folio{38}
semblent ne devoir par être mises à la
disposition du sort.

Voilà les abus, les excès du jeu ;
ceux contre lesquels la raison, l'hu%
manité invoquent des précautions et
des mesures, mais qu'il semble qu'au%
cun pouvoir ne saurait atteindre.

Quoique j'aie déjà mis le lecteur
en état d'examiner les déplorables
effets de cette fureur, je vais tâcher
de les rendre plus sensibles par des
exemples, en jetant, avec Dussaulx
et d'autres écrivains philosophes, un
coup-d'œil rapide sur ce qui a signalé
la passion du jeu en différens lieux
et en différens tems.

Chez les Gentous, le plus ancien
des peuples connus, le jeu avait causé
un tel désordre, qu'il fut nécessaire
de faire des lois pour en réprimer
\folio{39}
les excès. Un magistrat était payé
pour surveiller les rendez-vous de
jeu ; il avertissait des fautes, et fai%
sait couper les doigts aux prévarica%
teurs.

Les Romains, même dans l'état
républicain, qui suppose des vertus
plus pures, ont été des joueurs déter%
minés. Ovide, en parlant des joueurs
qu'il avait vus en action, dit : \frquote{On
sèche de désir, on frémit de colère,
on se meurt de rage. Que d'injures !
Quels cris ! Les malheureux ! ils
invoquent les dieux !}

On lit dans Juvénal : \frquote{On ne se
contente pas de porter sa bourse
au lieu de la séance, on y traîne
son coffre-fort.

On perd cent mille sesterces, et
on ne peut vêtir un esclave !

\folio{40}
Tous, jusqu'à la populace, sont
en proie à la fureur du jeu.}

On lit dans Tacite : \frquote{Quand les
Germains s'étaient ruinés au jeu,
ils se jouaient eux-mêmes. Le
vaincu, quoique plus jeune et plus
fort, se laissait garotter et vendre.}

Saint-Ambroise nous apprend que
chez les Huns, peuple grossier, mais
fidèle à sa parole, celui qui jouait sa
vie et perdait, se tuait quelquefois,
malgré son vainqueur.

Les nègres de Juida, les Chinois,
les Vénitiens jouaient leurs femmes
et leurs enfans.

Les Indiens jouaient jusqu'aux
doigts de leurs mains, et s'ils les per%
daient, ils se les coupaient eux-mêmes.

En Russie, on joue ses esclaves. Il
n'est pas rare de voir, soit à Moscow,
\folio{41}
soit à Pétersbourg, de pauvres fa%
milles appartenir successivement à
dix maîtres en un jour.

A Naples, et dans divers endroits
d'Italie, les bateliers jouent leur li%
berté pour un certain nombre d'an%
nées.

Aucun peuple n'a porté plus loin
la manie du jeu que les Anglais. Chez
eux, c'est presque l'esprit national.
Leurs factions, leurs affaires, leur
commerce, ils ont tout mis en jeu,
tout soumis au calcul. Navigateurs
insatiables, dit Dussaulx, ils se sont
familiarisés avec les dangers et le ha%
sard. Excepté quelques philosophes
et quelques-unes de ces ames que la
contagion ne saurait infecter, le reste
n'a étudié ses devoirs que sur des
tables de probabilités, dressées pour
\folio{42}
apprendre à faire des fortunes ra%
pides.

Mais, depuis le commencement de
notre monarchie, nous n'avons guè%
res, sur cet article, montré plus de
raison et de sagesse.

On voit dans nos annales, que ces
seigneurs hautains et fainéans qui ne
savaient guères que tourmenter leurs
vassaux, boire et se battre, étaient
pour la plupart des joueurs effrénés ;
qu'ils bravaient impunément la dé%
cence et les lois. Le frère de Saint-%
Louis jouait aux dés malgré les dé%
fenses réitérées de ce prince vertueux.
Duguesclin lui-même perdit, dans sa
prison, tout ce qu'il possédait ; le
duc de Touraine, frère de Charles~VI,
\emph{se mettait volontiers en peine,} dit
Froissard, \emph{pour gagner l'argent du
\folio{43}
Roi.} Transporté de lui avoir gagné
cinq mille livres, son premier cri fut :
\emph{Monseigneur, faites-moi payer.}

On jouait dans les camps et en
présence de l'ennemi. Des généraux,
après avoir ruiné leurs propres af%
faires, ont compromis le salut de la
patrie.

Philibert de Châlons, prince d'O%
range, commandant au siède de
Florence pour l'empereur Charles-%
Quint, perdit de l'argent qui lui avait
été compté pour la paye des soldats,
et fut contraint, après onze mois de
travaux, de capituler avec ceux qu'il
aurait pu forcer.

On parle dans le manuscrit d'Eus%
tache Deschamps, d'un hôtel de
Nesle, fameux par de sanglantes ca%
tastrophes : des acteurs y ont perdu,
\folio{44}
les uns la vie, les autres l'honneur.

Sous Henri II, dit Brantôme, un
capitaine français, nommé la Roue,
jouait cinq à six mille écus d'un
coup ; ce qui alors était exorbitant.
Il proposa de jouer vingt mille écus
contre l'une des galères de Jean-%
André Doria.

Un fils naturel du duc de Belle%
garde fut en état de lui compter, sur
ses gains, cinquante mille écus pour
s'en faire reconnaître juridiquement.
Il est vrai que la plus forte partie de
cette somme avait été gagnée en An%
gleterre.

Le peuple s'essayait déjà. Des fri%
pons s'étant concertés avec des Ita%
liens qu'ils avaient appelés à leur aide,
gagnèrent trente mille écus à Hen%
ri~III, \emph{qui avait}, dit un journaliste,
\folio{45}
\emph{dressé en son Louvre un déduit de
cartes et de dés.}

C'est surtout sous notre bon Henri~
IV, qui, jeune encore et peu fortu%
né, empruntait, pour jouer, de l'ar%
gent à tous ceux qu'il croyait de ses
amis, que la fureur du jeu a éclaté.
Qu'on en juge par quelques traits. En
une année, Bassompierre gagna cinq
cent mille livres, Pimentel deux cent
mille écus ; et le duc de Biron perdit
seul plus de cinq cent mille écus.
Qu'on considère le prix que l'argent
avait alors.

Henri IV, dit Péréfixe, n'était pas
beau joueur, mais âpre au gain, ti%
mide dans les grands coups, et de
mauvaise humeur dans la perte.

Comme les joueurs vulgaires, il
jouait tantôt avec audace, tantôt avec
\folio{46}
faiblesse. Le duc de Savoye jouant
avec lui et sachant qu'il aimait à ga%
gner, dissimula sont jeu, et, par po%
litique, renonça volontairement à
quatre mille pistoles.

On ne l'abandonnait pas impuné%
ment lorsqu'il perdait.

L'amour même ne pouvait le dis%
traire de sa malheureuse habitude.
On lui annonce qu'une princesse
qu'il aiamait va lui être ravie : \frquote{Prends
garde à mon argent, dit-il à Bas%
sompierre, et entretiens le jeu pen%
dant que je vais savoir des nou%
velles plus particulières.}

Lorsque, sous son règne, la Na%
tion, long-tems agitée par la guerre
civile, put enfin se reposer au sein de
la paix, presque toutes les professions
éprouvèrent la fureur du jeu. Des
\folio{47}
magistrats vendaient la permission
de jouer. Les joueurs avaient à la
cour un grand crédit, et jouissaient
de privilèges particuliers.

Paris se remplissait de joueurs : il
s'y forma, pour la première fois, des
académies de jeu, où \emph{la bourgeoisie,
les artisans et le peuple se précipi%
taient en foule.} Tous les jours il y
avait quelqu'un de ruiné.

On rapporte que Louis~XIII, celui
de nos Rois qui a le plus sévi contre
le jeu, aimait tant les échecs, que
pour qu'il y eût pas de temps perdu,
et qu'il pût y jouer en voiture, on
fit pour lui ce qu'on avait fait pour
l'empereur Claude : on plaça sans sa
voiture un échiquier bourré, sur le%
quel s'adaptaient les pièces montées
sur des aiguilles.

\folio{48}
Mazarin, dit l'abbé de St.-Pierre, in%
troduisit le jeu à la Cour de Louis~XIV
en 1648. Il engagea le roi et le reine
régente à jouer, et l'on préféra les jeux
de hasard. Le jeu passa de la Cour à la
ville, et de la capitale dans toutes les
petites villes de province.

Dès-lors on ne vit que des joueurs
d'un bout de la France à l'autre ; ils se
multipliaient rapidement dans toutes
les professions et même dans la robe,
qui se piquait encore d'une certaine
décence.

Le cardinal de Retz rapporte dans
ses Mémoires, qu'en 1650, le magis%
trat le plus âgé du parlement de Bor%
deaux, et qui passait pour être le plus
sage, ne rougissait pas de risquer tout
son bien dans une soirée, et cela,
ajoute-t-il, sans que sa réputation
\folio{49}
en souffrît, tant cette fureur était
générale !

Les États n'offraient plus, lors%
qu'ils étaient convoqués, que des as%
semblées de joueurs.

\frquote{J'ai vu, dit madame de Sévigné,
mille louis répandus sur le tapis ; il
n'y avait plus d'autres jetons ; les
poules étaient au moins de cinq, six
ou sept cents louis, jusqu'à mille,
douze cents…. On joue des jeux im%
menses à Versailles…. Le \emph{hoca} est
défendu à Paris, \emph{sous peine de la
vie, et on le joue chez le roi.} Cinq
mille pistoles avant le dîner, ce n'est
rien. C'est un vrai coupe-gorge !}

Dans les soupers clandestins et dans
les maisons de campagne du surinten%
dent Fouquet, vingt joueurs qualifiés,
tels que les maréchaux de Richelieu,
\folio{50}
de Clairembaut, etc., se rassemblaient
avec un peu de mauvaise compagnie
pour y jouer des terres, des maisons,
des bijoux, et jusqu'à des points de
Venise, jusqu'à des rabats ; on s'y avi%
lissait au point de circonvenir quel%
ques dupes opulentes, toujours in%
vitées les premières.

Les trois quarts de la nation ne sou%
pirèrent plus qu'après le jeu, qui, lui-%
même, devint un objet de spéculation
pour le gouvernement.

On connaît le fameux jeu auquel
Law, joueur étranger, devenu contrô%
leur général, entreprit de faire jouer la
nation, sous la minorité de Louis~XV.
On sait comme il séduisit ceux même
qui s'étaient garantis de l'épidémie
des jeux de hasard.

Vers le même tems, des ministres
\folio{51}
et des magistrats permirent des jeux
publics, parmi lesquels on distingue
ceux des hôtels de Gesvres et de Sois%
sons, où l'on a tant fait de victimes.

C'est là que le jeu de la roulette a
paru pour la première fois en France,
les joueurs de bonne foi ayant enfin
voulu jouer à un jeu où ils pouvaient
hasarder leur argent en toute sûreté.
En effet, il n'y a pas de jeu où les
chances soient plus égales \paren{je ne
parle pas de l'avantage du banquier}.

\frquote{Nous avons encore, dit Dussaulx,
qui m'a fourni presque tout ce que je
viens d'exposer, indépendamment de
cent maisons connues où l'on se ruine
tous les jours, dix fois plus de réduits
subalternes que l'on n'en comptait
sous Henri~IV, sous Louis~XIV et du 
tems de la régence.}

\folio{52}
On ne rougit plus, à l'exemple de
Caligula, de jouer au retour des funé%
railles de ses parens ou de ses amis.

La plupart de ceux qui vont aux
eaux sous prétexte de santé, n'y cher%
chent que des joueurs.

Aux États, c'est moins l'intérêt du
peuple qui rassemble une partie de la
noblesse, que l'attrait d'un jeu ter%
rible.

Tout est en feu au moment où j'écris,
ajoute Dussaulx ; sans parler des bas%
sesses, depuis deux jours je compte
quatre suicides et un grand crime.

Louis~XVI n'aimait pas le jeu. On
profitait de son absence pour se livrer
à la Cour aux jeux de hasard : on y
jouait sur-tout le pharaon et le biribi.
Il aurait été extraordinaire qu'au
milieu de la corruption des mœurs,
\folio{53}
corruption que ce roi vertueux n'a
jamais encouragée par son exemple,
l'épidémie du jeu ne se fût pas fait
sentir. On jouait gros chez des fi%
nanciers, des grands seigneurs, des am%
bassadeurs étrangers, et chez quel%
ques princes. La police autorisait ces
jeux, moyennant dix louis par maison.je

A l'hôtel d'Angleterre, tripot des
plus fréquentés, les jeux de commerce
étaient encore plus dangereux que les
jeux de hasard. De ceux-ci, on n'y a
guère joué que \emph{la belle}.

Madame de Polignac rassemblait
chez elle les personnes de distinction
les plus atteintes de la fureur du jeu.

On faisait des mises considérables
au trente-un, qui se jouait chez la res%
pectable et trop malheureuse prin%
cesse de Lamballe ; elle avait la fai%
\folio{54} 
blesse d'y faire une martingale de cent
louis.

On jouait aussi le plus gros jeu chez
la maréchale de Luxembourg, le duc
de la Trimoille, etc.

Il y avait un certain nombre d'hom%
mes habiles à diriger les jeux qui, au
premier mot, se rendaient dans les
maisons où ils étaient demandés, avec
les ustensiles de jeu et les fonds né%
cessaires. A la Cour, c'était toujours
les mêmes : on les appelait \emph{banquiers
de la société suivant la Cour.}

Dans quelques maisons de jeux les
mieux famées, on jouait le biribi, le
pharaon, le passe-dix, le pair et l'im%
pair, et le creps.

Il ne faut pas confondre ces mai%
sons avec les obscurs tripots, domaine
particulier de quelques inspecteurs.

\folio{55}
Le lieutenant de police appliquait
la rétribution de ces maisons de jeu
au soulagement de familles pauvres,
mais honnêtes ; et au soutien d'éta%
blissemens de bienfaisance, tels que
l'hospice des chevaliers de St.-Louis,
à la barrière d'Enfer.

Mais le centre des excès du jeu se
trouvait chez le dernier duc d'Orléans,
tantôt à son palais, tantôt à sa jolie
retraite de Mousseaux. Les étrangers
de distinction, de grands seigneurs,
des financiers, des commerçans, en
un mot, tous les gens à argent, étaient
recrutés pour ces parties, où l'on as%
sure qu'à l'insu du prince ne ré%
gnaient pas une exacte probité et une
extrëme délicatesse. Des Anglais y
étaient le plus remarqués. Les pertes
considérables qui s'y faisaient la nuit,
\folio{56}
étaient le jour le sujet des conversa%
tions.

Ces parties n'avaient plus lieu, lors%
qu'on établit, au Palais-Royal, une
maison de jeu de hasard autorisée par
la police. Là, de nombreuses victimes
ont encore été dépouillées.

Jusqu'au 10 août 1792, la plus bril%
lante et la plus forte partie des jeux de
hasard était à l'hôtel Massiac, place
des Victoires. Là se rendaient les mem%
bres les plus marquans de l'assemblée
constituante, et des personnes con%
nues par leur opulence.

Vers ce tems, la police municipale
autorisait des jeux au cirque du Pa%
lais-Royal, dans quelques autres par%
ties de ce palais, et dans divers quar%
tiers de Paris.

Le jeu n'avait jamais causé plus de
\folio{57}
ravages dans Paris que dans les deux
années qui ont précédé le 10 août ;
mais avant le 9 thermidor, c'est-à-%
dire, sous le règne affreux de la ter%
reur, sa fureur a paru céder à des
fureurs plus sanglantes.

Dès l'hiver de l'an 3, il a repris une
activité effrayante.

Je borne à cette époque le tableau
des principaux désordres attribués à
la passion du jeu, laissant à d'autres
le soin de porter dans les esprits une
terreur salutaire, par le recit de sui%
cides et d'autres accidens, résultats
inévitables de ces excès. Je ne pour%
rais signaler les traits qui, depuis ce
tems ont caractérisé la fureur des
jeux de hasard, sans accuser ou affli%
ger des personnes encore vivantes,
ce qui est loin de mon esprit et de
mes intentions.
