\chapter{De la Conduite du Jeu}

\folio{93}
\lettrine{C}{'est} se mal conduire que de jouer
aux jeux de hasard, et le meilleur
conseil que puisse donner un homme
raisonnable à ceux qui s'y livrent,
est d'y renoncer sans délai. S'ils
n'ont pas ce courage, la conduite
qu'ils doivent tenir consiste à tirer
le meilleur parti possible de leur po%
sition, c'est-à-dire à éviter ce qui
peut leur nuire davantage, en n'ou%
bliant pas néanmoins que dans tou%
tes les circonstances de la vie, l'équité
doit être la première règle de con%
duite.

Il y a une grande imprudence à
prendre pour son jeu sur son néces%
\folio{94}
saire ; car, loin de se mettre ainsi en
état de réparer sa mauvaise fortune,
on adopte précisément le moyen le
plus sûr de l'accroître et de la porter
à son comble.

Si la raison permettait de croire
aux pressentimens, je dirais qu'il
semble que la crainte des joueurs
qui exposent au jeu leur nécessaire,
crainte qui les fait jouer de la ma%
nière la plus désordonnée, se vérifie
presque toujours. Dans la vie ordi%
naire, ce qu'on appelle malheur, est
le plus souvent l'effet du défaut de
conduite : cela est encore plus vrai
au jeu. La moindre perte est un grand
malheur pour les personnes qui
jouent ce que réclament leurs be%
soins ou ceux de leur famille, ou
l'acquit de leurs engagemens ; aussi
\folio{95}
les entendez-vous se plaindre dou%
loureusement à chaque coup qui leur
est contraire, tandis que l'homme
riche, ou celui qui n'a pris ce qu'il
met au jeu que sur son superflu,
perd avec tranquillité, ignore même
quelquefois s'il perd ou s'il gagne.

Le petit jeu et le gros jeu ne sont
tels que relativement aux facultés
des joueurs. Celui qui porte au jeu
une forte somme, quand il n'a d'au%
tre intention que celle de jouer quel%
ques pièces, sans augmenter ses
mises pour courir après celles qu'il
aura perdues, commet aussi une
grande imprudence : il faut qu'il soit
bien maître de lui, pour ne pas aug%
menter son jeu, s'il est en perte. Les
exemples d'une pareille modération
sont très-rares, et les exemples de
\folio{96}
ceux qui, en pareil cas, ont exposé
et perdu leur somme entière, ne le
sont point. Celui-là même qui, por%
teur d'une forte somme, ne joue que
petit jeu, mais qui joue sans inter%
ruption, fait encore une grande faute,
car sa somme s'altère et diminue in%
sensiblement par l'avantage du ban%
quier.

Il ne convient de porter au jeu que
la somme qu'on peut perdre, sans dé%
ranger ses affaires et s'exposer à de
grandes privations.

Je citerai ici ce que la Bruyère, le
moraliste français que j'aime davan%
tage, et que j'ai lu avec le plus de
plaisir, dit de la conduite dans les
choses auxquelles le hasard participe
davantage.

« Le guerrier et le politique, non
\folio{97}
plus que le joueur habile, ne font
pas le hasard, ils l'attirent \emph{et sem%
blent presque le déterminer.} Non-%
seulement ils savent ce que le sot et
le poltron ignorent, je veux dire, se
servir du hasard quand il arrive,
ils savent même profiter, par leurs
précautions et leurs mesures, d'un
tel hasard et de plusieurs à-la-fois.
Si ce point arrive, ils gagnent ; si
c'est un autre, ils gagnent encore.
Ces hommes sages peuvent être
loués de leur bonne fortune, comme
de leur bonne conduite, et le ha%
sard doit être récompensé en eux
comme la vertus ».

Ces réflexions sont plus applicables
aux jeux dans lesquels le savoir et
l'habileté entrent pour quelque chose,
qu'aux jeux de pur hasard, auxquels
\folio{98}
cependant elles ne sont point entière\~%
ment étrangères.

Les joueurs ne se ruinent guères
que par leur inconduite ou par leur
imprudence.

La sagesse et la conduite ont des
effets bien plus sûrs aux jeux de com%
merce, c'est-à-dire, aux jeux mêlés
de hasard et de science, qu'aux sim%
ples jeux de hasard, parce que ces
qualités suffisent souvent pour y
faire pencher le sort du côté de ceux
qui les possèdent, sur-tout s'ils y joi%
gnent l'habileté ; mais je dois dire
aussi que pour la plupart des joueurs,
les jeux de commerce sont bien plus
à redouter que les jeux de hasard ;
car la faveur des chances y cède  pres%
que toujours à la supériorité du ta%
rent, supériorité d'autant plus dange%
\folio{99}
reuse, qu'il est aisé de la dissimuler.

Divisez la somme que vous consa%
crez au jeu en un nombre de portions
que vous fixez suivant les coups que
vous voulez jouer, ou les mises que
vous voulez faire.

Les mises au jeu doivent être faites
en proportion de ce qu'on craint de
perdre ou de ce qu'on aspire à ga%
gner.

Il vaut mieux, lorsqu'on perd,
diminuer sa mise que de l'augmen%
ter. Alors, comme le jeu a ses varia%
tions, si vous gagnez, vous pouvez
ressaisir votre argent ; si vous conti%
nuez de perdre, vous vous félicitez
de n'avoir pas, par de plus fortes
mises, rendu votre perte plus consi%
dérable ;  et cette réflexion contribue
à vous consoler.

\folio{100}
Êtes-vous en gain, doublez pro%
gressivement vos mises ; faites des
parolis, n'exposez point de votre ar%
gent avec celui que vous gagnez ; au
contraire, retirez à mesure partle de
ce gain : fixez le nombre des coups
de gain après lesquels vous retirerez
tout ce qui vous appartiendra au ta%
pis, sauf à recommencer le même
jeu.

Voici un exemple de cette manière
de jouer :

Vous avez sur vous vingt écus que
vous voulez hasarder successivement :
vous les divisez en dix portions, afin
que votre mise simple soit de six
francs. Vous jouerez donc d'abord six
francs ; si vous perdez, vous jouerez
encore six francs : il est indifférent
que ce soit au coup suivant ou que
\folio{101}
ce soit après que vous aurez laissé
passer plusieurs coups sans jouer ;
si tous gagnez, vous retirerez votre
mise ; si vous gagnez le second coup,
vous laisserez au jeu les douze francs
qui vous reviennent ; si vous gagnez
le troisième coup, vous retirez six
francs sur les vingt-quatre qui vous
reviennent, et vous laisserez au jeu
les dix-huit francs. Si vous gagnez
le quatrième coup, comme les séries
au-delà sont rares, et qu'au jeu que
vous jouez, une grande ambition ne
vous est pas permise, retirez douze
francs et n'en laissez que six. Con%
duisez-vous comme si vous commen%
ciez la première partie que je vous
indique plutôt pour vous faire con%
naître l'esprit dans lequel vous devez
jouer, que pour vous indiquer une
méthode déterminée.

\folio{102}
Gardez-vous sur-tout de penser que
vous serez plus près du gain en pour%
suivant une chance à laquelle vous
venez de perdre ou qui a été long-%
tems sans paraître. C'est un préju%
gé qui a été funeste à beaucoup de
joueurs que celui de croire qu'une
chance ou qu'un numéro en retard
est plus prêt qu'un autre à arriver.
Peu de personnes, il est vrai, ont
l'esprit assez sain et la tête assez froide
pour se garantir de ce préjugé, et ne
pas suivre les mouvements qu'excite
en eux leur imagination étrangement
trompée. Les hommes plus sensés
et les plus spirituels ne me nieront
pas qu'en pareille circonstance ils
n'ont pas su résister à l'empire de ces
mouvemens. On augmente sa mise
avec une effrayante progression ;
\folio{103}
bientôt le jeu est plus de la fu%
reur ; c'est un inconcevable aveugle%
ment, c'est une rage. J'ai vu alors des
joueurs obstinés, jusques-là tran%
quilles en apparence, mettre au jeu
en rugissant, or, billets, monnaie,
tout ce qu'ils avaient sur eux. Une
pareille erreur a causé plus d'une
ruine et plus d'un suicide ; elle ne
vous jeterait pas dans ces excès, mais
elle pourrait vous faire perdre en un
instant l'argent que vous aviez ré%
parti de manière à vous occuper pen%
dant une partie de la séance.

Persuadez-vous donc que chaque
coup étant isolé, étant indépendant
de ceux qui le précédent, n'étant que
le résultat de mouvemens plus ou
moins rapides, mais toujours incer%
tains, le retard de sortie d'une chance
\folio{104}
ou d'un numéro ne peut fonder au%
cune probabilité en sa faveur.

D'autres préjugés rendent les crain%
tes des joueurs aussi peu raisonnables
que leurs espérances. Pour suivre le
petit plan que vous vous êtes fait, et
dont l'exécution serait du moins un
amusement pour vous, puisque l'é%
vènement du jeu ne pourrait ni trou%
bler votre repos, ni compromettre
votre fortune, vous auriez besoin de
mettre sur une chance ; mais cette
chance est proscrite ; tout le monde
l'abandonne ; vous l'évitez comme on
évite quelquefois un ami dans le mal%
heur. Elle vient à gagner, et au regret
d'avoir écouté une fausse prévention,
se joint l'embarras de se faire un au%
tre plan. Comme on ne doit jouer
qu'à un jeu dont on connaît bien la
\folio{105}
théorie, et qu'on sait n'avoir d'autres
perfidies que celles dont le sort est
capable, et dont aucune réflexion,
aucune prudence ne peut vous ga%
rantir, il faut y apporter cette assu%
rance, j'allais dire cette confiance
qui donne au jeu plus de charme
lorsque le succès la justifie, et que
du moins la raison ne condamne
pas, si elle est trompée par l'évè%
nement.

Aux jeux de hasard, dont les mou%
vemens sont si prompts, si variés,
où les tentations renaissent à toutes
les minutes, et où les sens sont agités
si puissamment, même par d'autres
intérêts que les nôtres, il est bien
difficile de suivre le plan qu'on s'est
lait d'abord, et de se maintenir dans la
contention d'esprit qu'il exige : rien
\folio{106}
n'est cependant plus nécessaire, si
on veut éviter l'embarras de volon%
tés incertaines ou contraires l'une à
l'autre, et le défaut de règle à la suite
duquel vient toujours le défaut de
conduite.

Le sang-froid et la résignation font
mériter le titre de beau joueur, qui
est encore au jeu une sorte séduc%
tion : s'ils ne rendent pas le hasard
plus favorable, ils mettent plus en
état de profiter des avantages qu'on ob%
tient, et de ne pas accroître ses revers.

Lorsqu'on a peu d'argent et qu'on est
jaloux de le conserver, ou lorsqu'on
veut s'occuper au jeu quelque tems,
il faut jouer à des chances où le gain
et la perte puissent être en proportion
égale ; par exemple, si ne jouant à la
roumette qu'une pièce à la fois, vous la
\folio{107}
mettez ou à une transversale ou à un
carré, votre argent, à moins d'une fa%
veur du sort peu ordinaire, sera bien%
tôt épuisé. J'ai vu des joueurs, ou
plutôt des joueuses avides \paren{car cette
imprudence est plus commune aux
femmes}, n'ayant qu'une somme mé%
diocre, jouer sur deux numéros, ou
jouer sur un plein. Leurs lamenta%
tions, lorsqu'elles perdaient, n'étaient
assurément pas fondées.

Quoi qu'on en dise, j'ai reconnu
que le moyen d'obtenir au jeu les fa%
veurs de la fortune, n'était pas de la
brusquer.

Lorsqu'on est en gain, on est pres%
qu'excusable de tenter des coups plus
hasardeux, et d'aspirer même à une
forte somme. Le succès a quelquefois
couronné cette hardiesse, qui s'allie
avec la prudence. Perd-on son béné%
\folio{108}
fice ? on fait preuve de sagesse et de
conduite en revenant à ses mises sim%
ples, après avoir joué beaucoup d'ar%
gent.

Mais s'il est pour des joueurs qui
ont sur eux de fortes sommes, un sys%
tème désastreux et qu'il faille s'atta%
cher à proscrire, c'est celui des mar%
tingales portées à un grand nombre
de coups. Quelque faible que soit la
somme qui les commence, et quelque
régulière que soit la progression qu'on
y observe, elles menacent toujours du
plus grand danger. Le joueur imite
alors celui qui porte du feu dans un
magasin à poudre ; il peut y entrer cin%
quante fois sans qu'il lui arrive d'acci%
dent ; mais l'instant fatal de l'explosion
vient tôt ou tard ; il pouvait venir à la
première imprudence.
