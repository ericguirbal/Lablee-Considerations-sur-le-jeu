\chapter{Tripots}

\folio{175}

\lettrine{B}{eaucoup} appellent \emph{tripots} tous les
lieux où l'on se rassemble pour jouer :
c'est abuser du mot, et c'est ainsi 
qu'en confondant sous un seul nom,
ou sous un seul titre défavorable, des
choses ou des personnes qui ont dif%
férens caractères, on enlève au vice
ce qui serait le plus capable de le si%
gnaler et de la faire éviter ou pros%
crire.

On peut donner le nom de tripots
aux maisons de jeu, qui, quoiqu'elles
soient autorisées, ont une mauvaise
tenue et une partie des dangers de la
clandestinité.

\folio{176}
On peut aussi donner ce nom aux
maisons, soit de ville soit de cam%
pagne, où des particuliers abusent de
la liberté qu'ils ont de faire jouer leurs
amis réels ou prétendus, où l'on trouve
un peu de ce qu'on appelle mauvaise
compagnie, et où ne règnent point
l'ordre, la décence, la modération et
la probité ; mais il appartient sur%
tout aux lieux où, par entreprise, on
fait jouer clandestinement et sans au%
torisation à des jeux de hasard et à
d'autres jeux.

Dussaulx a indiqué l'esprit qui rè%
gne dans ces lieux, en disant : « Les
maisons trop attentives ou trop
difficiles au gré de certains joueurs,
en font refluer une partie dans les
tripots ».

En effet, on y voit figurer d'un ton
\folio{177}
avantageux des gens qui n'oseraient
reparaître dans des maisons publi%
ques.

Les cartes ou adresses qu'on fait
distribuer par de prudens affidés,
pour attirer dans ces lieux, n'annon%
cent certainement point ce qui s'y 
passe.

Il est difficile pour ceux qui n'ont
pas la connaissance du monde, de
n'être pas subjugués par les politesses,
les coups-d'{\oe}il, les agaceries des fem%
mes, qu'on a, en général, le soin de
faire paraître à la tête de ces jeux. Ce
sont d'aimables hôtesses dont, comme
on l'a déjà dit, on ne reconnaîtrait
pas les droits dans la maison, sans le
zèle qu'elles mettent à arranger les
parties, à échauffer le jeu, et à re%
commander le flambeau.

\folio{178}
Quelques mamans prévoyantes re%
cherchent de pareils emplois, qui 
leur donnent la facilité de marier
leurs filles.

De pareils traits sont dans nos 
m{\oe}urs, dont la réforme est le pre%
mier pas à faire vers un meilleur
ordre de choses.

L'histoire des sots ou scandaleux
mariages occupe une grande place
dans celle des jeux.

Des tripots fourniraient aux pein%
tres de bons sujets de caricatures, si
l'on pouvait s'amuser des ridicules
où l'on a trop souvent à gémir sur la
perte de la pudeur et de la bonne foi.

Il semble quelquefois que, pour
former ces parties, on a été choisir
dans les quatre parties du monde ce
qu'il y a de plus étranger au langage
\folio{179}
ordinaire et aux simples usages de la 
société.

Avez-vous remarqué en quelque
lieu une figure ou sinistre ou suspecte ?
si vous êtes dans ces tripots, regardez
autour de vous ; cette figure est là, ou
ne tardera pas à paraître.

Il y a peu de tripots où les femmes
qui les dirigent ne soient d'intelli%
gence avec quelques habitués pour
mettre à contribution le joueur novice
ou l'imprudent étranger.

Défiez-vous-y de ces hommes polis
et déliés qui s'y trouvent à toute
heure, et qui sont aux petits soins
avec les maîtresses de maison. Si vous
négligez le flambeau, ils ne manque%
ront pas de vous en avertir ; mais dans
un coup douteux, n'espérez pas que
leurs avis soit pour vous.

\folio{180}
Lorsque la bouillotte se joue dans
un tripot, il y au moins un joueur
qui ne joue pas avec loyauté.

C'est là principalement que les jeux
qui exigent du savoir et de l'habileté,
causent plus de ravages que les jeux
de hasard.

Voulez-vous quitter lorsque vous
êtes en gain, vous manquez à tous les
usages de la société. Pour être hon%
nête, il faut perdre.

Êtes-vous sensible à une injure, la
réparation ne vous est pas refusée ;
mais ne vous attendez pas à être en
tête-à-tête dans cette nouvelle partie.
Faut-il vous le répéter ? toutes les
mesures sont prises pour que, soit au
jeu, soit au combat, vous soyez une
victime.

L'art d'engager une querelle à pro%
\folio{181}
pos, et de tirer parti de la susceptibi%
lité d'un joueur, est familier à des ha%
bitués de tripots, et n'est pas point inconnu
à des amis de maisons particulières
où l'on joue gros jeu.

Ce n'est pas sans raison qu'on donne
à ces tripots les noms d'\emph{étouffoirs} et de
\emph{coupe-gorges:} il faut les leur conserver.

On appelle \emph{grecs} ou \emph{malins}, les
joueurs fins ou adroits, que des en%
trepreneurs de jeux savent mettre
dans leurs intérêts.

Les soupers sont les n{\oe}uds qui lient
les parties de jeu qu'on veut prolon%
ger. Que de trames coupables sont
cachées par le voile de la nuit !

Et si quelque désir impatient est
allumé en vous, par les soins attentifs
d'une jeune beauté qui vous montre
de l'intérêt, dans ces momens où la
\folio{182}
joie du gain a besoin de partage, où
le chagrin de la perte a besoin de con%
solation, ah ! puisse le jour ne pas
vous être importun ! et puisse l'éga%
rement du jeu être le seul sujet de
votre repentir et de vos larmes !

Le sort ne vous a pas été contraire :
vous croyez avoir augmenté la som%
me que vous avez apportée au jeu ;
mais rentré chez vous, examinez de 
près les pièces d'or et d'argent que vous
avez reçues, ou elles sont fausses, ou
elles sont tellement rognées, qu'au
lieu d'avoir gagné, vous avez réelle%
ment perdu.

Vous n'oserez vous plaindre ; et
contre qui vos plaintes seraient-elles
dirigées ?

Mais un autre accident vous fait à-%
la-fois rougir et gémir de la faiblesse
\folio{183}
que vous avez eue de vous être laissé
entraîner dans un de ces tripots. Hier
de faux agens de police y ont saisi, au
profit de l'entreprise du jeu, tout l'ar%
gent qui était au tapis : aujourd'hui
ceux qui ont su s'y introduire, avaient
titre pour cette expédition. Vous avez
aussi été arrêté et conduit devant l'au%
torité surveillante ; et sur la liste des
joueurs du nombre desquels vous
étiez, votre nom a été mis à côté
d'autres noms que demain la punition
du crime va flétrir.

J'en pourrais dire davantage sur ce
sujet ; mais il repousse ma plume, et
les maux que le jeu traîne à sa suite
ont quelque chose de si triste et de si
cruel, que je me félicite d'avoir plu%
tôt entrepris de les indiquer que de
les peindre.
