\chapter{Des Illusions des Joueurs}

\folio{74}
\lettrine{L}{orsqu'on} pense qu'il est aussi im%
possible de trouver le moindre moyen,
la moindre probabilité de gain à un
jeu dont les chances pour le gain et
pour la perte sont parfaitement éga%
les, qu'il est impossible de tracer sur
l'eau des caractères, ou de construire
une maison dans l'air, lorsqu'on
pense que cette égalité de chances
n'est jamais rompue qu'en faveur de
celui qui tient le jeu, et dont l'avan%
tage suffit pour opérer la ruine du
joueur d'habitude, on ne conçoit pas
l'empressement avec lequel des per%
sonnes, qui ne sont pas dépourvues
\folio{75}
de raison, vont ainsi exposer leur
fortune, troubler leur repos, user
leur tems et leur esprit, en un mot,
consumer toutes leurs facultés physi%
ques et morales. Imaginez un homme
qui sort de chez lui avec tout ce qu'il
possède d'argent, et le joue à croix
ou pile contre pareille somme que
met au jeu un premier venu, me le
croiriez-vous pas atteint de folie ? Eh
bien, voilà en beau l'histoire de tous
les joueurs. Et n'est-ce pas ce qu'un
joueur pourrait faire de mieux ? Ici il
n'y a pour lui aucune perte de tems ;
il s'épargne de longues transes et
une pénible contention d'esprit; il
est bientôt délivré de l'incertitude, le
plus cruel de tous les états. Ici le jeu
est égal, et on n'a pas contre soi un
avantage ruineux.

\folio{76}
Faisons une autre observation.

L'amour, le vin, la table, les jeux
d'exercice, affectent du moins agréa%
blement les sens, et si leurs excès
nous causent aussi de grands maux,
nous jettent dans de grands désor%
dres, ce n'est qu'après nous avoir
procure des plaisirs ; il n'en est pas
ainsi des jeux de hasard, et, pour
s'en convaincre, il suffit de jeter un
coup-d'œil sur les figures groupées
autour d'une table de roulette ou de
trente-un. J'invoque le témoignage
des joueurs ; le gain même donne de
la gravité, et laisse une vague inquié%
tude : on n'est point ainsi lorsqu'on
tient dans sa main le prix de son
travail.

Comment ce fait-il donc que la cu%
pidité, qui donne de l'esprit aux plus
\folio{77}
Sots aveugle ici les hommes qui ail%
leurs montrent le plus de lumières et
de finesse ? Et quel est donc ce charme
puissant qui attire dans un précipice
ceux-même qui en connaissent toute
la profondeur ?

Je reconnais là un des effets les
plus merveilleux de l'imagination : oui
c'est l'imagination, trompée par tout
ce qui est capable de la séduire, qui
nous fait voir les choses moins telles
qu'elles sont, que telles que nous dé%
sirons qu'elles soient et c'est le plus
souvent le besoin d'être remué, d'être
jeté, pour ainsi dire, hors de soi par de
fortes sensations, qui nous fait obéir
à l'impulsion donnée par notre imagi%
nation, et rechercher une position
qui, par son attrait et son danger, met
en mouvement toutes nos facultés.

\folio{78}
« En général, le jeu nous plaît, dit
Montesquieu, parce qu'il attache
à l'espérance d'avoir plus ; il flatte
notre vanité par l'idée de la préfé%
rence que la fortune nous donne,
et de l'attention qu'ont les autres
sur notre bonheur ; il satisfait no%
tre curiosité, en nous procurant un
spectacle ; il nous donne les diffé%
rens plaisirs de la surprise. »

On ne peut caractériser l'amour
du jeu d'une manière plus vraie, plus
piquante et plus précise. Voilà bien
la manière des grands maîtres. Je vais
tâcher de donner du mouvement à ce
caractère.

L'ardeur du jeu vous presse : il n'y
a qu'un instant vous éprouviez un
mal-aise, une anxiété dont vous vou%
liez être délivré ; vous aviez pensé à
\folio{79}
une situation dans laquelle vous vous
trouveriez beaucoup mieux : ce mal%
aise, c'était ou le poids de l'ennui, ou
celui du besoin, ou le désir importun
de jouissances que vous ne pouvez
vous procurer : cette meilleure situa%
tion, c'est le jeu ; car lorsque vous
serez au jeu, vous aurez bientôt ce
qui vous manque. Vous y voyez d'ail%
leurs une agréable distraction ; vous
êtes frappé par l'idée d'un succès peu
ordinaire ; vous calculez déjà les heu%
reux effets de votre savoir et de votre
prudence : si vous aviez en votre pou%
voir une ressource plus sûre contre
l'ennui, ou plus conforme à la nature
de vos vœux, vous l'adopteriez de
préférence.

Cependant vous êtes loin du lieu où
vous pourrez essayer des parolis et des
\folio{80}
martingales, que vous avez négligés
jusqu'ici, et que vous ayez vu réussir
à d'autres. Qu'importe ? la route s'a%
brège par vos réflexions sur votre
prochain bonheur, et par l'emploi
idéal des bénéfices que vous allez faire.

Vous rencontrez un ami, confident
ordinaire de vos projets : il trouve que
vos calculs manquent de base, et ne
se rapportent à rien ; mais quel est le
joueur qui ne repousse pas, comme
les efforts d'un démon jaloux de son
bonheur, toute idée qui contrarie ses
espérances ? Est-il possible d'ailleurs
que ce que vous désirez si fortement,
ayant pris dans votre tête ardente le
caractère de la certitude, ne vous
paraisse pas infaillible ? Vous quittez
votre ami pour aller donner par le
fait un démenti à son opinion. Enfin
\folio{81}
vous respirez l'air qui vous est favo%
rable ; vous êtes aux salons de jeu :
l'accueil qu'on vous y fait vous sem%
ble d'un bon augure ; tout vous y sou%
rit ; vous y souriez à tout. La brillante
lumière que des lustres nombreux ré%
pandent dans toutes les parties de ces
salons dorés, les valets qui s'empres%
sent de venir vous offrir leurs soins,
ces monceaux d'or et d'argent dont
des joueurs, habiles sans doute, at%
tirent à eux des débris, l'absence to%
tale des images de gêne et de misère,
cet air de fête, cette aimable hilarité
qui épanouit toujours le visage des
joueurs aux commencemens des
parties, tous les objets enfin pren%
nent à vos yeux une teinte douce et
flatteuse. Votre esprit, déjà calmé, se
repose mollement dans la contempla%
\folio{82}
tion des tableaux de jeu ; votre œil
est recrée par la variété des chances
qu'ils présentent. Quelle abondante
moisson est offerte à l'art des combi%
naison ! Vous ne pouvez vous défen%
dre d'une émotion légère, mais vous
vous gardez de montrer un empres%
sement qu'on prendrait pour de l'a%
vidité. Vous êtes un instant témoin
de la lutte qui vient de s'engager au
trente-un entre les pontes et le ban%
quier. --- Oh ! quelle faute, dites-vous
en vous-même, vient de faire cette
dame en mettant dix louis sur la cou%
leur qui a passé sept fois ! Il est bon
de mettre sur la gagnante ; mais au
huitième coup, quelle folie ! Cette
dame gagne ; vous en êtes étonné ; elle
fait paroli ; vous haussez les épaules :
elle gagne et laisse tout au jeu ; elle
\folio{83}
réussit encore. Alors elle retire. avec
sa mise, son bénéfice de soixante-dix
louis : vous ne lui reprochez pas moins
d'avoir joué contre toutes les règles
et toutes les probabilités. Au coup
suivant la couleur perd. « Voyez, lui
dites-vous, ce qui vous serait ar%
rivé si vous aviez fait un pas de
plus ! --- Oh ! je m'en étais douté,
vous répond-elle, mes pressenti%
mens ne me trompent guères. »

Vous portez votre attention sur un
joueur qui attendait ce moment pour
commencer une martingale à la con%.
tre-couleur : vous êtes tenté de jouer
le même jeu, mais vous vous êtes fait
la règle de ne vous engager que sur
la noire, et lorsque la rouge aura pas%
sé cinq fois ; vous savez que rien ne
porte plus malheur au jeu que de ne
\folio{84}
pas tenir à sa première idée. « Ce mon%
sieur, dit un voisin, est le joueur
le plus sage que je connaisse ; il ne
manque jamais son coup, et se re%
tire chaque jour avec une vingtaine
de louis de bénéfices. »

Cependant la couleur sort six fois ;
le martingale saute. « Ma faute,
dit-il, est de n'avoir pas pris plus
d'argent sur moi. »

Tandis que vous vous applaudissez
d'avoir su résister à la tentation, un
autre joueur murmure. « Je ne perds,
dit-il, que dans cette infernale mai%
son. Puis reprenant sa sérénité :
Je suis sur d'être plus heureux dans
une autre ; » et il sort. « Il n'est point
étonnant qu'il ait perdu, » observe
un homme âgé, à qui l'on attribue une
longue expérience du jeu; » il s'ob%
\folio{85}
stine à suivre une chance qui, de%
puis quinze jours, n'a réussi à per%
sonne. »

Enfin la rouge a passe cinq fois :
vous commencez vos parolis ; mais
les intermittences durent un quart%
d'heure ; mais le refait du trente-un
est fréquent : vous renoncez aux pa%
rolis, et augmentant vos mises, vous
ne jouez plus que sur la rouge. Il vient
une longue série de noires. Dix fois
vous vous êtes dit : \emph{La prudence veut
que je me retire ;} des combinaisons
qui vous ont paru plus sûres, vous
ont tenu enchaîné au jeu : vous êtes
d'ailleurs encouragé et cousolé par
les éloges de vos voisins, qui, applau%
dissant à la manière dont vous avez
coutume de faire vos mises, rejettent
vos pertes sur une fatalité dont il y a
\folio{86}
peu d'exemples. Rien ne vous réussit.
Vous remarquez dans la foule une fi%
gure qui vous parait sinistre ; vous
donneriez une partie de ce qui vous
reste pour être débarrassé de cette
vue qui vous occupe malgré vous, et
à laquelle vous attribuez toutes les
fautes que vous avez faites. Incapable
de réflexions, vous jouez sans plan,
sans ordre ; alors tout vous prospère ;
vos fonds sont rentrés ; ils se doublent ;
c'est l'instant de la retraite. En vous y
disposant, vous promenez d'un air a%
vantageux vos regards sur la galerie,
étonnée sans doute de l'habileté avec
laquelle vous avez su faire fléchir le
sort ; vous croyez lire sur tous les vi%
sages que la joie de votre triomphe
est partagée, elle en devient plus
vive.

\folio{87}
En vous retirant, vous traversez
une salle de roulette. Un homme qui
vous a toujours porté bonheur, vous
salue et vous attire auprès de lui :
vous lui annoncez votre gain et votre
intention de vous y tenir. Cependant
d'heureuses tentatives de quelques
joueurs vous rappellent votre plan
favori de martingale. Après quel%
ques hésitations, vous en faites l'es%
sai aux petits ecus : vous avez pour
vous les cinq-sixièmes des numéros ;
mais votre martingale est déjà portée
assez haut pour que vous ayez sur
le tapis toute la somme que vous avez
gagnée au trente-un. Les zéros tom%
bent ; la terreur s'empare de vous :
vous avez dans la main, pour un der
nier coup, tout l'argent que vous avez
apporté au jeu ; vous n'osez vous en
\folio{88}
dessaisir : laboule se câse; vous auriez
gagné ; votre cœur se comprime ;
vos idées s'égarent ; vous restez long%
tems stupéfait ; mais vous savez que
les yeux sont fixés sur vous ; vous
craignez qu'on ne vous accuse de
timidité, ou qu'on attribue votre re%
tenue à l'épuisement de votre bourse.
Vous vous avisez alors de réaliser un
projet de distribution de mises iné%
gales sur une partie du tableau de
la roulette. Combien de fois n'avez%
vous pas reconnu l'avantage de ces
mises compliquées ! Infortuné ; vo%
tre imagination vous promène d'er%
reurs en erreurs, jusqu'à la fatale
catastrophe où vous ne verrez plus
que la vérité. En vain vous variez ves
mises, en vain vous vous emparez
des trois-quarts tableau, votre
\folio{89}
opération n'est point compliquée ;
elle est simple ; vous jouez un contre
un, et vous avez toujours contre vous
les zéros, terrible ascendant qui se
fait une fois moins sentir à celui qui
joue sur de simples chances, comme
la rouge, le passe, 1 impair, car il a
le privilége des refaits auxquels, par
vos mises concentrées sur les numé%
ros, vous avez renoncé. Vous pros%
pérez un instant, mais bientôt votre
chute est rapide, et vous payez
cher l'erreur funeste qui vous a fait
préférer le moyen le moins propre
à écarter de vous le besoin et les en%
nuis, ou à vous procurer des jouis%
sances.

Ainsi finit le jeu ; vous avez d'a%
bord été épris de l'éclat et des traits
d'un beau visage ; vos regards s'y
\folio{90}
sont fixés ; mais insensiblement vous
avez vu cet éclat s'affaiblir, ces traits
s'altérer, et vous n'avez plus eu de%
vant les yeux qu'une affreuse tête
de mort qui a porté dans vos sens
l'horreur et l'épouvante.

Tandis que vous retournez d'un
pas lent dans vos obscurs foyers, vo%
tre ame est plongée dans une morne
tristesse, et votre esprit s'occupe en
vain de la recherche des moyens de
remplir le lendemain des engage%
mens sacrés, et d'alimenter vos mat%
heureux enfans qui vous attendent.

Cruel aveuglement ! ce que vous
regrettez davantage, c'est de n'avoir
pu porter au jeu une plus forte som%
me. Vous croyez que s'il vous avait
été possible de jouer quelques coups
de plus, le sort aurait cessé de vous
\folio{91}
être contraire ;  et si une espérance
vous ranime par intervalles, c'est
celle de recouvrer le lendemain ce
que vous avez perdu en livrant aux
hasards du jeu le prix de là vente des
effets qui vous sont le plus nécéssaires.

Voilà une faible esquisse des illu%
sions du joueur : mais comment les
attaquer ? Comment les détruire ? Je
le répète ; je doute qu'on y réussisse
par le langage austère de la morale,
et par la touchante effusion du sen%
timent : je n'en suis pas moins per%
suadé, je prouverai peut-être que les
lois prohibitives, que des mesures de
rigueur ne feraient aujourd'hui que
rendre l'incendie plus considérable.
Encore une fois, les joueurs calcu%
lent ; il faut compter avec eux ; il faut
que les vérités de calcul les plus sim%
\folio{92}
ples, les plus claires, les poursuivent,
les atteignent, frappent leurs oreilles
et leurs yeux, s'emparent de tous
leurs sens dans les maisons de jeu,
et que leurs pertes leur paraissent
inévitablement la preuve de la règle
dont vous leur avez donné la leçon.

Et nous, écrivains moralistes, n'ou%
blions pas que l'homme du peuple
qu'on n'avertit point assez, par des
écrits à sa portée, des dangers aux%
quels l'expose sa cupidité, se rendra
plutôt à la démonstration du préju%
dice que cause aux joueurs honnêtes
l'inégal contrat du jeu, que l'homme
du monde qui, livré à ses désirs et
à ses passions, est entouré de plus de
prestiges.
