\chapter
  [Maisons publiques de Jeu]
  {Maisons publiques de jeu \notemark}

\notetext{En 1803.}

\folio{151}
\lettrine{O}{n} a eu, ou on doit avoir du pour
principal but, en établissant des mai%
sons publiques de jeu, de donner un
contre-poids aux dangers et aux dé%
sordres auxquels le jeu expose dans
des tripots et des maisons particu%
lières.

L'emplacement et le nombre des
maisons publiques sont réglés par
l'autorité surveillante.

Il y a différentes maisons de jeu
pour différentes classes de citoyens ;
la plus marquante est le salon de la
Paix, rue Grange-Batelière. Outre
les jeux dont j'ai parlé et les jeux
de commerce, on y joue aux krabs,
fameux jeu anglais. Ce salon, monté
sur le ton des maisons les plus opu%
lentes, est fréquenté par une société
choisie, dans laquelle se trouvent des
personnes jouissant d'une bonne ré%
putation, et beaucoup d'étrangers
distingués par leur nom, leur rang
ou leur fortune. On y entre par ca%
chets, ainsi que dans la maison des
arcades du Palais Royal, et dans
quelques autres.

La précédente administration avait 
la faculté d'étendre le nombre de ces
maisons, et elle en usait : elle avait
sous elle une administration ambu%
lante, et envoyait des missionnaires
dans les départemens, en Italie, etc.

L'administration des jeux tient pour
son compte, et avec ses propres fonds,
un petit nombre de maisons. D'autres
maisons subalternes ont seulement
des permissions ou privilèges, pour
lesquels une rétribution est payée à
l'entreprise générale, suivant la loca%
lité et la nature du jeu.

On a réduit de plus de moitié le
nombre des maisons où va joueur la
classe ouvrières.

Lorsque dans les maisons particu%
lières, des jours de fête ou de bal,
on veut faire jouer des jeux de hasard,
l'administration y envoie des tables et
ustensiles du jeu demandé, des tail%
leurs, et employés, et même des
fonds. Le minimum de la mise pour
chaque coup dans presque toutes les
\folio{154}
maisons, est de trente sous \notemark. Anté%
\notetext{
  Le minimum de la mise dans les maisons publiques
  \emph{subalternes}, était de trente sous. Depuis quelques années
  il a été fixé à deux francs.
}%
rieurement, plusieurs avaient la fa%
culté désastreuse de recevoir de moin%
dres mises. Depuis cette rigoureuse
fixation, on voit aux jeux publics
moins d'artisans et de cultivateurs.

La ferme des jeux n'est que pour
Paris ; son bail n'est que pour une
année, et peut se résilier \notemark.
\notetext{
  Présentement le privilège de la ferme des jeux
  s'étend dans plusieurs lieux où les eaux se prennent.
}
Pour son bénéfice ou son avantage,
l'administration a, au trente-un, les 
refaits du trente-un ; à la roulette, le
zéro et le double zéro ; au biribi, une
petite colonne particulière ; au pair
et impair, un certain nombre de
\folio{155}
points donnés par les dés. Là-dessus,
je n'ai pas besoin de m'expliquer
mieux ; ceux qui ont le bonheur d'être
étrangers au jeu n'ont pas besoin de
me comprendre : je ne le serai que
trop par ceux qui ont le malheur de
les pratiquer.

Dans les maisons publiques, les fri%
pons n'ont rien à faire contre la ban%
que. Les yeux des surveillans et des
spectateurs tiennent lieu de cons%
cience à ceux qui n'en ont pas.

Les joueurs aussi n'ont rien à re%
douter de la banque ; outre que les
ruses et les fraudes n'y seraient guè%
res au pouvoir des tailleurs les plus
adroits, les joueurs ont une garantie
de leur fidélité dans l'intérêt des pon%
tes, la haine et la jalousie des spec%
tateurs bénévoles qui inclinent ordi%
\folio{156}
nairement contre la banque, et les
regards de tous habituellement fixés
sur les mouvemens de ces tailleurs.
Je ne hasarde rien, sans doute, en
assurant qu'on peut aussi compter sur
la loyauté et la probité des employés
actuels dans les jeux. En général,
ces employés sont scrupuleusement
choisis d'après de rigoureuses infor%
mations. Leur air, leur politesse, leur
langage, annoncent qu'ils sont bien
nés, et ont reçu de l'éducation. La
plupart sont fils de familles ruinées
ou mutilées dans le cours des dé%
sastres publics, ou ont été employés 
dans les armées et les administrations,
et se sont trouvés inoccupés par l'effet
des réformes nécessaires, ou sont
victimes de malheurs particuliers.
L'entreprise des jeux a un si grand
\folio{157}
intérêt à n'y placer que des personnes
sur la probité desquelles elle puisse
compter, qu'à cet égard on ne peut
la soupçonner de négligence. C'est
par de pareils choix sans doute qu'elle
a tâché de se sauver en partie de la
défaveur dont l'opinion frappe des
opérations telles que les siennes. Je
crois donc qu'on pourrait avoir un
motif de plus de confiance dans un
employé qui, après quelque tems
d'exercice, serait sorti sans reproche
d'une administration telle que celle
des jeux. Ce ne serait pas par des
exceptions qu'on serait fondé à ne
pas trouver de la vérité dans mes
observations.

Les maisons publiques, telles qu'el%
les sont aujourd'hui, ont encore d'au%
tres avantages qu'on ne trouverait
pas ailleurs.

\folio{158}
Il y a évidemment moins de dangers
que dans des lieux où se jouent des
jeux qui tiennent beaucoup de l'a%
dresse et de l'habileté. On sait du
moins à quoi on s'expose.

Non-seulement on n'y dépouille
personne de dessein prémédité, mais
ke joueur qui entre et sort librement,
absolument maître de son sort, peut
quitter à volonté sa partie adverse,
sans craindre ni reproches, ni mur%
mures.

Le joueur, s'il est en perte, peut
prendre la revanche contre le ban%
quier, et la refuser s'il est en gain.

Dussaulx paraît attacher du prix à
cet avantage.

Pour quelqu'un qui n'est pas riche
et qui a le malheureux goût du jeu,
ces maisons sont encore préférables
\folio{159}
aux maisons particulières, où l'on
sonde pour ainsi dire la bourse, où
l'on règle sur cette connaissance la
manière dont on doit se comporter 
avec vous, où les égards, d'ailleurs,
sont toujours proportionnés à la for%
tune, et où la fausse honte et la crainte
du discrédit vous font souvent jouer
plus gros jeu que vous n'y seriez na%
turellement porté.

Si on gagne aux jeux publics, on
n'a pas le remords d'avoir soi-même
plongé des malheureux dans la mi%
sère et le désespoir.

Ce n'est pas là aussi que se forment
des liaisons souvent plus nuisibles
que de fortes pertes. Les pontes n'y
ont affaire qu'aux tailleurs, et tout
le tems y est pris par le jeu, qui exige 
le silence.

\folio{160}
A la vérité, les banques y font de
forts gains ; mais les maîtres de mai%
sons particulières ou de tripots ne
lèvent-ils pas sur vous de plus fortes
contributions ? Il faut encore le re%
connaître, ce maisons ne sont pas
sans agrémens pour des personnes
qui, sans manquer de raison et de con%
duite, puevent ne pas aimer les com%
mérages, le cérémonial ennuyeux,
et toutes ces petites malignités qu'on
n'évite point dans les sociétés ordi%
naires. Et, je le répète, il est des 
peronnes à qui la distraction d'un
faible jeu est nécessaire.

Si les maisons publiques ne sont 
pas sans avantage pour les amateurs
du jeu, elles en ont aussi pour les
mœurs et pour l'ordre public. Là,
peu de joueuses osent se montrer, et
\folio{161}
elles ont peu de communication avec
les joueurs. Je crois être dispensé de
dire ce que produit dans d'autres lieux
le mélange de joueurs des deux sexes :
on peut s'en former une idée ; mais
voici des considérations d'une autre
importance. N'est-il pas, je le de%
mande, et moral et politique d'amener
les grands joueurs à comparaître de%
vant le public, pour le rendre le té%
moin de leur audace et le surveillant
de leur loyauté ? N'est-ce pas en for%
cer un grand nombre à la probité et
et à la modération ? Là, du moins,
les fortes pertes, les ruines, sont des
leçons qui profitent et aux joueurs
et à ceux qui seraient tenté de le
devenir. Il n'en est point ainsi des au%
tres rendez-vous de jeu.

Enfin, je dois rendre justice à l'ex%
\folio{162}
cellente police qui s'exerce, pour
ansi dire sans se montrer, dans les
maisons de jeu. Jusqu'ici des maisons
de ce genre n'avaient point donné
l'exemple d'autant d'ordre, de calme
et de décence.

Il y a un commissaire du gouver%
nement près les jeux.

Si mon attachement à la vérité et
à la justice me fait dire ici des mai%
sons de jeux publics ce qui est connu
et ne peut m'être contesté ; si je leur
donne hautement la préférence sur
les autres lieux où l'on joue, je n'en
pense pas moins qu'en y entrant on
s'expose aux plus grands malheurs ;
et je regarderais comme un beau
jour pour la chose publique, celui
où la fureur du jeu s'altérant par
dégrés, et ne trouvant plus d'aliment
\folio{163}
dans les antres qui lui sont ouverts
de tous côtés, il n'y aurait plus ni
obstacle ni danger à les détruire.
