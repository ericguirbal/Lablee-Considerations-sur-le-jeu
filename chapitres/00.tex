\chapter{Introduction}

\folio{i}
\lettrine{D}{ans} le roman de la Roulette
\footnote{
  La sixième édition de cet ouvrage, avec
  un tableau gravé et colorié, représentant un
  jeu de roulette, se vend chez Rapet, commissionnaire
  en librairie, rue Saint-André-des-Arcs,
  n.~41. Prix : 1~franc 5o~centimes ;  et
  franc de port, 1~fr. 75 cent.},
j'ai parlé le langage du sentiment ;
j'ai tâché d'offrir des tableaux qui
pussent émouvoir l'imagination ;
j'ai peint un joueur en action,
\folio{ii}
entraîné, comme ils le sont pres%
que tous, par des erreurs de cal%
culs, et livrés aux illusions d'une
passion désastreuse. Quel écrivain,
ami de l'humanité et des mœurs,
n'est pas pressé par le besoin d'ar%
rêter dans leur cours les vices et les
excès qui leur portent le plus d'at%
teinte, et d'attirer tous les regards
sur les pièges tendus à la crédulité,
à l'ignorance et à la faiblesse ?

Je vais encore m'occuper du
même sujet : j'en parlerai dans les
\folio{iii}
mêmes sentimens et dans les
mêmes principes ; mais je le con%
sidérerai sous d'autres rapports.
Je réprimerai tout mouvement
passionné ; et, recherchant plus
ce qui est vrai que ce qui peut
faire sensation, je donnerai à mes
idées un développement plus mé%
thodique. Moins jaloux d'émou%
voir, que d'éclairer et de con%
vaincre, je m'adresserai moins au
cœur qu'à l'esprit ; en un mot,
j'écrirai plutôt sur le jeu que con%
tre le jeu. Lorsqu'on suit un pareil
\folio{iv}
plan, si les effets qu'on produit
sont moins brillans et moins vifs,
ils peuvent être plus sûrs et plus
durables.

Certes, je ne serai jamais l'apo%
logiste des goûts et des habitudes
du jeu ; mais il me semble que la
meilleure manière de les com%
battre, n'est pas d'annoncer d'a%
bord le vœu et l'intention de les
détruire ; et s'il est vrai que leur
destruction soit regardée comme
impraticable, n'y a-t-il pas quel%
\folio{v}
que chose de mieux à faire que
d'appeler sur les joueurs et sur les
lieux qui les rassemblent, le mé%
pris et la proscription ? On a su
extraire des sucs bienfaisans de
plantes vénéneuses ; ne peut-on
enlever au jeu ce qu'il a de plus
dangereux et de plus funeste ?
n'en peut-on du moins tirer quel%
ques fruits ? Ce désordre, ces
pertes sont-ils sans dédommage%
mens, sans compensations ? et n'y
a-t-il aucun bien à côté d'un si
grand mal ?

\folio{vi}
Il faut parler aux hommes éga%
rés par des passions un langage
qui leur soit familier, ou qu'ils
puissent entendre ; il faut comp%
ter, pour ainsi dire, avec eux,
dans leurs propres affaires ; ainsi
on se rend maître de leur atten%
tion, ce qui est déjà avoir beau%
coup obtenu : ils peuvent alors
apercevoir eux-mêmes le danger
qui les menaçait, le précipice
dans lequel ils allaient tomber ;
alors il est plus facile de leur faire
quitter la ligne sur laquelle ils
\folio{vii}
étaient placés, et de les attirer sur
un point qui concilie mieux leurs
intérêts et leurs goûts.

Je vais donc tâcher d'alléger
le poids énorme qu'un destin
aveugle fait peser sur les joueurs ;
et en examinant ce qui doit leur
être ôté, et ce qu'il convient en%
core de leur conserver, je m'ap%
plaudirai, si je peux aussi, par de
faibles, mais nouveaux aperçus,
aider l'administration publique à
remplir un de ses devoirs les plus
difficiles.

\folio{viii}
Je garderai le silence sur les
désagrémens et les défaveurs que
m'ont causé mes ouvrages contre
les jeux. Il en coûte souvent pour
faire connaître d'utiles vérités,
mais les écrivains moraux rem%
pliraient-ils leur devoir s'ils ne
savaient faire le sacrifice de leur
intéret personnel ?
