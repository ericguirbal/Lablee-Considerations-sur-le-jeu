\chapter
  [Etat du Jeu en \emph{1803}]
  {État du jeu en 1803}

\folio{142}
\lettrine{L}{orsque} les français, n'étant plus
comprimés par la terreur, commen%
cèrent à reprendre avec leur caractère
une partie de leurs anciennes habitu%
des, lorsque l'argent rentra dans la
circulation, par-tout on se dédomma%
gea, comme à l'envi, de la privation de
presque toutes les jouissances ; et la
foule des joueurs s'empressa de venir
se ranger autour des tapis de jeux de
hasard. Ceux dont l'état était perdu
ou la fortune altérée, ceux qui n'a%
vaient pas réussi dans leurs spécula%
tions ou dans leurs intrigues, espé%
raient y trouver un prompt moyen de
\folio{143}
réparer leurs pertes ou leurs fautes,
ou d'alimenter leur soif de l'or : des
hommes nouveaux, embarassés de
l'emploi de ce qu'ils avaient gagné dans
un facile agiotage, et aussi peu pro%
pres aux plaisirs ordinaires de la so%
ciété, qu'aux travaux qui demandent
des connaissances, étaient jaloux, par
de fortes mises, d'acquérir aux jeux
publics une sorte d'importance. Cette
fureur de jeu effaça tous les excès dont
Dussaulx avait fait l'effrayante pein%
ture. Le sol français avait été couvert
de bastilles et de tombeaux ; il le fut
de théâtres, de salons de bals et de
maisons de jeu. Les principales villes
des départemens ont eu, comme à
Paris, des parties considérables qui se
sont tenues soit dans des cafés, soit aux
lieux connus auparavant par le nom
\folio{144}
d'académies, soit dans des maisons
bourgeoises, dont les maîtres, réduits
par la révolution à ces propriétés et
à un stérile mobilier, n'ont pas cru
pouvoir trier un meilleur parti. Com%
munément on n'entrait là que par ca%
chets, ou par invitations, ou par pré%
sentations d'affiliés ; mais il n'était pas
difficile d'obtenir ce droit d'entrée. A
la tête de ces maisons étaient des
dames, dont le nom était une sorte de
garantie qu'on trouverait chez elles
de la probité ! Elles attiraient l'habi%
tant et l'Etranger par des repas et des
bals. La police civile ou militaire leur
accordait appui et protection, moyen%
nant une faible rétribution, destinée
au soulagement des pauvres, ou à des
actes de bienfaisance.

Le pharaon, le biribi, le trente-un,
\folio{145}
le passe-dix, ou le pair et l'impair,
sont devenus successivement les jeux
à la mode. Des banquiers ambulans
se transportaient où ils étaient de%
mandés, avec les instruments de jeu,
et le plus souvent en faisaient les
fonds. Mais Paris est toujours resté
le grand théâtre des jeux ; et certes,
à l'époque où ils ont repris leur acti%
vité, à laquelle la rentrée des troupes
dans l'intérieur a beaucoup contri%
bué, il aurait été aussi impolitique
que difficile de les prohiber, ou d'y
mettre de trop fortes entraves. Ils ont
été tolérés, surveillés, et, comme ce%
la devait être, mis à contribution.

Au milieu de ces jeux dont je viens
de parler, s'est élevé le plus impo%
sant et le plus séducteur de tous, \emph{la
roulette.} Les tables s'en sont multi%
pliées, et ont fait déserter une foule
de maisons particulières et de tri%
pots établis ouvertement ou clandes%
tinement, mais qui n'offraient pas
ou autant de liberté, ou autant de
sureté.

L'administration publique, forcée,
pour ainsi dire, de capituler avec les
passions, a tâché de donner de la
fixité aux maisons en possessions d'at%
tirer le plus grand nombre de joueurs,
en leur accordant des privilèges, et
prenant des mesures pour que les
vols, les friponneries, les querelles n'y
fussent point impunis. Si cet ordre et
cette sûreté devaient avoir pour effet
d'accroître le nombre de joueurs,
ils devaient aussi avoir l'avantage
d'assembler sur quelques points ce
peuple cupide, inquiet et efferves%
\folio{147}
cent, jusques-là inégalement répandu
dans des repaires.

Mais bientôt la France ayant joui
d'un système mieux combiné d'admi%
nistration intérieure, et une unité de
mesures ayant dû être adoptée pour
qu'on pût rapprocher davantage de
l'œil du Gouvernement ce qui inté%
resse la tranquilité et les mœurs, les
jeux dans Paris ont été affermés, et la
ferme a été chargée de les administrer
de manière qu'en conservant aux ci%
toyens la liberté de leurs actions, et des
habitudes dont la subite réforme au%
rait pu être mise au nombre des rêves
politiques, cette antique manie fût
purgée de beaucoup d'abus et de 
beaucoup d'excès que l'autorité n'au%
rait pu directement atteindre.

\folio{148}
De cette manière, si la fureur du
jeu n'a point été altérée, du moins les
jeux ont été administrés. Leur état
est aujourd'hui à-peu-près tel que je
viens de le décrire
\footnote{Il faut se reporter à cette année de 1803, dans le
  cours de laquelle j'ai composé et publié ces Considé%
rations.} ; on joue beau%
coup, parce qu'on sent fortement le
besoin de jouer, parce que c'est un
goût né du caractère indestructible
et de la position de beaucoup d'indi%
vidus. L'ardeur du jeu a depuis quel%
ques années gagné dans Paris presque
toutes les classes de la société, et ce
n'est pas dans les maisons publiques
que se joue le plus gros jeu. Chez des
particuliers, on joue entre prétendus
ami ce qu'on appelle un jeu infernal.

La roulette, le trente-un, le pair et
\folio{149}
l'impair, se jouent dans les maisons
publiques de jeu. Depuis quelque
tems, on a réduit de plus de moitié
le nombre de ces maisons : on asure 
que le nombre des joueurs est aussi
diminué, et que les parties ne sont
plus aussi fortes ; mais on ne dit pas,
ou on ne sait pas que depuis la réduc%
tion du nombre des maisons de jeu,
la bouillotte seule, dont les parties se
sont formées librement dans des mai%
sons particulières \paren{à la plupart des%
quelles le nom de tripot pourrait fort
bien convenir} absorbe peut-être plus
d'argent qu'il ne va s'en perdre dans
les maisons qui ont le droit exclusif
d'attirer le public.

Mais il me semble que je ferais mal
connaître l'état actuel du jeu, si je
n'entrais séparément dans des détails
particuliers aux différens lieux où
l'on joue. Ils peuvent être rangés sous
trois dénominations : maisons publi%
ques, maisons particulières, tripots.
