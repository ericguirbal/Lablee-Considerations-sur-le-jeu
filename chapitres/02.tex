\chapter[Du Jeu]{DU JEU}

\folio{19}
\lettrine{L}{e} jeu, fruit de l'amour et du plaisir,
et aussi variable, me fut d'abord qu'un
exercice agréable ou salutaire de l'es%
prit ou du corps ; il n'est pas d'autre
chose pour beaucoup de personnes.

Si l'on fait attention à la manière
dont se développent les facultés intel%
lectuelles de l'homme ou de tout être
vivant, on verra que, presque dès sa
naissance, il joue avec des objects pu%
rement physiques ou avec des êtres
animés : s'il joue avec des objets pure%
ment physiques, il ne tarde pas à se
lasser de celui qui l'occupe ; il le quitte
pour en prendre un autre qu'il va quit%
\folio{20}
ter à son tour : s'il joue avec des êtres
animés, sur tout ceux de même nature
que la sienne, son action est plus vive,
sa gaîté plus bruyante, son attache%
ment plus prolongé. Ces mouvemens,
d'abord vagues et irréguliers, lorsque
l'intelligence se forme, et que la joie est
partagée, acquièrent insensiblement
de la règle et de la mesure. L'attrait du
jeu n'est encore que l'attrait du plaisir.
Le talent du joueur est l'habileté, la
ruse, l'adresse ou l'industrie. Le jeu
consiste à faire des sous, courir, s'éle%
ver, atteindre un but, prévenir ou re%
pousser une attaque, saisir prompte%
ment un objet idéal ou matériel ; enfin,
il se compose suivant le goût de celui
qui s'y livre, et offre presque toujours
une difficulté à vaincre. Un prix est
donné à celui qui l'a vaincue ; c'est une
\folio{21}
fleur, un fruit, un sourire, un baiser :
ce prix tente celui qui ne l'a point ob%
tenu ; celui qui en a remporté un pre%
mier veut en remporter un second ;
l'amour-propre est piqué ; l'émulation
naît et est excitée ; les défis se propo%
sent ; on n'aspire plus après des baga%
telles ; la nature du prix a changé ; celle
du jeu n'a plus la même simplicité : elle
se varie, elle se complique ; les inéga%
lités de force ou de talent, le doute,
les diverses interprétations font naître
les disputes ; l'adresse, l'industrie ins%
pirent du découragement ou de la dé%
fiance ; on leur associe une puissance
aveugle, le sort, qui agit tantôt avec
elles, tantôt sans elles : l'ignorant
s'étonne de son savoir ; le faible de sa
force ; l'infortuné de ses ressources ; le
téméraire de son triomphe. Le joueur,
\folio{22}
dans sa joie, croit que le sort a des
yeux, puisqu'il le favorise. Bientôt ce
tyran, interrogé de toutes parts, ras%
semble autour de lui la foule de ses
favoris, même celle de ses victimes.
Ses arrêts sont prompts, ses faveurs
faciles. L'ennui, la paresse, l'ambition
assiégent ses portes ; et les plus aima%
ble enchanteresses, l'espérance et
l'imagination, sont là, qui rassurent
les timides, flattent les orgueilleux,
consolent les mécontens et ramènent 
les fugitifs.

Déjà l'ardeur du jeu, celle des pas%
sions cupides que la plupart des hom%
mes éprouvent la première, fait naître
ou met les autres en mouvement, et
le monde habité est infecté d'un vice
d'autant plus funeste, d'autant plus
contagieux, qu'il s'embellit toujours
\folio{23}
du nom, de l'éclat et de la séduction
du plaisir.

Tels me paraissent être les commen%
cemens, les progrès, les variations de
ce qu'on appelle le jeu. Est-il donc né%
cessaire de fouiller dans les annales de
l'antiquité, pour découvrir son ori%
gine et étudier son histoire ? Si nous
consultons le livre de la nature, qui
nous est toujours ouvert, nous ne dou%
tons pas que les passions de l'homme
n'aient eu, ainsi que sa figure, dans
tous les lieux et dans tous les tems, à-%
peu-près les même traits, le même
caractère. Les différens climats, les
lois, les mœurs, les usages, mettent
peu de différence dans leur dévelop%
pement et dans les excès auxquels
elles conduisent. C'est un fleuve ra%
pide dont on peut prévenir l'entière
\folio{24}
corruption, mais qu'il est aussi diffi%
cile d'épurer que d'en arrêter le cours.

Le jeu est pur dans sa source. Mais
de quoi n'abuse-t-on pas ! Les excès
ont lieu jusque dans le travail.

« Si la fureur du jeu, dit Dussaulx,
est universelle en France, c'est parce
qu'une corruption générale est im%
punie ; c'est parce que l'amour des
richesses l'emporte sur l'honneur,
à mesure que les empires vieillis%
sent
\footnote{Grande vérité par laquelle s'expliquent
les désordres dont nous avons tant à gémir.}.

Le mal existe sans qu'on puisse
en accuser personne. »

Ce que j'ai dit du développement de
ce goût naturel qui nous porte vers le
jeu, a son application chez les peuples
sauvages comme chez les peuples civi%
\folio{25}
lisés. Le sauvage, en se mettant à la
merci du sort par des règles précises
et déterminées, en même tems qu'il 
prouve son ignorance, prouve qu'il a
fait un pas de plus vers la civilisation ;
et il est aisé de remarquer ici que la
passion du jeu réunit la sagacité à
l'aveuglement.

Presque partout le jeu a été la re%
présentation des combats. Les hom%
mes, naturellement imitateurs, et en%
clins à engager des luttes, se dédom%
magent, dans cette autre guerre, des
langueurs d'un honteux repos. On
peut régler leurs mouvemens ; mais
arrêtez-les dans leurs courses !

Ce besoin de jouer qui se manifeste
dès l'enfance, et fait contracter de
douces et fatales habitudes, a dans
la société bien d'autres effets que ceux
\folio{26}
qu'on leur attribue. Tous les jeux ne
sont pas ceux qui se pratiquent dans
les académies, dans les maisons de jeu ;
tous les joueurs ne sont pas désignés
par ce nom : il s'en trouve ailleurs, en
plus grand nombre, qui confient de
même au sort leurs plus grands inté%
rêts, et dont les calculs sont aussi faux,
les combinaisons aussi absurdes et les
espérances aussi chimériques. Une
ruine totale, la perte même de la vie,
est le résultat fréquent de ces autres
jeux : je veux parler de ce que, dans
les différens états de la vie sociale,
des hommes, égarés par leurs vœux
ou leurs désirs, exposent ou sacrifient
sans prudence, sans nécessité, ou sans
motifs raisonnables, dans la poursuite
des faveurs de la gloire, de l'amour
et de la fortune.

\folio{27}
Dans ces jeux, comme dans les pre%
miers, on est justifié par le succès ; et
l'opinion, toujours complice des vices
heureux, attribue à des calculs plus
médités, à une conduite plus sage, ce
qui n'est que l'effet d'un hasard favo%
rable ou d'une coupable audace, tan%
dis qu'on condamne et flétrit celui que
plus d'ordre et de modération n'a pas
garanti des revers du sort.


