\chapter{Réformes, Améliorations}

\folio{195}
\lettrine{E}{n} 1803, le nombre des maisons de
jeu s'acroissait par l'effet de la fa%
culté laissée à l'entreprise générale
d'en donner les permissions : plus elle
en accordait, plus elle grossissait la
masse des rétributions qu'elle exigeait.
Dans chaque quartier de Paris on
trouvait des roulettes, des trente-un,
des biribis, des passe-dix ; le même
quartier, lorsqu'il était très-habité,
possédait plusieurs jeux : c'étaient au%
tant de pièges tendus par une énorme
avidité, non-seulement à l'ignorance
et à la faiblesse, mais au malheur et
au besoin qui, à la suite de calamités
\folio{196}
publiques, devaient du moins trouver
dans les c{\oe}urs, si la piété était stérile,
des ménagemens et du respect. La
crédule espérance était horriblement
imposée : on ne pouvait échapper aux
invitations qui se faisaient, par car%
tes, de venir à un bal, à un festin,
dans une maison où l'on avait établi
une roulette ou un trente-un. Le
bruit de gains énormes faits, disait-%
on, par des joueurs prudens, se ré-%
pandait avec adresse : hommes, fem%
mes allaient se pressser autour d'un ta%
pis vert, dans des salons étroits et
obscurs, à qui des permissions n'ô%
taient pas l'air hideux des plus vils
tripots. Le mal faisait des progrès ef%
frayans et s'était déjà emparé pres%
qu'entièrement de la classe ouvrière.
La partie la plus éclairée du public
\folio{197}
était d'autant plus indignée qu'on n'i%
gnorait pas que l'entreprise des jeux
faisait des bénéfices considérables.

Enfin une grande réforme eut lieu.
Les journaux s'empressèrent de l'an%
noncer : tous les c{\oe}urs honnêtes y ap%
plaudirent. C'est aux vives réclama%
tions de M.~Davelouis, devenu depuis
administrateur des jeux, qu'on en a 
été redevable.

« Dix à douze millions de bénéfice
annuel partagés entre quelques in%
dividus ! et rien, absolument rien pour
la classe indigente ! \ldots

M.~Davelouis ne fit pas là-dessus
d'inutiles reflexions : son mémoire pa%
rut ; il était clair, précis, d'une sim%
plicité énergique.

On y trouvait cette phrase remar%
quable.
\folio{198}
« Certes un gouvernement révo%
lutionnaire qui voudrait, comme
on l'a vu, envahir la fortune des
particuliers, au lieu d'imposer des
emprunts forcés, ne pourrait mieux
faire que d'établir dans chaque rue
une maison de jeu, et de donner
des séances permanentes : il finirait
par attirer à lui la fortune de la
moitié de la population. »

« On lisait encore dans ce mémoire.
S'il est des m{\oe}urs sur lesquelles
gémit la raison, et que sa puissance
ne peut tout au plus qu'affaiblir,
pourquoi ne pas appliquer de fu%
nestes mais énormes produits au 
soulagement des malheureux ? »

Depuis ce tems peu de changemens
ont eu lieu dans l'administration des
jeux.

\folio{199}
On n'a conservé dans Paris que cinq
à six maisons, outre celles qui sont
ouvertes au Palais royal.

La maison la plus remarquable est
celle qui se tient sur le boulvard
Poissonière, et qu'on appelle le \emph{Grand
Salon}. Là, ne vont en général que des
hommes distingués par leur rang ou
leur fortune ; on n'y joue guères que
de l'or. Toutes sortes de rafraîchisse%
mens y sont servis. On peut y passer
la nuit.

Au \no.~113 du Palais royal, outre
les roulettes, un biribi est établi. On
y reçoit de la petite monnaie.

Au \no.~18 on fait jouer la nuit à la
roulette, au trente-un, au creps. On
y donne le bal, là des filles publiques
sont admises.

Un jeune homme s'étant suicidé il
\folio{200}
y a quelques mois, en sortant d'une
des maisons du Palais royal, il a été
décidé qu'avant l'âge de 21 à 22~ans,
on n'aurait point entrée dans les mai%
sons de jeu.

Nous vivons présentement sous un
règne où l'on peut espérer, dans les
différentes parties de l'administration
publique, toutes les améliorations
qu'il est possible d'opérer ; mais notre
position est telle qu'on ne peut, sans
une extrême injustice, demander que
toutes les réformes nécessaires se fas%
sent à-la-fois. Les hommes éclairés qui
veulent le bien, et qui croyent que
leurs vues pourraient y coopérer, ne
doivent les proposer qu'après les avoir
muries par la réflexion ; en les propo%
sant, ils ne doivent vouloir que se%
conder les efforts des magistrats, par
\folio{201}
lesquels seuls ces vues pourraient être
réalisées ; ils ne cherchent point à vio%
lenter l'opinion, ils la disposent par
un langage sage et modéré en fa%
veur de leurs plans, et attendent que
les tems soient favorables à leur exé%
cution.

Si l'on ne peut, par des lois, enchaî%
ner la fureur du jeu, s'il est plus juste
et plus sûr d'attaquer cette passion
dans son principe, c'est-à-dire, dans
l'esprit humain, et de la garantir des
pièges tendus par une coupable avi%
dité pour la porter aux plus grands
excès, on aura servi à-la-fois la so%
ciété et les joueurs, en mettant en
harmonie et en mouvement les diffé%
rens moyens propres à faire arriver à
ce double but.

J'ai dit au commencement de cet
\folio{202}
ouvrage que les moyens de réforme
proposés par le moraliste Dussaulx
me paraissaient ou insuffisans, ou
impraticables. C'est en vain qu'on
conseille à des hommes faits d'acqué%
rir des lumières, des vertus, du cou%
rage que leurs préjugés, leurs habi
tudes et la nature leur refusent ; c'est
aussi en vain qu'on conseille eux dé%
positaires du pouvoir de prohiber, de
poursuivre, de punir les excès qui
sont dans l'esprit et les m{\oe}urs de la
plis grand partie des hommes qu'ils
gouvernent ; il manquera toujours,
d'an et d'autre côté, la grande puis%
sance de l'exécution. Que l'opinion
seconde l'autorité ! que l'autorité se%
conde l'opinion ! blessez l'amour-%
propre des joueurs, facile à s'irriter,
en leur démontrant leur sottise ; éclai%
\folio{203}
rez-les en leur rendant sensible la
preuve de leurs faux calculs, le ridi%
cule de leurs espérances ; poursuivez-%
les par l'image des dangers auxquels
ils s'exposent, et des malheurs qui les 
attendent ; frappez leur esprit et leur
imagination dans tous les sens ; vous
avez affaire à des hommes légers et
frivoles, plutôt qu'à des hommes à
réflexion et à caractère.

C'est ici que les écrivains ont un
grand devoir à remplir ; et quel plus
bel emploi pourraient-ils faire de
leurs talens, que d'attaquer les er%
reurs les plus funestes, les vices les
plus désastreux ?

Si de son côté l'autorité joint son
influence à celle des lumières, s'ils
agissent de concert, je conçois que
cette horrible puissance du jeu puisse
\folio{204}
être affaiblie ; mais le besoin de cette
réforme fait sentir plus vivement ce%
lui d'une réforme des m{\oe}urs, à la%
quelle l'ordre public est si fortement
intéressé, et qui ne me paraît point
impraticable.

Il ne m'est pas permis de dire ici
quelles institutions nous manquent.
Quand on reconnaîtrait la justesse 
de mes idées sur ces objets impor%
tans, on la reconnaîtrait en vain.
Ce n'est point que les dépositaires,
les agens actuels du pouvoir ne me
paraissent mériter de la confiance,
mais encore une fois je sais ce qu'il
faut céder à l'empire des circons%
tances.

Puisse-t-on du moins reconnaître
que les bonnes m{\oe}urs sont la plus
grande richesse d'un Etat, qu'on perd
\folio{205}
plus qu'on ne gagne en faisant aux
vices qui les outragent moyennant
de plus fortes contributions pécu%
niaires, plus de concessions que celles
qu'on ne pourrait leur refuser sans
danger, et qu'il est des actes d'admi%
nistration d'une telle nature, que le
plus grand embarras des finances ne
pourrait les justifier.

Avant les changemens qui ont eu
lieu dans le ministère de la police gé%
nérale, je parlais à un ministre des
ressources qu'on trouverait dans les
produits légitimes d'une administra%
tion de tous les établissemens qui con%
cernent les amusemens et les différens
jeux publics, si cette administration
était centralisée et dans des mains à-la-%
fois pures et habiles ; je me complai%
sais dans la peinture des avantages qui
\folio{206}
résulteraient pour la classe indigente
ou pour celle qu'il est le plus difficile
de contenir de l'application de ses pro%
duits
\footnote{Jean Leclerc, à la suite de ses \emph{Réflexions sur ce
  qu'on appelle bonheur et malheur à la loterie}, dit :
  « Il y aura toujours des joueurs conjurés l'un contre
  l'autre, sans fruit pour la chose publique. Servons-%
  nous de leur manie pour ériger des temples, bâtir des
  hôpitaux, décorer des villes ».
  Voilà un emploi d'argent bien indiqué ; mais tâchons
d'en avoir le moins possible à appliquer à cet usage.} ;
enfin je jugeais qu'il était
possible de concilier les intérêts de la
morale avec ceux du trésor public ;
\emph{Bah ! bah ! la morale}, me dit le mi%
nistre avec impatience, \emph{il s'agit bien
de cela,} et il me tourna le dos, et de%
puis\ldots Mais je dois me renfermer
dans le silence. Il y aurait un beau
chapitre à faire sur la punition pres%
que toujours infligée aux écrivains
\folio{207}
qui osent plaider la cause de l'huma%
nité en présence des hommes à qui la
dépravation morale profite le plus.
Mais qu'un désordre social soit
causé afin que quelques individus fas%
sent en peu de tems une fortune co%
lossale !\ldots Encore une fois, il faut me
taire jusqu'au tems où je pourrai ser%
vir les vues de magistrats intègres,
honorés de l'estime du peuple et de la
confiance du Souverain, en ajoutant
aux vérités que je crois avoir rendu
sensibles, des vérités non moins im%
portantes. Je finis par exprimer le
v{\oe}u formé sans doute par tous les
gens de bien, pour que le Gouverne%
ment, dans sa restauration, ne perde
rien de son droit de placer près de
l'administration des jeux, telle qu'elle
puisse être, des surveillans qui l'ins%
\folio{208}
truiraient de ce que l'ordre public,
la morale, l'humanité et son propre
intérêt lui recommandent. Alors des
abus réformés, des améliorations
opérées, de sages précautions prises,
contribueraient à calmer les plaintes,
les murmures, et à amortir les fou%
dres que l'opinion lance sur ces an%
tres effrayans, où de viles passions
ne doivent être attirées qu'afin que
leur torrent ne puisse causer plus de
ravages \footnote{
  Ce Chapitre a été écrit vers la fin de l'an~1815.
}.
