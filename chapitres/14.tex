\chapter{Faut-il fermer les maisons de jeu ?}

\folio{184}
\lettrine{I}{l} n'est que trop évident, à l'époque
où j'écris \notemark, que la fureur du jeu
\notetext{
  Toujours en 1803.
}
s'est emparée de presque toutes les
classes de la société. J'ai fait voir
qu'elle agissait avec plus de violence
et de danger dans les maisons de par%
ticuliers, que dans les maisons publi%
ques. Le luxe, qui entre quelquefois
dans les vues d'une saine politique et
dans des convenances sociales, paraît
encore favoriser l'activité des jeux de
hasard, qui ont toujours été ses com%
pagnons inséparables. Telle est la dis%
position actuelle des esprits. Faut-il
\folio{185}
la changer ? faut-il la rompre tout-à
coup par des mesures de rigueur ? ou
faut-il, par des moyens adroits et
prudens, faire en sorte de la rendre
moins funeste et moins contagieuse ?

Dois-je le répéter ? le désordre du
jeu, produit par une vie oisive, le
besoin inquiet et l'aveugle cupidité,
est indestructible. Cependant on de%
mande qu'il soit détruit ; on le de%
mande à l'autorité, comme si elle n'é%
tait pas elle-même forcée de fléchir
devant l'énergie des passions humai%
nes ; et ce sont ceux dans qui cette ar%
deur de jeu est la plus effrénée, et
qui prennent le moins d'empire sur
eux-mêmes, qui accusent le plus hau%
tement les dépositaires même de l'au%
torité, de protéger leurs excès ! Li%
vrez-les à eux-mêmes, ils crieront ;
\folio{186}
contenez-les, ils crieront encore da%
vantage.

Je n'examinerai point ici comment
on pourrait amortir la fureur des
jeux de hasard, mais s'il convient de
les défendre, ou de fermer les maisons
publiques de jeu, et quels seraient les
résultats de cette mesure.

D'autres écrivains l'ont dit, c'est au
gouvernement à voir jusqu'où l'inté%
rêt de l'Etat ou des particuliers exige
qu'il défende le jeu ou le tolère.

Mais j'observerai que les bons gou%
vernemens profitent de l'expérience
de ceux qui les ont précédés : ils em%
ploient, le moins qu'il est possible,
des mesures prohibitives et inquisi%
toriales, qui entraînent presque tou%
jours plus d'incovéniens et de maux
que la tolérance ; il savent surtout
\folio{187}
que leur autorité est compromise et se
dégrade lorsqu'elle rend des lois qui 
ne s'exécutent pas.

L'inutilité ou l'insuffisance des lois
de rigueur contre les jeux, les te%
neurs de jeu et les joueurs, est bien
prouvée. On a vu ce qui était arrivé à
ce sujet.

On lit dans Blachstone, que j'ai dé%
jà cité : \frquote{Les rois ont fait de vains
efforts pour flétrir le jeu et dégoû%
ter les joueurs. Toutes les défenses
ont été éludées.}

\frquote{Les joueurs, dit ingénieusement
Dussaulx, sont plus agiles que la 
verge des lois qui les poursuit sans
les atteindre.

Les hommes de génie, les grands
écrivains, dit-il encore, n'ont point
traîté cette manière.} Ils n'ont pas
\folio{188}
cru sans doute devoir invoquer des
rigueurs inutiles.

A Rome on a fait des traités sur
tous les sujets ; on n'en a point fait
sur celui-ci. Fénélon, si courageux
lorsqu'il s'agissait de dire d'utiles vé%
rités, n'a osé blâmer le jeu, voyant
que Louis~XIV, qu'il aurait en vain
attaqué, avait pris le partie d'en faire
un attribut de sa grandeur.

{
  \fontsize{7.2pt}{10.3pt}\selectfont
  \settowidth{\versewidth}{Les plaisirs, et sur-tout ceux que le \emph{jeu} nous donne.}
  \begin{verse}[\versewidth]
    La défense est un charme ; on dit qu'elle assaisonne \\
    Les plaisirs, et sur-tout ceux que le \emph{jeu} nous donne.
  \end{verse}
}

Ces vers du bon La~Fontaine, dans
lesquels j'ai substitué le mot \emph{jeu} au
mot \emph{amour}, ont ici leur application.

Si quelquefois on est parvenu à
modérer la fureur du jeu, c'est qu'a%
lors les mœurs étaient plus pures, ou
le système qui tendait à les améliorer
les embrassait toutes à-la-fois.

\folio{189}
Mais lorsque d'anciennes habitudes
et des abus multipliés ont corrompu
les mœurs, n'y aurait-il pas encore
un grand danger pour la chose pu%
blique, si la puissance de l'autorité
se déployait pour leur subite ré%
forme ?

Le jeu, porté à l'excès, est un mal
qui se complique avec d'autres maux.
C'est un grand symptôme d'immora%
lité, qui suppose l'alliance de différens
vices. Ces désordre ayant différentes
causes, ce serait une grande erreur
ou une grande faute que d'entre%
prendre sa cure par un traitement
isolé et indépendant de ce qu'exige le
vice général. Ce sont les têtes de l'hy%
dre qu'on voudrait couper l'une après
l'autre ; il faut les abattre toutes à-%
la-fois.

\folio{190}
Effrayés par le débordement de la
passion du jeu, les gens de bien 
éclairés qui lui ont vu rompre toutes
les digues, désireront qu'elle se porte
sur des lieux où elle fera le moins de
ravages.

Les maisons publiques de jeu sont
pour ainsi dire le contre-poison du
mal que cause à la société l'excès du
jeu dans les maisons particulières, et
dans le sein même des familles. Qui
calculera les désordres que ce mal
secret causerait dans l'économie do%
mestique, si on n'attirait le venin sur
la partie du corps social où il a le
moins de prise et d'activité ?

Un mal qui se montre à découvert,
et n'a pas son siège dans l'intérieur,
est plus facile à traiter et à guérir.

Le mal du jeu ne pouvant s'extir%
\folio{191}
per, il faut lui enlever peu-à-peu sa
partie vénéneuse. Les calmans lui
enlèvent son acrimonie et son plus
grand danger. Fermer aujourd'hui
les maisons de jeu, comprimer le jeu
par des lois prohibitives et des actes
de rigueur, c'est faire rentrer une
dartre ; c'est, par des matières acides
et corrosives, répercuter un virus
dans l'intérieur, et infecter la masse
du sang social.

Qu'on considère l'état d'où nous
sortons, et celui dans lequel nous
sommes ! Prohibez le jeu, et avec les
joueurs, des brigands, qui étaient
pour ainsi dire hors de la société,
rentrent dans son sein ; et si cette ac%
tivité inquiète qui pousse vers le jeu
ne tendait plus vers cet objet, elle se
porterait sur d'autres, et causerait
de plus grand désastres ; et des mai%
\folio{192}
sons particulières s'empresseraient
d'offrir la réparation de ce qu'elles
appelleraient une grande atteinte
portée à la liberté, et le nombre des
tripots se multiplierait à l'infini : on
ferait circuler partout des invitations
plus séduisantes les unes que les au%
tres, et il y aurait pour l'étranger et
l'habitant des départemens, le plus
grand danger à se laisser entraîner
dans des maisons dont l'honnêteté
leur serait attestée ; et il arriverait ce
qui est déjà arrivé en France comme
en Angleterre, de petites loteries ; de
petits jeux prendraient des noms et
des formes qui les affranchiraient de
l'application de la loi et de l'action
de la justice ; et pour donner l'appli%
cation à la loi et l'action à la justice,
\folio{193}
il vous faudrait une armée d'inqui%
siteurs, qui aurait à ses ordres une
armée de sbires, lesquels auraient
le droit, sur des \emph{dénonciations} vraies
ou fausses, de s'introduire à toute
heure dans les maisons des particu%
liers ; et lorsqu'il y aurait erreur ou
méprise\ldots Je ne crois pas avoir be%
soin de peindre en entier le tableau
des effets que produirait la subite
et rigoureuse répression des jeux de
hasard.

Mais, me dira-t-on, des saisies ! des
des supplices ! des exemples ! \ldots Je
le sais, des hommes ignorent qu'il
est encore plus facile de prévenir
les excès que de les punir ; je sais
qu'il en est, même parmi ceux qui
ne sont délicats ni dans leurs senti%
mens, ni dans leur conduite, qui
\folio{194}
vont jusqu'à demander du sang pour
toutes les fautes, et qui, sans consi%
dérer que la plupart des joueurs sont
plus faibles, plus malheureux que
coupables, voudraient qu'on tuât
un homme pour lui apprendre à
vivre, ou au moins qu'on le mît à
nu pour l'empêcher de se ruiner.
L'esprit d'une bonne administration
dispensera toujours de répondre à
de pareils vœux et à de telles con%
ceptions.

Quant aux exemples donnés par
la fréquente punition des délits, ce
n'est pas ici le lieu d'examiner, s'ils
ne sont pas en général plus nuisibles
à l'humanité que profitables à l'ordre
social.

