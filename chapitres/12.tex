\chapter{Maisons particulières où l'on joue gros jeu}

\folio{164}
\lettrine{I}{l} y a à Paris une classe d'hommes
de qui les lumières et l'étude ne sont
pas le partage, et qui exagèrent les
modes, dans la crainte qu'on ne les
soupçonne de ne pas les connaître.
On pense bien que la plupart de ces
gens-là, sachant à peine d'où leur est
venue la fortune, après la renais%
sance des jeux de hasard, n'ont pas
été les derniers à s'y livrer. Ils y ont
joué d'abord par ton, ensuite par
cupidité, enfin par habitude.

Peu de modernes enrichis vont
dans les maisons publiques de jeu :
\folio{165}
ils croient avoir un rang à tenir et
une réputation à conserver ; mais
on en connaît plusieurs dont le jeu,
soit là, soit ailleurs, a vu finir la
métamorphose, et qui sont deve%
nus pauvres aussi rapidement qu'ils
étaient devenus riches.

J'ai dit comment s'étaient établies
des maisons particulières de jeu dans
les départemens : il s'en est formé et
s'en forme encore beaucoup dans
Paris du même genre ; mais celles-ci,
pour n'être point soumises aux re%
gards de la police, n'ouvrent point
leurs portes à tout le monde, et ne
font point distribuer d'adresses ni de
cartes d'invitation à un grand nom%
bre de personnes ; seulement l'ami y
mène son ami ; et ces amis, d'autant
plus attachés l'un à l'autre qu'ils ne
\folio{166}
se connaissent pas, se lient subite%
ment par de tels rapports, qu'ils n'ont
d'autre désir, d'autre but, d'autre 
soin que de s'enlever les uns aux
autres tout ce qu'ils possèdent.

On n'a point, dans ces maisons,
de grandes tables garnies de machines
de différentes formes, qui trahiraient
les intentions des maîtres, ou plutôt
des maîtresses, car ce sont le plus
souvent des dames qui sont à leur tête
ou en font les honneurs : il n'y a que
de simples tables de bouillotte, ou
d'autres qui servent au besoin pour
un vingt-un, un trente-un, un loto
à fortes mises, etc. Il vous serait
difficile d'y échapper aux différens
moyens de séduction qui vous envi%
ronnent. Les jeux à argent s'y con%
fondent parmi d'autres jeux ; souvent
\folio{167}
ils n'y paraissent qu'un amusement
accessoire : ils ne tardent pas à deve%
nir une grande occupation ; et les
cris de joie qui partent d'un salon
voisin accompagnent les cris de dé%
tresse des victimes qui tombent l'une
après l'autre dans un précipice dont
les bords n'en restent pas moins cou%
verts de fleurs.

Vous dites que ce ne sont pas là des
maisons honnêtes : le nom des maîtres
ne s'est-il pas rendu recommandable
par des emplois dans la robe, dans la
finance, dans le militaire ? N'avez-%
vous pas trouvé chez eux, comme on
vous l'avait annoncé, des gens distin%
gués par leurs places, leurs richesses
ou leurs talens ? Le goût, la décence,
la délicatesse n'y brillaient-ils pas
également dans les discours et dans
\folio{168}
la parure des femmes ? Ah ! il n'est
pas moins vrai que ce sont de ces
maisons honnêtes que sortent confu%
sément, comme Dussaulx l'a observé,
le parjure, la misère, l'opprobre, le
duel et la mort.

Les maisons publiques où la plu%
part des joueurs considérés dans le
monde, craignent d'être aperçus, fa%
vorisent certainement moins les ex%
cès du jeu que ces brillans rendez-%
vous de société, où, avec de pareils
goûts, on serait bien fâché de n'être
pas admis.

Il n'y a pas encore de long-tems, on
citait des pertes considérables faites
par des personnes connues, \emph{sur-tout
par des étrangers}, dans ces société
du bon ton.

Les maisons particulières où l'on
\folio{169}
joue gros jeu à Paris, sont de diffé%
rens genres : il y en a, comme des
maisons publiques de jeu, pour les
différentes classes de citoyens. Les
unes reçoivent tous les soirs, les au%
tres donnent un grand dîner un jour
fixe de la semaine. Après le dîner les
parties s'arrangent : ce ne sont pres%
que que des parties carrées, les sim%
ples jeux de commerce. Une bouil%
lotte cependant rassemble ceux qui
n'aiment pas le petit jeu. De ceux-là
quelques-uns, pour faire un double
emploi de leur tems, parient de fortes
sommes à la queue ou aux marqués
d'un piquet qui se joue à côté d'eux à
cinq sols la fiche.

A mesure que les petits jeux finis%
sent, les belles dames se rassemblent
autour de la bouillotte ; elle encou%
\folio{170}
ragent des yeux les joueurs de leur 
connaissance ; celui qui ayant devant
lui une forte somme d'argent, fait ou
tient le tout contre une masse à-peu-%
près égale, s'il gagne, est compli%
menté sur son bonheur, et s'il perd,
est consolé par le titre de beau joueur
qu'on ne peut lui contester.

On vante le sang-froid et la témé%
rité d'un joueur qui se ruine, comme
on vante le calme et le courage d'un
homme condamné qui marche au
supplice.

Il est nuit : les personnes raisonnables
se retirent peu-à-peu ; le nombre des
rentrans à la bouillotte a contraint d'en
former plusieurs tables. Cette dame,
qui au reversis a eu, pour une fiche
à deux sols, une contestation d'un
quart d'heure, s'est cavée de dix louis.

\folio{171}
La bouillotte ne se quitte pas aussi
promptement qu'un autre jeu ; d'ail%
leurs, des perdans trouveraient fort
mauvais qu'on les abandonnât de 
bonne heure. Le jeu se prolonge
donc dans la nuit : il s'échauffe ; les
caves se centuplent ; le tems n'a plus
d'heures ; le jour vient, et les dames
de la maison songent enfin qu'il se%
rait bon de prendre quelque repos.
Hélas ! il n'en est plus pour de nou%
velles victimes que vient de faire ce
jeu de société, qui, plus que tous les
jeux publics, a porté depuis quelque
tems dans les familles, la misère et
la désolation.

Oui, on assure que plus de banque%
routes, de duels, de suicides ont été
causés par ce jeu dans des maisons
particulières, que par tous les jeux
\folio{172}
de hasard dans les maisons publiques.
Dans plusieurs, la première cave est
de cinq louis. On assure que chez des
parvenus, on se cave le plus souvent
de quatre à cinq cents louis : on en
cite même où les caves ont été portées
à mille.

Ce n'est pas seulement à la ville
qu'on joue un jeu énorme chez des
particuliers ; pour être plus à l'aise,
et pour que les femmes ne soient pas
toujours sur les épaules des maris,
on se donne rendez-vous dans des
maisons de campagne, que des par%
venus et des enrichis appellent \emph{leurs
petites maisons}, à l'imitation des 
grands seigneurs et des financiers des
derniers règnes. Là, l'ivresse du vin
accroît l'ivresse du jeu ; là, on con%
vient de jouer jusqu'à extinction de
\folio{173}
bourse ; mais on va plus loin, on joue
jusqu'à épuisement de crédit ; car un
maître de maison , ou un des joueurs,
trouve souvent son compte à prêter
à celui qui est encore dans l'usage et
le pouvoir de rendre, ressource bien
fatale pour celui qui la possède ! Le 
crédit, qui dans le commerce a des
effets bienfaisans, a au jeu les effets
les plus désastreux. Mais des négo%
cians, des banquiers, des agens de
change, qui se garderaient bien de se
montrer dans des jeux publics, jouent
volontiers dans ces parties de cam%
pagne, qu'ils appellent \emph{des parties
fines.} Leurs conventions au jeu se
font comme celles de la bourse ; le
mot suffit.

C'est ainsi que certains, qui avaient
fait quelque tems grand bruit dans la
\folio{174}
banque ou le commerce, \emph{ont joué de
leur reste.}

On sait ce qui à la bouillotte peut
résulter d'une facile intelligence éta%
blie entre quelques joueurs, et s'il t
a des joueurs honnêtes et délicats, on
ne croira pas que c'est parmi ceux
qui jouent le jeu le plus considérable
qu'ils se trouvent le plus.

\emph{Le flambeau}, à la bouillotte, est
d'un produit si abondant, qu'on ne
doit pas être étonné que tant de mai%
sons particulières cherchent à en éta%
blir une. On ne découvre pas le secret
de sa spéculation : ce sont ses an%
ciennes et ses nouvelles connaissances
qu'on est bien aise de réunir de tems
en tems. Les profits, au surplus, ne re%
gardent, dit-on, que les domestiques.

Il y a dans ces spéculations, comme
au jeu, bonheur et malheur.
