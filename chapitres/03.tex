\chapter[Des joueurs]{Des Joueurs}

\folio{28}
\lettrine{L}{es} joueurs n'ont pas un caractère
unique, déterminé, susceptible d'être
traité avec les mêmes procédés, ou
combattu avec les mêmes armes. Leur
caractère a des nuances extrêmement
variées : en cela il diffèrent des ava%
res, des envieux, des jaloux, des ivro%
gnes, des débauchés, dont la passion
ou le vice a un principe connu, ou
commun à presque tous.

L'ambition, l'orgueil, la cupidité,
l'ennui, le besoin, font des joueurs de
différents espèces.

On joue par caprice ou par système,
\folio{29}
par occasion ou par habitude, aux jeux
de hasard ou de commerce.

Les différents manières de jouer
tiennent à la différence des motifs, de
l'esprit, du tempérament et de la po%
sition des joueurs.

A voir de l'audace et le sang-froid des
uns, la timidité et la turbulence des
autres : on juge aisément si les mêmes
leçons ou les mêmes mesures de ré%
pression leur conviennent.

Tel n'a joué que quelques jours, et
a joué un jeu considérable ; tel autre
ne peut se priver du jeu un seul jour,
qui ne joue qu'un jeu modéré. A qui le
nom de joueur convient-il davantage ?

Il me semble du moins qu'il n'est
pas juste de comprendre sous la même
dénomination le goût et la passion, le
caprice et l'habitude du jeu.

\folio{30}
Je me demande si ce sont ceux qui ai%
ment le jeu, ou ceux qui ne l'aiment
pas, qui dans la société ont exception ?

Le nombre des joueurs honteux est
plus considérable qu'on ne le croit.

Je rencontre un homme de ma con%
naissance peu favorisé de la fortune.
On parle des jeux de hasard : « C'est
une fureur, dit-il, et on ne songe
pas à y mettre un frein !» Le soir,
je le trouve dans une maison particu%
lière. Il jouait à la bouillotte ; la cave
était de cinq louis.

Il y a de la différence entre le ca%
ractère et les procédés des joueurs aux
jeux de commerce, et ceux des joueurs
aux jeux de hasard.

Il convient aussi de distinguer les 
joueurs d'habitude et les joueurs de
profession. Si on ne sait poser une
\folio{31}
ligne de séparation entre les diffé%
rentes espèces de joueurs, on est ex%
posé à commettre des erreurs et des
injustices.

Des joueurs aux jeux de commerce
peuvent tirer parti de leur expérience,
de leur savoir, de leur finesse ; d'au%
tres sont trop légers, trop distraits,
ou d'une ignorance trop présomp%
tueuse pour ne pas donner à ceux-ci
beaucoup d'avantages. Cependant,
comme le hasard y une part plus ou
moins grande, les plus habiles y sont
quelquefois maltraités ; mais ils ne
tardent pas à reprendre leur supério%
rité. Voilà pourquoi on cite des joueurs
presque constamment heureux ; et
ceux qui donnent au goût justifié de
ces joueurs un aliment facile, n'osent
avouer et ne s'avouent pas eux-mê%
\folio{32}
mes leur ignorance ou leur faiblesse.
Mais imaginez qu'une perfide adresse,
qu'une coupable industrie vienne en%
core seconder l'art et l'expérience,
vous connaîtrez mieux les motifs pour
lesquels certains hommes font du jeu
leur unique occupation. Aussi ce n'est
pas aux jeux de hasard que si livrent
ces joueurs, dont l'honnête Dussaulx,
a fait, avec raison, un épouvantail ;
car je ne parle point encore de ces
banquiers de société, à qui l'art de
mettre en défaut les regards les plus
attentifs, réussit d'autant plus, que
là on s'en défie le moins : là on crain%
drait, par une accusation directe, de
paraître impudent ou grossier ; et il se
trouverait difficilement quelqu'un qui
oserait vérifier et constater le délit.

On conçoit que les joueurs de pro%
\folio{33}
fession ont pour la plupart les doigts
agiles, la tête froide, et cette absence
de passions favorable aux calculs. Mal%
heureusement ils ne prennent pas le
titre de joueurs, et ils souffriraient
impatiemment qu'on le leur donnât.
Ils sont toujours surchargés d'autres
affaires, d'autres soins, et le jeu est
le moindre sujet de leur conversation.
De qui les favorise surtout, c'est qu'il
est aisé de les confondre avec ceux qui
ne sont que des joueurs d'habitude.
Ceux-ci ont besoin de jouer, comme le
besoin de manger et de boire : l'heure
du jeu est marquée pour eux comme
celle de leur repas, de leur sommeil,
de leur dévotion. Il leur arrive sou%
vent de bâiller, de s'assoupir au jeu.
Ils jouent à un jeu de hasard comme
à tout autre, et ne sont étonnés que
\folio{34}
du coup qui les ruine sans ressource ;
c'est leur maison qui vient de s'é%
crouler. S'ils survivent à cet accident,
ils passeront le reste de leur vie à ra%
conter ce qu'il avait d'extraordinaire.

Les grands joueurs, j'entends ceux
qui ont la fureur du jeu, sont en gé%
néral des hommes à caractère et à
grandes passions, c'est-à-dire, qu'ils
ont le sang vif, la tête sulfureuse,
l'âme brulante, l'imagination exaltée,
la sensibilité profonde, et en cela, je
ne diffère pas autant qu'on le croirait
d'opinion avec Dussaulx, qui, tout
en refusant cette qualité aux joueurs,
prouve, par beaucoup de traits sail%
lans sur leur compte, qu'ils la pos%
sèdent à un degré éminent. Il est vrai
qu'un sort très-heureux ou très-mal%
heureux paraît rendre, même rend
\folio{35}
quelquefois les hommes insensibles ;
mais l'état d'enivrement ou d'apathie
ne dure pas long-tems, et la nature
ne tarde pas à reprendre son cours et
son énergie.

C'est de la classe de ces joueurs dont
je viens de parler, que sort la plus
grande partie de ceux que la ruine et
le désespoir portent au suicide.

Parmi les hommes de mérite que
présente Dussaulx comme ayant été
de grands joueurs, j'ai déjà nommé
Caton, Henri~IV, Montaigne, Des%
cartes, Collardeau, et lui-même, qui
en a fait l'aveu : je dois ajouter les
noms célèbres de Duguesclin, le Gui%
de, Rotrou, Voiture, Cardan, Halli%
fax, Schafsterbury, même du méde%
cin allemant Ponchasius Justus, aussi
auteur d'un livre contre le jeu.

\folio{36}
Qu'on ne s'y méprenne pas, quand
je dis que les grands joueurs sont des
hommes à passions, je ne dis pas que
les hommes à passions éprouvent né%
cessairement celle du jeu. Parmi ces
être peu communs, il en est qui ont
parcouru le cercle entier des passions ;
il en est dont une seule a consommé
la vie entière.

Les états sui laissent le plus de loisir,
sont ceux qui fournissent le plus grand
nombre de joueurs. C'est parmi les
ecclésiastiques que j'ai pris le goût du
jeu ; c'est parmi les militaires que je
l'ai vu régner avec le plus d'éclat. En
jouant aux jeux de hasard, ils ne sor%
tent pour ainsi dire, ni de leur profes%
sion, ni de leurs habitudes.

Mais quel long chapter il y aurait
à faire sur les joueurs ! …


