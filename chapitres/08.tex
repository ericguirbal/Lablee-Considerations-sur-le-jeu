\chapter
  [Des avantages et des dangers du Jeu : Bonheur et malheur]
  {Des avantages et des dangers du jeu : bonheur et malheur}

\folio{109}
\lettrine{Q}{uoi} qu'en dise une philosophie trop
austère, le jeu n'est pas sans quelques
avantages pour ceux qui n'y portent
point de passions désordonnées ; et si
nos mœurs ne nous avaient accoutu%
més à ne voir que des excès dans les
choses qui peuvent nous procurer des
jouissances, nous reconnaîtrions que
les personnées douées d'un caractère
de modération et capables de se sous%
traires à de vicieuses habitudes, peu%
vent mettre, même les jeux de ha%
sard, au nombre de leurs plaisirs.
Ces jeux, il est vrai, disent peu de
choses à l'esprit, mais quelquefois on
\folio{110}
leur est redevable d'une distraction
dont on a besoin, et qu'on ne cherche%
rait pas dans les jeux de commerce,
lesquels, comme les autres, ne tien%
nent pas l'âme dans cette légère émo%
tion que cause l'incertitude de l'évè%
nement ; lesquels aussi exigent, pour
qu'on soit en état de s'y défendre,
une science ou une expérience qu'on 
n'a point acquise. Les premiers ont
encore sur ceux-ci le grand avantage
de pouvoir être pris et quittés à vo%
lonté ; ils donnent aux sens une agita%
tion salutaire, pour ceux qui languis%
sent dans de trop fortunés loisirs. En%
fin, ce n'est guère qu'en exposant à
ces jeux ce dont on peut se passer, ou
ce à quoi on est peut attaché, qu'il est
possible que ceux qui sont dénués de
talens, ou d'industrie, ou d'amour du
\folio{111}
travail, obtiennent légitimement les
moyens d'accroître leurs jouissances.
Qu'on ne perde point de vue que je
ne pardonne pas aux joueurs de mettre
au jeu leur nécessaire ou celui de leur
famille. Je conviens qu'on ferait beau%
coup mieux d'appliquer son superflu
à des actes de bienfaisance, qui sont
pour certaines ames des objects de né%
cessité; mais je dois considérer ici les
hommes tels qu'ils sont en général,
tels qu'on n'empêchera pas qu'ils
soient, et non tels qu'ils devraient être.
C'est trop accorder que de trop exiger
en morale : la difficulté de faire tout
le bien qui est demandé, semble dis%
penser de faire aucun bien : c'est ainsi
qu'on peut entendre la maxime sou%
vent citée, malgré sa fausseté : \emph{le
mieux est l'ennemi du bien.}

\folio{112}
Cette observation me porte à m'ex%
pliquer sur la légimité que j'attribue
aux gains des joueurs, et que Dussaulx
leur conteste, plutôt d'après des con%
sidérations, que d'après un principe
juste. Le contrat du jeu peut n'être
pas moral dans son esprit, et cepen%
dant être légitime dans ses effets : il
peut n'être pas légitime relativement
à la loi, et cependant l'être relative%
ment aux contractans. S'il n'avait pas
ce caractère de légitimité, il ne serait
pas obligatoire ; et alors il y aurait au
moins deux maux pour un. Ce qui lui
donne ce caractère est l'égalité des
mises, des risques et des espérances.
Si aux jeux de hasard, tels qu'ils se
pratiquent aujourd'hui, l'égalité pa%
raît être rompue, c'est en faveur du
banquier ; mais on ne peut dire qu'elle
\folio{113}
le soit réellement, puisque son avan
tage reconnu et consenti par le joueur,
n'est regardé que comme le paiement
des travaux et des dépenses de l'admi%
nistration : quant au joueur, qui a eu
plus de chances contre lui qu'il n'en
a eues pour lui, il est au moins singu%
lier qu'on l'accuse de recueillir injus%
tement ce que le sort lui accorde, et
qu'on compare ses bénéfices à des ra%
pines et aux pillages, tels que ceux
qu'excercent les Arabes sur les cara%
vanes. Ces bénéfices sont, dit-on, la
dépouille de pères de famille, réduits
au désespoir ; cet argent qu'on vous
donne, sort des mains de celui qui l'a
volé : je le veux ; mais cette dépouille
et cet argent volés, vous ne les recevez
pas des joueurs malheureux ou fri%
pons ; ils sont devenus la propriété
\folio{114}
des banquiers, les seuls avec lesquels
vous ayez contracté : l'unique effet
de votre gain ne se fait sentir qu'à la
banque, et quel argent aurait-on si
on se faisait scrupule de recevoir ce%
lui qu'on croit avoir passé par des
mains impures ?

Voici les erreurs et les exagérations
dans lesquelles on est entraîné par
l'amour du bien. Je ne reconnais pas
plus la justesse de l'idée de Dussaulx,
lorsque contestant au jeu ses avan%
tages, il dit : \frquote{Quand vous jouez la
moitié de votre bien, si vous ga%
gnez, votre capital n'augmente que
d'un tiers ; si vous perdez, il dé%
croît de moitié}. Qui ne serait pas
étonné de voir que ce savant Dus%
saulx ne connaît rien de plus fort
contre la séduction du jeu que cet
\folio{115}
argument qui n'est que spécieux ?
Celui qui joue la moitié de son bien,
s'il gagne, l'augmente réellement de
la moitié et non d'un tiers : à la vé%
rité, lorsque cette valeur de la moitié
de son bien y est jointe, elle n'est
plus comptée que comme le tiers de
la totalité ; mais il n'est pas moins
vrai que, par l'effet de son gain, ses
jouissances sont augmentées de moi%
tié, comme elles seraient diminuées
de moitié s'il eût perdu. Il n'y a donc
là qu'un jeu de mots ou une fausse
image propre à donner une fausse idée
à ceux qui lisent superficiellement et
non cette inégalité de condition dont
Dussaulx a voulu frapper les sens.

S'agit-il, au surplus, d'établir des
proportions entre les privations et
les jouissances ? Je dirai qu'il est pos%
sible que celui qui hasarde la moitié
de sa fortune, ait assez de philosophie
pour la perdre sans peine, mai qu'il
peut trouver dans l'augmentation
d'un tiers de son bien, la jouissance
de ce qui était le terme de ses vœux.

J'ai dit les avantages que paraissent
avoir les jeux de hasard, qui devraient
être abandonnées aux grands et aux
riches, pour lesquels il semblent faits ;
dirai-je leurs dangers ? On a déjà pu
s'en former une idée, par les désordres
dont j'ai précédemment esquissé le
tableau. Peu de personnes sont ca%
pables de se maintenir dans l'esprit de
modération et dans les règles de con%
duite que j'ai recommendées : le nom%
bre de celles qui n'ont jamais su résis%
ter à l'attrait de ces jeux, après l'avoir
une fois connu, est considérable.
\folio{116}
Les désirs immodérés, les vaines es%
pérances, les trompeuses illusions,
abandonnent difficilement les joueurs
dont ils se sont emparés : ils les pour%
suivent jusques dans leurs songes, les
dégoütent insensiblement des plaisirs
simples et honnêtes, et leur rendent
bientôt l'acquit de leurs devoirs et
toutes leurs occupations ennuyeux
et insupportables. Quelques faibles
que soient les pertes qu'on ait faites,
on est préssé de les réparer, et l'on se
flatte de s'en tenir là lorsqu'elles se%
ront réparées ; mais, ou l'on tâche
avec plus d'ardeur de ressaisir ce
qu'on a perdu de nouveau, ou l'on
trouve agréable et facile d'ajouter
d'autres gains à ceux qu'on a faits.
Ainsi le tems se perd, l'esprit se con%
sume, le cœur s'endurcit ou se cor%
\folio{118}
rompt, le sang s'aigrit, la santé s'al%
tère et la fortune s'épuise dans des ha%
bitudes qu'on a bientôt contractées,
et auxquelles on ne renonce point
sans une force plus qu'humaine.

Voilà, non une peinture exagérée
des dangers du jeu, mais des vérités
qu'atteste l'expérience de presque
tous les joueurs, et qui doivent ren%
dre bien faibles à des yeux éclairés,
les rares avantages qu'on peut trouver
dans les jeux de hasard.

Quelques-uns observent sérieuse%
ment qu'il faut laisser ces jeux aux
personnes à tout réussit, parce
qu'elles sont nées heureuses, et ils
vous citeront beaucoup d'exemples
d'un bonheur constant ; d'autres vous
disent qu'ils ne sont point étonnés
de ce qu'ils perdent toujours, parce
\folio{119}
qu'ils sont nés malheureux, et ils n'en
jouent pas moins. Pour se faire en%
tendre de ceux qui tiennent de bonne
foi un pareil langage, dicté souvent
par l'humeur ou l'impatience, il fau%
drait leur donner quelques facultés
intellectuelles que la nature leur a
refusées, ou la première instruction
que reçoit le jeune âge ; pour moi qui
n'ai pas entrepris cette tâche, je me
contenterai de leur dire qu'ils pren%
nent des effets pour des causes ; qu'il
n'y a de bonheur que pour ceux qui 
ont gagné, de malheur que pour ceux
qui ont perdu ; que le gain et la perte
viennent de causes que personne ne
peut ni prévoir, ni éviter, et qu'il ne
manque rien à l'aveuglement de cette
puissance à la disposition de laquelle
les joueurs mettent leurs destinées.
