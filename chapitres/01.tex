\chapter
  [De l'ouvrage intitulé : \emph{de la Passion du Jeu,} par Dussaulx]
  {De l'ouvrage intitulé : \emph{De la Passion du Jeu,} par \bsc{Dussaulx}}

\folio{1}
\lettrine{L}{e} bon, l'honnête Dussaulx a fait
sur la passion du jeu un traité histo%
rique et moral, qui est ce que nous
avons de plus complet, de mieux
pensé et de mieux écrit sur cette ma%
tière. On y remarque une érudition
\folio{2}
facile, des anecdotes curieuses et ins%
tructives, des réflexions originales,
piquantes et quelquefois profondes,
une sorte de verve poétique, et des
vues portant le cachet d'un bon es%
prit et d'un bon cœur : on partage
la juste indignation qu'excitent dans
l'âme de l'auteur les excès du jeu et
le crime de ceux qui les favorisent ;
mais souvent l'on sourit à sa con%
fiante bonhomie. A-t-il pu croire que
l'énergie des passions cupides céde%
rait à des moralités et à des citations ?
En le lisant avec attention, on doute
qu'il s'en soit flatté ; mais si on n'y
trouvait de ces pensées fortes, de ces
traits qui caractérisent la vraie sen%
sibilité et le besoin de la répandre,
on serait tenté de penser que l'au%
teur a voulu faire plutôt un ouvrage
\folio{3}
savant, curieux, orné des fleurs de
l'éloquence, qu'un ouvrage dont on
pût retirer beaucoup de fruit. On le
voit plus appliqué à peindre le mal
qu'à en indiquer le remède. Lui-%
même il parle de l'inutilité des lois et
des efforts des gouvernemens contre
cette fureur aveugle ; il dit et il prouve
que le jeu a dans tous les lieux et dans
tous les tems subjugué l'esprit des
hommes de toutes les classes. \frquote{Par%
courez la terre depuis le Japon
jusqu'à l'extrémité du nouveau
monde, quels que soient le culte,
les lois et les opinions, vous trouve%
rez des joueurs dans les climats
glacés et dans les climats brûlans}.

Voilà ce que dit Dussaulx ; et pour
démontrer par les faits que cette épi%
démie universelle est indestructible,
\folio{4}
je n'aurais besoin que de reproduire
ceux qu'il rapporte.

Je citerai plus d'une fois cet au%
teur, le seul qui chez nous se soit fait
entendre sur cette matière. N'ayant
pour but que d'offrir la vérité, je ne
dois rien négliger de ce qui me semble
propre à la faire connaître. C'est dans
cet esprit et dans cette obligation que
je relèverai aussi les défauts et les er%
reurs qui m'ont frappé dans l'ouvrage
dont il s'agit.

Dussaulx prononce d'abord trop
fortement son intention de peindre
les joueurs et leur manie avec les
couleurs les plus noires ; il commence
par les dévouer à la haine et à la pros%
cription ; et à ce sujet il s'exprime
ainsi : \frquote{Si les écrivains ont montré
les côtés séduisans du jeu, ils sont
\folio{5}
des corrupteurs ; s'ils n'en ont ex%
primé que la difformité, ils sont
les vrais amis de l'humanité}.

Aussi représente-t-il souvent les
joueurs moins tels qu'ils sont, que
tels qu'il faut qu'ils soient pour pa%
raître odieux. Voilà bien ce qui con%
vient pour faire briller le talent d'un
écrivain, pour qu'il puisse donner à
ses sentimens un développement éner%
gique ; mais ce n'est pas la meilleure
règle d'instruction. Sans doute c'est
en offrant aux hommes le flambeau
de la vérité, c'est en les éclairant qu'on
sert le mieux leurs intérêts. Si vous ne
leur montrez qu'un côté des choses,
et qu'ils viennent à découvrir celui
que vous voulez leur cacher \paren{ici ils
le découvriront ; ils l'ont même déjà
découvert, car sans cela vous n'au%
\folio{6}%
riez pas de leçons à leur faire}, ils
seront en droit de se plaindre de ce
que vous avez voulu les tromper ; ils
vous accuseront de mauvaise foi, et
n'auront plus de confiance dans ce
que vous leur direz.

Les écrivains moraux ont besoin,
pour fixer nos idées, d'un grand ca%
ractère d'impartialité et de désinté%
ressement. Dussaulx, par la sorte d'en%
gagement qu'il a pris dès le commen%
cement de son livre, manque une
partie des effets qu'il pouvait pro%
duire. On prévient, on devine sa pen%
sée ; on va jusqu'à suspecter la vérité
de ses tableaux ; ce n'est plus qu'un
avocat qui, dans un procès impor%
tant, rassemble tout ce qui lui paraît
favorable à sa cause, crie contre ses
adversaires, exagère ses accusations,
\folio{7}
ses reproches. On l'écoute ; il inté%
resse ; mais pour fixer son opinion,
on attend que ses adversaires lui aient
répondu.

Ainsi, Dussaulx, en voulant trop
prouver les dangers du jeu, a paru
mettre en question une vérité géné%
ralement sentie.

Pour prendre plus d'avantage sur
les joueurs, il en a trop simplifié le
caractère, et il a commis évidemment
une erreur, en considérant moins le
jeu comme une de ces passions inhé%
rentes, pour ainsi dire, à la faiblesse
humaine, et qu'il est plus facile d'é%
nerver, de diriger, que de détruire,
qu'en le considérant comme un vice
absolu, déterminé, sur lequel la loi
pouvait avoir une action directe, ou
auquel on pouvait appliquer, comme
\folio{8}
à des maux connus, des remèdes gé%
néraux. Je ferai voir que le caractère
du Joueur, extremement composé,
tient à différentes causes qui le mo%
difient et qu'il faudrait connaître,
pour être en état d'employer à la gué%
rison du mal des remèdes particuliers.

Mais comment concilier les diffé%
rentes idées que Dussaulx donne du
jeu et des joueurs ?

Il dit : \frquote{Il s'agit ici d'un vice pur
et sans mélange. Quel joueur a le
droit de s'estimer ? Un joueur ! ce
titre est une insulte}.

Et ailleurs : \frquote{La manie du jeu roule
sur trois pivots, la sottise, la fu%
reur et la fourberie}.

Et ailleurs : \frquote{On citerait moins de
joueurs sensibles que de bourreaux
compâtissans}.

\folio{9}
Et ailleurs : \frquote{Les joueurs manquent
de sensibilité comme de probité.}

Enfin, avec Aristote, il refuse aux
joueurs toutes les qualités du cœur.

Cependant il met au rang des plus
grands joueurs, les hommes doués
de plus d'imagination ; et parmi ces
grands joueurs, il cite d'excellens
hommes, tels que Caton, Henri~IV,
Montaigne, Descartes, Collardeau et
lui-même.

Il dit aussi que la fureur du jeu,
par un alliage monstrueux, se joint
quelquefois à de grands talens et à
de grandes vertus.

Et il rend encore moins effrayante
la laideur de ses portraits, en obser%
vant que l'ennui fait plus de joueurs
que la cupidité ; que le goût du jeu
\folio{10}
est quelquefois moins un symptôme
de cupidité que d'ambition.

Toutes ces contradictions sont-%
elles assez évidentes ? II est vrai que,
pour se mettre à l'abri du reproche
d'avoir désigné les joueurs par d'o%
dieuses qualifications, il applique,
vers la fin de son livre, ce qu'il en a
dit aux joueurs de profession ; mais
cette explication prudente et tardive
est loin d'être satisfaisante. Dussaulx
n'ignorait pas que les joueurs de pro%
fession n'ont pas de passions, n'ont
pas même de caractère, et que cette
classe est trop peu nombreuse, trop
peu importante, trop peu susceptible
d'impressions morales, pour qu'on
doive prendre la peine de faire pour
elle un gros livre.

Dussaulx, revenant au caractère de
\folio{11}
fourberie qu'il attribue injustement
aux joueurs en général, dit : \frquote{Vous
trouverez des joueurs suspects dans
tous les rangs ; parmi les gens de
lettres vous ne verrez que des vic%
times résignées aux caprices du
sort.}

J'observe d'abord que les joueurs
les plus nombreux, ceux du moins
qu'il faut le plus s'attacher à guérir
de leur frénésie, sont ceux qui jouent
aux jeux de hasard : or, d'après les
précautions prises ordinairement par
les banques, un joueur assez adroit
pour être fructueusement un fripon,
est une exception très-rare ; et encore
une fois ce n'est pas pour ceux qui
vont au jeu, avec l'intention d'y vo%
ler, qu'on fait des traités de morale :
ensuite si, par ces mots \emph{victimes ré%
\folio{12}
signées}, Dussaulx a entendu incapa%
bles de fourberie, je crois que d'autres
rangs ont également cet avantage ;
et s'il a entendu, disposées à souffrir
la perte avec patience, j'ai remarqué
que cette résignation se trouvait plus
chez les sots que chez les gens d'es%
prit, dont l'imagination est plus facile
à s'exalter, quoique la réflexion et la
philosophie les modèrent ensuite.

Dussaulx a trop confondu les rap%
ports sous lesquels le jeu peut être
considéré : il devait sans doute pré%
senter séparément l'influence qu'il a
sur les mœurs et sur la fortune pu%
blique, et celle qu'il a sur les mœurs
et sur la fortune des particuliers ;
mais il revient trop fréquemment aux
mêmes idées ; ce qu'on peut attribuer
au défaut d'ensemble de son ouvrage.
\folio{13}
Il me parait au moins que les parties
en sont trop détachées ; que ses ta%
bleaux ne sont pas liés de manière à
soutenir l'intérêt ; que ses raisonne%
mens ne sont pas assez suivis pour
porter dans les esprits cette convic%
tion dont plus de conversions et de
réformes auraient été les résultats.
Sa marche est quelquefois embarras%
sée : on voit qu'il veut dire tout ce
qu'il sait, tout ce qu'il a lu sur les
jeux ; et on peut lui appliquer ce qu'il
a dit de Barbeyrac : \frquote{II s'est jeté trop
souvent dans des discussions super%
flues ou étrangères à son sujet ;}
de manière qu'on jugerait difficile%
ment s'il a écrit pour les joueurs ou
pour ceux qui ne le sont pas.

La plupart de ces défauts, cet em%
barras, ces contradictions ne s'ex%
\folio{14}
pliquent-ils pas d'abord par l'incon%
venance de présenter à-la-fois, dans
un ouvrage moral, ce qui devait ne
contenter que la curiosité, avec ce
qui devait servir à l'instruction ? En
outre, l'espèce de nécessité dans la%
quelle l'auteur s'était mis de rendre
odieux et méprisables les joueurs et
les maisons de jeu, l'obligeait de s'é%
carter de tems en tems de la vérité, à
laquelle la franchise de son caractère
le ramenait bientôt.

Quant aux remèdes, Dussaulx,
après avoir cité une foule de traits
qui semblent en montrer l'ineffica%
cité ; après avoir dit qu'on n'a rien à
attendre des Gouvernemens, \frquote{tou%
jours si pauvres, qu'on ne saurait
compter sur eux, lorsqu'il s'agit
d'argent,} ni sur l'experience, qui
\folio{15}
appartient plus à ceux qui méditent
qu'à ceux qui agissent, et qui est tou%
jours impuissante contre le désir et la
séduction, indique cependant quel%
ques moyens de réforme, tels que les
amusemens naturels, la suspension
de la pratique du jeu par un violent
effort sur soi-même, l'exercice de la
bienfaisance, des entreprises labo%
rieuses, le recours dans le sein d'une
sage et prudente amitié. Voilà ce qu'il
conseille aux joueurs eux-mêmes.
Sans doute ces moyens sont salutaires,
et leur indication seule prouve la
bonté, la candeur d'ame de l'auteur,
dans l'ouvrage de qui je ne relève qu'à
regret, parmi des beautés et des vé%
rités du premier ordre, ce que je
regarde comme des défauts ou des
erreurs ; mais plusieurs de ces moyens
\folio{16}
ne sont-ils pas hors du pouvoir des
uns ? Ne sont-ils pas insuffisans pour
les autres ? Et qui donnera aux joueurs
la force d'exécution sans laquelle la
force de volonté n'est rien ? Une
grande passion se guérit-elle avec des
calmans ? Et si ce joueur croit voir
dans ce qu'il va faire un avantage sûr
et prochain, l'en détournerez-vous
en lui offrant, dans ce que vous lui
proposez, un avantage douteux et
éloigné ?

Dussaulx expose encore quelques
autres moyens de réforme ; mais
ceux-ci il les fait dépendre de l'auto%
rité. Ce sont les refus d'honneurs et
d'emplois aux joueurs incorrigibles,
l'obligation aux joueurs fortunés de
nourrir des vieillards, des pauvres,
des infirmes \paren{je n'ai pas besoin de
\folio{17}
dire quels maux pourraient résulter
de l'erreur ou de la mauvaise foi de
rapports faits à l'autorité sur la con%
duite des particuliers qui d'ailleurs
auraient tant de moyens de dérober
à ses agens la connaissance de leurs
gains ou de leurs excès}, la suppres%
sion des jeux d'état, l'abolition des
privilèges de jeux, la réformation des
mœurs, l'éducation.

Les lecteurs ayant de l'expérience
et de l'instruction, apprécieront, je
ne dirai pas les remèdes découverts,
mais les vœux formés par Dussaulx :
j'y reviendrai en m'occupant aussi
des moyens praticables de réforme
ou d'amélioration.

J'ai rempli un devoir pénible en 
manifestant mon opinion sur l'insuf%
fisance de l'ouvrage de Dussaulx con%
\folio{18}
tre la passion et les excès du jeu. Je
le répète, je n'en reconnais pas moins
le mérite supérieur de cet ouvrage,
et j'en recommande vivement la lec%
ture aux personnes malheureuses ou
trompées que le jeu entraîne ou sé%
duit. Je n'ai point l'orgueilleuse pré%
tention de le refaire ; mais en traitant
beaucoup moins d'objets, et me ren%
fermant dans un cadre étroit, je veux
rechercher s'il n'y aurait pas de nou%
velles lumières à répandre sur ce
sujet. D'ailleurs, l'état des jeux n'est
pas aujourd'hui ce qu'il était lorsque
Dussaulx a écrit.
