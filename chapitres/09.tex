\chapter
  [De l'action du Gouvernement sur les Jeux. Des lois prohibitives]
  {De l'action du gouvernement sur les jeux : des lois prohibitives}

\lettrine{L}{'ordre} public, l'intérêt des mœurs
et la sureté des fortunes, rendent
nécessaire l'action du gouvernement
sur les jeux de hasard, en quelques
lieux qu'ils se jouent.

Observons qu'il ne s'agit point de
ces simples amusemens, qui se con%
centrant dans le sein des familles et
des amis, sont hors du domaine de
l'autorité : on doit sans doute en être
affranchi lorsqu'on ne fait rien qui
porte préjudice à la chose commune
et aux droits d'autrui ; et les parti%
culiers qui ne s'écartent pas de ces
\folio{121}
justes règles, peuvent se livrer li%
brement à leurs goûs, à leurs vo%
lontés, même à leurs passions ; mais
ici l'abus est si près de la chose, les
infidélités sont si aisées à commettre
et auraient de telles conséquences ;
tant de désordres sont nés de ces
sources impures, que le gouverne%
ment, gardien et conservateur de la 
morale et de la fortune publique,
trahirait un de ses principaux de%
voirs, s'il se montrait étranger à des
actions qui ont sur elles une si grande
influence. C'est ici sur-tout que des
étincelles inaperçues ou négligées
causeraient un violent incendie.

Je lis dans Barbeyrac, un des écri%
vains qui ont traité les joueurs avec
le plus d'indulgence : \frquote{La faculté
d'avoir à point nommé le moyen
\folio{122}
de satisfaire un désir innocent,
mais sujet à mener au crime, est
une tentation très-dangereuse, et
contre laquelle on ne saurait trop
se précautionner.}

Mais quelle sera la nature de cette
action du gouvernement sur les jeux
de hasard ?

Il me semble qu'elle doit être l'exer%
cice habituel d'une surveillance ac%
tive, et que, s'adaptant aux localités
et aux circonstances, elle doit admet%
tre plutôt des règles particulières d'or%
dre et des mesures de répression, que
des lois générales et prohibitives.

Je m'expliquerai sur le danger que
je vois dans ces lois, sur-tout sur
celui que je crois qu'elles auraient
aujourd'hui, et je dirai pourquoi je
regarde comme nécessaire que le
\folio{123}
Gouvernement étende dans tous les
lieux son action sur les jeux de ha%
sard, lorsque j'aurai fait connaître
l'état actuel de ces jeux, soit dans
les maisons publiques, soit dans les
maisons particulières. Je me conten%
terai de jeter ici un coup-d'œil sur
une partie des lois et des réglemens
de cette nature, portés contre les jeux
sous différens règnes, et sur les effets
qu'elles ont produits.

Chez les Grecs, les joueurs étaient
flétris, et il était enjoint aux citoyens
de dénoncer ceux qui jouaient furti%
vement.

A Rome, il y eut un sénatus-con
sulte qui ne défendit de jouer de l'ar%
gent qu'aux jeux qui avaient pour
objet l'exercice du corps et qui étaient
utiles pour la guerre. Il n'était per%
\folio{124}
mis d'y jouer que son écot dans un
festin, ou des raffraîchissements. De%
puis, Charles~IX, Roi de France,
voulut qu'on ne jouât que des oublis ;
et un duc de Savoie, que des épin%
gles.

Chez les Romains, dont on se plaît
tant à citer les lois, quiconque don%
nant à jouer perdait le droit de citoyen,
et restait à la merci de ceux à qui il
avait gagné de l'argent.

Sous Cicéron, ceux qui étaient
reconnus pour joueurs, n'étaient pas
admis à se plaindre des insultes qu'on
leur faisait ou du dommage qu'on leur
causait.

Les pères de l'Eglises et les Conciles
ont constamment lancé leurs foudres
contre le jeu. Le cardinal Pierre Da%
mien, au onzième siècle, condamna
\folio{125}
un évêque de Florence, pour avoir
joué dans une auberge, à réciter
trois fois de suite le Psautier, à laver
les pieds de douze pauvres, et à leur
compter un écu par tête.

Justinien ordonna qu'on ne pût
jouer plus d'un écu par partie. Il
accorda pendant cinq années le droit
de réclamer juridiquement contre les
gains des joueurs ; et lorsque personne
ne se présentait, le trésor public pro%
fitait de la confiscation.

Les Rois de France, d'Espagne,
d'Angleterre, et tous les potentats
de l'Europe ont sévi contre les jeux.
Charlemagne, Louis le Débonnaire,
S.~Louis et Charles~V, se sont signa%
lés dans cette tâche honorable, mais
difficile.

Pour seconder les intentions de
\folio{126}
Charles~V, le prévôt de Paris rendit,
en janvier~1397, une ordonnance
dans laquelle il déclarait qu'en inter%
rogeant les criminels, il avait décou%
vert que la plupart des crimes ve%
naient du jeu.

Charles~VIII, par une ordonnance
de 1485, permit seulement aux per%
sonnes de distinction arrêtées pour
des causes légères, de jouer au tric-
trac et aux échecs.

En 1532, François~I\ier, instituteur
d'une loterie qui ne fut pas tirée,
parce que le peuple ne fut pas dupe
des motifs qui la faisaient créer, se
contenta, par un édit donné à Châ%
teaubriant en 1532, de condamner
quiconque jouerait contre des comp%
tables, à restituer le double de ce
qu'il leur aurait gagné.

\folio{127}
Louis~XIII ne fut pas plutôt sur
le trône, qu'il fit une déclaration
énergique contre les brelans, les aca%
démies de jeu, etc. Il déclara \emph{infa%
mes}, intestables, et incapables de
tenir jamais offices royaux, quicon%
que se livrait aux jeux de hasard.
Il voulut en outre que l'argent et les
effets mis au jeu, fussent saisis au
profit des pauvres. Cette déclaration
fut suivie de plusieurs ordonnances
sur le même objet.

Sous Louis~XIV, plus de vingt
ordonnances, déclarations ou édits,
furent publiés contre les jeux de
hasard. \emph{La bassette et le hoca} furent
sur-tout défendus sous les peines les 
plus graves.

Sous Louis~XV, les permissions
de jeux éprouvèrent seulement quel%
ques restrictions ; et ce qu'on a le
plus remarqué, est une ordonnance
rendue le 6~mai~1760, par les maré%
chaux de France, portant qu'ils n'au%
raient aucun égard aux demandes 
qui leur seraient adressées pour des
créances procédant de pertes faites
au jeu, excédant la somme de mille
livres.

Dans les Etats voisins de la France,
des lois de même nature ont été por%
tées contre le jeu.

En Angleterre, pour arrêter ce
désordre dans les derniers rangs de
la société, Henri~VIII défendit aux
artisans, sous peine de prison, de se
livrer, excepté pendant les fêtes de
Noël, aux jeux qui de son tems étaient
en vogue.

Georges~III, par un statut qui con%
\folio{129}
firme cette défense, inglige les mêmes
peines à ceux qui donnent publique%
ment à jouer aux domestiques.

\frquote{Si quelqu'un, dit Charles~III, soit
en jouant, soit en pariant, perd
plus de cent livres, je le dispense
du paiement : je condamne son ad%
versaire à compter le triple de la
somme gagnée, moitié à la cou%
ronne, moitié au dénonciateur.}

La reine Anne déclare nuls et de
nul effet les billets, l'argent prêté,
et tous les engagemens contractés au
jeu : elle donne action au perdant
contre le gagnant, et, au défaut de
ce dernier, à quiconque voudra pour%
suivre le délit, adjugeant à celui-ci
le quintuple de la somme perdue. Ce
qu'il y a de plus remarquable, c'est
qu'elle permet à ceux qui sollicitent
\folio{130}
la confiscation des gains faits au jeu,
de prendre à serment l'infracteur,
de quelque qualité qu'il soit, voulant
que les actions de cette nature sus%
pendent les privilèges des membres
du parlement. Si des joueurs infidèles
gagnent plus de dix livres, soit en
argent, soit en effets, elle les con%
damne à rendre le quintuple, les sou%
mettant d'ailleurs à des notes d'in%
famie et à des peines afflictives.

Georges~II condamna les moteurs
de différens jeux, par lesquels on
cherchait à éluder les défenses, à cinq
cents livres d'amende, leurs dupes à 
cinquante. Tout ce qui équivalait aux
loteries, comme le pharaon, la bas%
sette, etc., fut défendu par un grand
nombre de statuts.

Georges~II défendit aussi, sous peine
\folio{131}
de deux cents livres d'amende, de
parier plus de cinquante livres aux
courses de chevaux.

Le gouverneur de Rome, en~1776,
a rendu une ordonnance contre les
jeux de hasard.

Le roi de Prusse, en~1777, a re%
nouvelé les anciens édits contre les
joueurs. Ils étaient condamnés à trois
cents ducats d'amende ; faute de paie%
ment, ils devaient être détenus à la
forteresse de Landau pendant trois
mois, et n'y vivre que de pain et d'eau.

Au Japon, quiconque risque de
l'argent aux jeux de hasard, doit être
\emph{puni de mort!\ldots}

Je reviens aux lois rendues en
France.

Suivant l'article~15 du titre~19, et
l'art.~28 du titre~20 de l'ordonnance
\folio{132}
du roi du 1\ier~mars~1768, les officiers
généraux et commandans de place
sont tenus d'empêcher que les troupes
qui sont sous leurs ordres jouent aux
jeux de hasard.

Tout officier qui joue malgré cette
défense, doit être mis la première fois
en prison pour trois mois ; la seconde
fois pour six mois ; la troisième, il
doit être cassé et renfermé dans une
citadelle.

Suivant un autre article, les sol%
dats, cavaliers ou dragons tenant des
jeux défendus, doivent être condam%
nés suivant la rigueur des lois, et les
joueurs doivent subir quinze jours
de prison ; et suivant l'art.~16, un des
plus importans, puisqu'il concerne
toutes les classes de la société, les com%
mandans doivent s'informer quels ha%
\folio{133}
bitans donnent à jouer aux jeux dé%
fendus, les faire arrêter et punir sui%
vant l'exigeance des cas.

Par arrêt du 16~décembre~1780,
le parlement de Paris a défendu les
jeux de hasard et les académies de jeu,
à peine de trois mille livres d'amende.

Il avait déjà ordonné l'exécution
des anciens arrêts et ordonnances sur
les mêmes objets.

Le 1\ier mars 1781, le roi en a aussi
rappelé les dispositions, et en a or%
donné l'exécution rigoureuse.

Sont réputés prohibés les jeux dont
les chances sont inégales, et qui pré%
sentent des avantages certains à l'une
des parties au préjudice de l'autre : les
banquiers condamnés et par corps, en
trois mille livres d'amende, les joueurs
en mille livres ; après deux condam%
\folio{134}
nations, punis de peines afflictives
et infamentes.

Presque toutes ces lois ont déclaré
nuls les billets, promesses et autres
actes ayant le jeu pour cause.

L'assemblée nationale, par un dé%
cret du 22~juillet~1791, a formé le der%
nier état de la jurisprudence des jeux.

Les jeux de hasard où l'on admet ou
le public ou les affiliés, sont défendus.

Des amendes sont prononcées contre
les propriétaires ou principaux loca%
taires des maisons de jeu qui n'ont
pa averti la police : elles sont la pre%
mière fois de 200~liv., la seconde de 
1,000 livres.

Les officiers de police peuvent en%
trer en tout tems dans les maisons de
jeu, ur la désignation donnée par
deux citoyens domiciliés.

\folio{135}
Suivant l'art.~36 du titre~2 du Code
de police correctionnelle, les teneurs
de maison où le public est admis,
sont punis d'une amende de 1,000 à
3,000~liv., de la confiscation des fonds
trouvés exposés au jeu, et d'un em%
prisonnement qui ne peut excéder une
année. En cas de récidive, l'amende
est de 5,000 à 10,000~liv., et l'em%
prisonnement peut être de deux an%
nées.

Les teneurs de jeux pris en flagrant
délit, doivent être arrêtés et conduits
devant le juge-de-paix.

Voilà beaucoup de lois : que rap%
porte-t-on des effets qu'elles ont
produits ? et quel sont ceux que nous
avons vus nous-mêmes ?

On a déjà pu s'en faire quelqu'idée,
par ce que j'ai dit des progrès de la
\folio{136}
fureur du jeu chez les peuples où ces
lois ont été portées.

A Athènes et à Rome, lorsque
l'aéropage et le sénat se signalaient
de part et d'autre par la censure des
vices, les magistrats en donnaient
eux mêmes l'exemple.

Les Grecs, pour éviter les dénon%
ciations, partaient d'Athènes et al%
laient à Scyros dans le temps de
Minerve.

Le clergé, qui s'est tant déchaîné
contre le jeu, a le plus participé à
ses excès.

Les grands vassaux, la noblesse et
le clergé rendaient nulles toutes les
mesures prises par Charles~V pour
enchaîner le jeu et les joueurs.

Les lois de St.-Louis, de Charles~V,
de Henri~VIII, de la reine Anne, etc.,
\folio{137}
se sont abolies, parce que la difficulté
ou l'impossibilité de leur exécution
les a fait négliger.

Le frère de Saint-Louis bravait, en
jouant, des lois impuissantes.

Les grands seigneurs, sous Louis~%
XIII, r'ouvrirent les jeux que ce roi
était parvenu à faire fermer. Les lois
ne portèrent que sur quelques plé%
béiens obscurs. Les joueurs, compri%
més quelque tems, devinrent bientôt
plus effrénés et plus nombreux.

Sous le règne suivant, pour éluder 
impunément les termes de la loi, il a
suffi de déguiser les jeux prohibés sous
d'autres noms et sous d'autres for%
mes.

\emph{On se cachait}, dit Dussaulx, \emph{mais
on n'en jouait que plus gros jeu.}

Les princes, les ministres, fatigués
\folio{138}
de contredire \emph{un penchant si général,}
ont fait grace à cette antique manie.
Souvent ils en ont fait un objet de
spéculation.

Suivant Blackston, qui a fourni à
Dussaulx ce que j'ai cité des lois ren%
dues en Angleterre pour la repres%
sion des jeux, les joueurs anglais ont
aussi trouvé le moyen de se soustraire
aux châtimens. Les jeux, les loteries
se multipliaient impunément dans les
tems même où l'on paraissait le plus 
sévir contre eux.

Qu'on n'oublie point le trait que
j'ai rapporté du vieux et plus sage
magistrat du parlement de Bordeaux,
dont la réputation ne souffrait pas de 
ce qu'il risquait au jeu, dans une
soirée, tout ce qu'il possédait : et le
parlement de Bordeaux s'était dis%
\folio{139}
dingué par des ordonnances contre
les jeux !

Lorsque les vingt déclarations ou
ordonnances de Louis~XIV eurent
été portées, les trois quarts de la na%
tion ne respirèrent plus qu'après le
jeu : et il est bon de remarquer que
c'est toujours à la suite des lois pro%
hibitives, que les jeux ont repris le
plus d'activité.

Comme les différens symptômes
d'immoralité et de corruption se ma%
nifestent à la fois, tandis que le sys%
tème de \emph{Jean-Law} se propageait,
des ministres, des magistrats, loin de
faire exécuter les lois contre les jeux
publics, les permirent généralement.
Qu'on ne s'étonne donc pas de la
prospérité des jeux dans les hôtels de
Gêvres et de Soissons !

\folio{140}
Comprimait-on réellement les
joueurs, ils s'expatriaient : ainsi des
Vénitiens, à l'exemple des joueurs
émigrés d'Athènes et retirés à Scyros,
se sont réfugiés en France et en An%
gleterre, pour y savourer sans con%
trainte l'horrible volupté des jeux.

S'il y avait à l'époque où Dussaulx
a écrit son livre sur le jeu, \emph{cent mai%
  sons connues où l'on se ruinait tous
  les jours, et dix fois plus de réduits
  subalternes où l'on se ruinait, que
  l'on n'en comptait sous Henri~IV,
  sous Louix~XIV, et du tems de la
régence,} on est dispensé de recher%
cher davantage si les lois prohibitives
sur les jeux ont des effets salutaires.
Ce que nous avons vu après que l'as%
semblée nationale eut aussi participé
à la gloire de faire des lois aussi bril%
\folio{141}
lantes en motifs que stériles en résul%
tats, n'achève-t-il pas d'éclairer cette
question ?

Il est inutile d'observer que des lois
que j'ai citées, les unes sont ridicules
et bizarres, les autres injustes ou in%
convenantes ; plusieurs sont aussi
cruelles ou aussi immorales que les ha%
bitudes qu'elles tendaient à détruire ;
d'autres enfin, n'étaient que des pri%
vilèges exclusifs accordés à une classe
particulière d'hommes, pour qu'ils
pussent porter impunément, par
l'exemple du vice le plus séduisant, les
dangereuses tentations du jeu dans la
partie la plus saine et la plus utile du
corps social.
