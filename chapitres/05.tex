\chapter{De la Théorie des Jeux de hasard}

\folio{58}
\lettrine{C}{es} vœux, ces espérances qui gui%
dent et animent les joueurs dans la
poursuite des faveurs de la fortu%
ne, sont-ils donc sans fondement ?
La recherche de la vérité, les ef%
forts de la raison, les lumières, la
prudence, ne peuvent-ils procurer
des avantages certains dans les jeux
de hasard ? Voilà ce qu'il me convient
le plus d'examiner, car si des béné%
fices assurés peuvent être le résultat
de calculs et de combinaisons, la con%
duite des joueurs instruits est presque
justifiée ; et si ni lumières, ni calculs,
ne peuvent rendre le sort plus favo%
\folio{59}
rable aux joueurs qui les possèdent,
qu'à ceux qui ne les possèdent pas,
il résultera de cette certitude, ou de
cette véritée démontrée, que la pra%
tique du jeu dans l'espérance du gain,
est une folie à la fois la plus ridicules
et des plus funestes.

Il est incontestable que, dans les
jeux mêlés de science et de hasard, un
joueur peut avoir pour lui la faveur
des chances.

« Un joueur habile, dit l'abbé Du%
bos, pourrait faire tous les jours un
gain certain, en ne risquant son
argent qu'aux jeux où le succès dé%
pend plus de l'habileté des tenans,
que du hasard des cartes et des
dés ».

Aussi c'est à de pareils jeux que se
livrent le plus ceux qu'on nomme
\folio{60}
\emph{joueurs de métier}, ou ces hommes
dits de \emph{bonne compagnie}, qui veulent
au moins tirer parti de leur réputa%
tion. \paren{Il ne manque à cette conduite
que la moralité}. C'est pour l'instruc%
tion d'élèves dans ces jeux qu'on a
multiplié les méthodes et les traités.

Les gens honnêtes qui aiment à in%
téresser leur jeu, font donc très-bien,
s'ils ne veulent pas être dupes, de ne
s'engager que dans des parties où la
probité et l'égalité de talent puissent
au moins être présumées.

Quant aux jeux de pur hasard, pour
savoir s'il y a une manière plus ou
moins avantageuse d'y engager son
argent, indépendamment de ce qu'on
livre aux profits de la banque, je n'au%
rais pas eu besoin de recourir à des
études de mathématiques, et de consul%
\folio{61}
ter les auteurs qui ont écrit sur cette
matière ; la seule dénomination de
\emph{jeux de hasard} suffit pour dispenser 
d'une pareille recherche ; elle exclut
toute idée de science et de calculs ;
mais j'ai été curieux de connaître par
quels moyens on était parvenu à faire
croire qu'il était possible de fixer l'in%
constance du sort, et de donner l'exis%
tence au néant.

J'ai donc jeté un coup-d'œil sur
quelques écrits des auteurs qu'on cite
le plus en faveur de cette singulière
doctrine. Nommer les célèbres mathé%
maticiens Pascal, Bernouilli, Mont-%
Mort, d'Alembert, Fermat, Euler,
Ozanam, c'est dire que le charlata%
nisme avec lequel on annonce la so%
lution de leurs problèmes, ne peut
être mis que sur le compte de leurs
\folio{62}
éditeurs. Les écrits de peu de ces au%
teurs ont ce caractère : on trouve
dans leurs ouvrages des recherches
savantes et curieuses sur les jeux :
ils ont fait la décomposition et l'ana%
lyse des plus connus, et en ont pré%
senté les résultats ; mais il est aisé de
se convaincre qu'ils n'ont eu pour
but, dans leurs travaux, que d'éta%
blir des rapports du joueur avec le
joueur dans des chances inégales, ou
du joueur avec le banquier dans les
avantages accordés à celui-ci. Pour
cela, ils ont fait l'énumération des
différentes combinaisons résultantes
des différentes manières de faire ses
mises et ses paris ; ils ont déterminé
les proportions dans lesquelles les
paiemens devaient se faire ; ils ont dit
ce qui rendait égales ou inégales les
\folio{63}
conditions des joueurs dans les di%
verses conventions du jeu, ils ont
même indiqué, d'après des mises
faites, les degrés de probabilités de la
perte et du gain.

Voici, par exemple, un des pro%
blèmes qu'ils exposent.

« Lorsqu'on jour à croix ou pile,
la probabilité que croix ou pile arri%
vera sur un coup, est égale à ½. Il y a
également à parier pour ou contre ;
mais si l'on joue deux coups, et que
quelqu'un parie un écu d'amener
croix dès les deux coups, quelle somme
son adversaire doit-il mettre au pari,
pour que la condition des joueurs soit
égale » ?

Voici un autre de leurs problèmes :

« Dans une partie liée, un des
joueurs a gagné une partie ; on pro%
\folio{64}
pose de quitter ; la mise de chaque
joueur est d'un écu : quel est le droit
sur le fond du jeu de celui qui, sur
trois parties, a gagné la première » ?

De problèmes simples et facile à
résoudre, les mathématiciens passent
à d'autres plus difficiles ; les chances
se varient, se multiplient : les combi%
naisons se compliquent, et vous vous
égarez dans un labyrinthe de calculs,
si vous quittez un instant la ligne que
vous tracée l'esprit d'analyse.

Peu de joueurs sont capables de se
livrer à des études aussi abstraites ; et,
ne nous y trompons pas, le seul fruit
qu'on peut en retirer est d'apercevoir
dans le jeu ou dans la condition des
joueurs, c'est-à-dire, dans les mises et
dans les paiemens, des disproportions
dont ils seraient dupes, ou enfin de
\folio{65}
connaître les moyens de diminuer à
plusieurs de ces jeux, l'avantage du
banquier.

Mais comme en général les jeux qui
se jouent par entreprise sont calculés
de manière que le plus grand avantage
est toujours pour le banquier, la con%
naissance de cette vérité ne peut être
utile qu'à ceux qu'elle porte à renon%
cer à ces jeux ; car lorsque votre mise
est faite, à quoi vous servirait de sa%
voir que si telle carte sort, ou si les
dés donnent tel nombre, ou si la boule
tombe sur tel numéro, telle devra être
votre condition comparativement à
celle de la banque ? Irez-vous repro%
cher au banquier de ne pas payer 
votre gain en proportion de la perte à
laquelle vous étiez exposé ?

Il est bon cependant de savoir dis%
tinguer les inégalités de chances qui
résultent de la nature même du jeu,
indépendamment de l'avantage du 
banquier, et celles qui ne résultent
que de cet avantage, car alors on peut
ne jouer qu'aux jeux qui n'ont que
cette défaveur, ou on profite de la 
connaissance qu'on a des autres iné%
galités pour rendre son sort préfé%
rable à celui de son adversaire.

Les jeux qui ont d'autres chances
inégales que celles qui font l'avantage
de la banque, ne sont point des jeux
de pur hasard : c'est dans les sociétés
particulières que ces autres jeux se
jouent le plus et font le plus de vic%
times ; mais je considère ici que
les jeux de hasard, tels qu'ils se jouent
dans les maison de jeu.

L'avantage de la banque étant dif%
\folio{66}
férent pour les différens jeux, c'est
aux joueurs, s'ils sont susceptibles
de prudence, à préférer ceux qui
donnent à la banque le moindre avan%
tage ; mais cet avantage que néces%
sitent pourtant les dépenses et les 
frais d'administration, est désastreux
pour les personnes qui jouent fré%
quemment ou long-tems de suite. En
effet, je suppose que l'entreprise des
jeux ait le droit de deux pour cent 
sur l'argent exposé au tapis, en pre%
nant le terme moyen de son avan%
tage aux différens jeux qu'elle fait
jouer, celui qui joue un écu de cinq
francs chaque coup, après cinquante
coups, a nécessairement donné son
écu à l'entreprise ; et à ces jeux cin%
quante coups sont joués en bien peu
de tems.

\folio{68}
J'invite les joueurs à porter toute
leur attention sur cette vérité simple
et aussi facile à saisir que cette autre
vérité de calcul, \emph{un et un font deux.}
Elle est seule l'arrêt de leur ruine ; et
ce triste résultat de l'avantage inévi%
table de la banque, est ce que pré%
sente de plus clair et de plus certain
la solution des problèmes des célèbres
mathématiciens que j'ai nommés.

On chercherait inutilement dans
leurs écrits des règles et des méthodes
pour obtenir un gain assuré, même
des probabilités de gain aux jeux de 
hasard, dont les chances sont égales.

Cependant on voit de tems en tems
paraître de petits livrets dont le titre
annonce qu'on a enfin trouvé le
moyen d'enchaîner le sort aux jeux
de hasard, sur-tout à la roulette et
\folio{69}
au trente-un. Ceux de ces ouvrages
qui ne vous offrent pas des certi%
tudes de gain, vous offrent au moins
des probabilités.

Ces petits livrets sont écrits avec
une risible assurance : tout y est en
assertions, rien en preuves. On y in%
dique, par des colonnes de chiffres,
la manière de faire des mises ; on vous
dit gravement que tel numéro amène
tel autre ; on vous conseille de ne pas
mettre sur telle chance, parce qu'elle
est ingrate ; et on vous met à portée
de vérifier la justesse de ses combi%
naisons, en vous renvoyant à des ta%
bleaux qui contiennent la série des
numéros sortis ou des cartes tirées
pendant de nombreuses séances, où
l'on a eut la patience de les recueillir,
au lieu de les écrire chez soi en un
\folio{70}
instant, ce qui serait revenu au même.

Il n'est pas besoin de dire que le
style de ces petites brochures répond
à la justesse et à la lucidité des idées.
On reproche à leurs auteurs de ne
point mettre à profit, pour eux-%
mêmes, ces secrets de fortune ; mais
pour ne point trop m'écarter du ton
qui convient à mon sujet, j'avoue
que, soupçonnant à ces prétendus
écrivains plus de besoins que de ma%
lice, je crois qu'ils sont moins dignes
de mépris que de pitié.

Je déclare ici que les joueurs qui
fondent des soupçons ou des espé%
rances sur des inégalités ou des er%
reurs, dans la combinaison de chan%
ces des jeux de hasard auxquels font
jouer les banques publiques, se trom%
pent également : l'usage n'a consacré,
\folio{71}
pour ainsi dire, ces jeux, que parce
qu'ils sont calculés dans une exacte
proportion, et indépendamment de
ce qui forme l'avantage de la banque,
qui peut en être facilement distrait.
Ils sont, s'il m'est permis de hasarder
cette définition, la mise en action de
vérités de calcul mathématiquement
démontrées. Séparez-en l'avantage de
la banque, je donne publiquement le
défi aux plus habiles calculateurs de
prouver qu'aucune manière d'enga%
ger son argent, que des mises faites
dans un ordre progressif ou diminutif,
simples ou compliquées sur diffé%
rentes chances, puissent rompre cette
égalité, et donner aux joueurs le moin%
dre avantage.

« Les conjectures des joueurs, dit
Dussaulx, portent sur le néant ».
\folio{72}
Des listes de perte et de gain, faites
par des joueurs, avaient convaincu
cet excellent observateur, qu'aux
jeux de hasard il n'y a point de for%
tunes qui ne s'épuisent en peu de
tems.

Lorsqu'il reste aux hommes quel%
que raison, ils doivent s'en servir
pour régler leurs actions sur ce qui
leur paraît utile ou préférable.

Que les joueurs tâchent donc de
donner à la réflexions sur la nature et
les effets du jeu, une faible partie du
tems qu'ils destinaient à sa pratique !
Que sur-tout ils ne soient pas dupes
des noms ! On parle dans l'Encyclo%
pédie méthodique d'un jeu de dés ap%
pelé \emph{la parfaite égalité.} A ce jeu, six
chances, qui sont celles de raffles,
font perdre le banquier : et quatre-%
\folio{73}
vingt-dix autres, qui sont les dou%
blets, le font gagner.

Ainsi, dans le cours de deux cent
seize coups, où les pontes auront mis
un écu sur chaque case, le ban%
quier devra, toutes choses égales,
perdre trente écus, et en gagner
quatre-vingt-dix.

Son bénéfice sera donc de soixante
écus. Qu'on juge de la chose par le 
nom !
